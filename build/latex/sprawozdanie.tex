%% Generated by Sphinx.
\def\sphinxdocclass{report}
\documentclass[letterpaper,10pt,polish]{sphinxmanual}
\ifdefined\pdfpxdimen
   \let\sphinxpxdimen\pdfpxdimen\else\newdimen\sphinxpxdimen
\fi \sphinxpxdimen=.75bp\relax
\ifdefined\pdfimageresolution
    \pdfimageresolution= \numexpr \dimexpr1in\relax/\sphinxpxdimen\relax
\fi
%% let collapsible pdf bookmarks panel have high depth per default
\PassOptionsToPackage{bookmarksdepth=5}{hyperref}

\PassOptionsToPackage{booktabs}{sphinx}
\PassOptionsToPackage{colorrows}{sphinx}

\PassOptionsToPackage{warn}{textcomp}
\usepackage[utf8]{inputenc}
\ifdefined\DeclareUnicodeCharacter
% support both utf8 and utf8x syntaxes
  \ifdefined\DeclareUnicodeCharacterAsOptional
    \def\sphinxDUC#1{\DeclareUnicodeCharacter{"#1}}
  \else
    \let\sphinxDUC\DeclareUnicodeCharacter
  \fi
  \sphinxDUC{00A0}{\nobreakspace}
  \sphinxDUC{2500}{\sphinxunichar{2500}}
  \sphinxDUC{2502}{\sphinxunichar{2502}}
  \sphinxDUC{2514}{\sphinxunichar{2514}}
  \sphinxDUC{251C}{\sphinxunichar{251C}}
  \sphinxDUC{2572}{\textbackslash}
\fi
\usepackage{cmap}
\usepackage[T1]{fontenc}
\usepackage{amsmath,amssymb,amstext}
\usepackage{babel}



\usepackage{tgtermes}
\usepackage{tgheros}
\renewcommand{\ttdefault}{txtt}



\usepackage[Sonny]{fncychap}
\ChNameVar{\Large\normalfont\sffamily}
\ChTitleVar{\Large\normalfont\sffamily}
\usepackage{sphinx}

\fvset{fontsize=auto}
\usepackage{geometry}


% Include hyperref last.
\usepackage{hyperref}
% Fix anchor placement for figures with captions.
\usepackage{hypcap}% it must be loaded after hyperref.
% Set up styles of URL: it should be placed after hyperref.
\urlstyle{same}

\addto\captionspolish{\renewcommand{\contentsname}{Spis treści:}}

\usepackage{sphinxmessages}
\setcounter{tocdepth}{1}



\title{Sprawozdanie}
\date{05 lip 2025}
\release{1.0.0}
\author{Szymon Piskorz}
\newcommand{\sphinxlogo}{\vbox{}}
\renewcommand{\releasename}{Wydanie}
\makeindex
\begin{document}

\ifdefined\shorthandoff
  \ifnum\catcode`\=\string=\active\shorthandoff{=}\fi
  \ifnum\catcode`\"=\active\shorthandoff{"}\fi
\fi

\pagestyle{empty}
\sphinxmaketitle
\pagestyle{plain}
\sphinxtableofcontents
\pagestyle{normal}
\phantomsection\label{\detokenize{index::doc}}


\sphinxstepscope


\chapter{1. Wprowadzenie}
\label{\detokenize{rozdzial1/rozdzial1:wprowadzenie}}\label{\detokenize{rozdzial1/rozdzial1::doc}}
\sphinxAtStartPar
Autor: Szymon Piskorz

\sphinxAtStartPar
Prowadzący: Piotr Czaja

\sphinxAtStartPar
Niniejsze sprawozdanie stanowi kompleksową dokumentację projektu bazodanowego, którego realizacja przyświecała dwóm głównym celom: dydaktycznemu oraz praktycznemu.


\section{Cel dydaktyczny}
\label{\detokenize{rozdzial1/rozdzial1:cel-dydaktyczny}}
\sphinxAtStartPar
Nadrzędnym celem było pogłębienie wiedzy i zdobycie praktycznych umiejętności z zakresu zaawansowanej administracji systemami baz danych. Proces ten obejmował badanie i aplikację następujących zagadnień:
\begin{itemize}
\item {} 
\sphinxAtStartPar
\sphinxstylestrong{Infrastruktura sprzętowa i konfiguracja:} Analiza wymagań sprzętowych oraz optymalna konfiguracja parametrów serwera bazy danych pod kątem wydajności i stabilności.

\item {} 
\sphinxAtStartPar
\sphinxstylestrong{Kontrola, konserwacja i diagnostyka:} Poznanie narzędzi do monitorowania stanu bazy, diagnozowania wąskich gardeł oraz wdrażanie procedur konserwacyjnych, takich jak aktualizacja statystyk czy reindeksacja.

\item {} 
\sphinxAtStartPar
\sphinxstylestrong{Wydajność, skalowanie i replikacja:} Techniki optymalizacji zapytań, strategie skalowania (pionowego i poziomego) oraz konfiguracja mechanizmów replikacji w celu zapewnienia wysokiej dostępności i rozłożenia obciążenia.

\item {} 
\sphinxAtStartPar
\sphinxstylestrong{Partycjonowanie danych:} Zrozumienie i zastosowanie metod partycjonowania tabel w celu poprawy zarządzania dużymi zbiorami danych i zwiększenia wydajności zapytań.

\item {} 
\sphinxAtStartPar
\sphinxstylestrong{Bezpieczeństwo:} Implementacja polityk bezpieczeństwa, zarządzanie użytkownikami i uprawnieniami, a także ochrona przed nieautoryzowanym dostępem.

\item {} 
\sphinxAtStartPar
\sphinxstylestrong{Kopie zapasowe i odzyskiwanie danych:} Projektowanie i automatyzacja strategii tworzenia kopii zapasowych (pełnych, różnicowych, przyrostowych) oraz procedur odtwarzania danych po awarii.

\end{itemize}


\section{Cel praktyczny i zakres projektu}
\label{\detokenize{rozdzial1/rozdzial1:cel-praktyczny-i-zakres-projektu}}
\sphinxAtStartPar
Drugim celem była realizacja kompletnego projektu bazy danych o nazwie „Karnety na siłowni”. Projekt ten ilustruje pełen cykl życia produktu bazodanowego, od analizy wymagań, przez modelowanie, aż po wdrożenie i analizę.


\section{Wykorzystane technologie}
\label{\detokenize{rozdzial1/rozdzial1:wykorzystane-technologie}}\begin{itemize}
\item {} 
\sphinxAtStartPar
\sphinxstylestrong{System Bazy Danych:} PostgreSQL, SQLite

\item {} 
\sphinxAtStartPar
\sphinxstylestrong{Język skryptowy:} Python 3.11 (z biblioteką \sphinxtitleref{psycopg2})

\item {} 
\sphinxAtStartPar
\sphinxstylestrong{System dokumentacji:} Sphinx/Latex

\item {} 
\sphinxAtStartPar
\sphinxstylestrong{System kontroli wersji:} Git / GitHub

\end{itemize}


\section{Struktura sprawozdania}
\label{\detokenize{rozdzial1/rozdzial1:struktura-sprawozdania}}
\sphinxAtStartPar
Dokument został podzielony na pięć rozdziałów. Po niniejszym wprowadzeniu, rozdział drugi odnosi się do zrealizowanego projektu grupowego. Rozdział trzeci szczegółowo opisuje projekt bazy danych „Karnety na siłowni”, prezentując jego modele i procesy. Rozdział czwarty skupia się na analizie normalizacji, wydajności i bezpieczeństwa. Całość zamyka rozdział piąty, zawierający podsumowanie, wnioski oraz listę wykorzystanych repozytoriów.

\sphinxstepscope


\chapter{2. Projekt grupowy}
\label{\detokenize{rozdzial2/rozdzial2:projekt-grupowy}}\label{\detokenize{rozdzial2/rozdzial2::doc}}
\sphinxAtStartPar
Poniżej zamieszczono dokumentacje poszczególnych przeglądów literatury:

\sphinxstepscope


\section{Sprzęt dla baz danych}
\label{\detokenize{rozdzial2/Sprzet-dla-bazy-danych/source/SprzetDlaBazyDanych:sprzet-dla-baz-danych}}\label{\detokenize{rozdzial2/Sprzet-dla-bazy-danych/source/SprzetDlaBazyDanych::doc}}

\subsection{Wstęp}
\label{\detokenize{rozdzial2/Sprzet-dla-bazy-danych/source/SprzetDlaBazyDanych:wstep}}
\sphinxAtStartPar
Systemy zarządzania bazami danych (DBMS) są fundamentem współczesnych aplikacji i usług \textendash{} od rozbudowanych systemów transakcyjnych, przez aplikacje internetowe, aż po urządzenia mobilne czy systemy wbudowane. W zależności od zastosowania i skali projektu, wybór odpowiedniego silnika bazodanowego oraz towarzyszącej mu infrastruktury sprzętowej ma kluczowe znaczenie dla zapewnienia wydajności, stabilności i niezawodności systemu


\subsection{Sprzęt dla bazy danych PostgreSQL}
\label{\detokenize{rozdzial2/Sprzet-dla-bazy-danych/source/SprzetDlaBazyDanych:sprzet-dla-bazy-danych-postgresql}}
\sphinxAtStartPar
PostgreSQL to potężny system RDBMS, ceniony za swoją skalowalność, wsparcie dla zaawansowanych zapytań i dużą elastyczność. Jego efektywne działanie zależy w dużej mierze od odpowiednio dobranej infrastruktury sprzętowej.


\subsubsection{Procesor}
\label{\detokenize{rozdzial2/Sprzet-dla-bazy-danych/source/SprzetDlaBazyDanych:procesor}}
\sphinxAtStartPar
PostgreSQL obsługuje wiele wątków, jednak pojedyncze zapytania zazwyczaj są wykonywane jednordzeniowo. Z tego względu optymalny procesor powinien cechować się zarówno wysokim taktowaniem jak i odpowiednią liczbą rdzeni do równoczesnej obsługi wielu zapytań. W środowiskach produkcyjnych najczęściej wykorzystuje się procesory serwerowe takie jak Intel Xeon czy AMD EPYC, które oferują zarówno wydajność, jak i niezawodność.


\subsubsection{Pamięć operacyjna}
\label{\detokenize{rozdzial2/Sprzet-dla-bazy-danych/source/SprzetDlaBazyDanych:pamiec-operacyjna}}
\sphinxAtStartPar
RAM odgrywa istotną rolę w przetwarzaniu danych, co znacząco wpływa na wydajność operacji. PostgreSQL efektywnie wykorzystuje dostępne zasoby pamięci do cache’owania, dlatego im więcej pamięci RAM tym lepiej. W praktyce, minimalne pojemności dla mniejszych baz to około 16\textendash{}32 GB, natomiast w środowiskach produkcyjnych i analitycznych często stosuje się od 64 GB do nawet kilkuset.


\subsubsection{Przestrzeń dyskowa}
\label{\detokenize{rozdzial2/Sprzet-dla-bazy-danych/source/SprzetDlaBazyDanych:przestrzen-dyskowa}}
\sphinxAtStartPar
Dyski twarde to krytyczny element wpływający na szybkość działania bazy. Zdecydowanie zaleca się korzystanie z dysków SSD (najlepiej NVMe), które zapewniają wysoką przepustowość i niskie opóźnienia. Warto zastosować konfigurację RAID 10, która łączy szybkość z redundancją.


\subsubsection{Sieć internetowa}
\label{\detokenize{rozdzial2/Sprzet-dla-bazy-danych/source/SprzetDlaBazyDanych:siec-internetowa}}
\sphinxAtStartPar
W przypadku PostgreSQL działającego w klastrach, środowiskach chmurowych lub przy replikacji danych, wydajne połączenie sieciowe ma kluczowe znaczenie. Standardem są interfejsy 1 Gb/s, lecz w dużych bazach danych stosuje się nawet 10 Gb/s i więcej. Liczy się nie tylko przepustowość, ale też niskie opóźnienia i niezawodność.


\subsubsection{Zasilanie}
\label{\detokenize{rozdzial2/Sprzet-dla-bazy-danych/source/SprzetDlaBazyDanych:zasilanie}}
\sphinxAtStartPar
Niezawodność zasilania to jeden z filarów bezpieczeństwa danych. Zaleca się stosowanie zasilaczy redundantnych oraz zasilania awaryjnego UPS, które umożliwia bezpieczne wyłączenie systemu w przypadku awarii. Można użyć własnych generatorów prądu.


\subsubsection{Chłodzenie}
\label{\detokenize{rozdzial2/Sprzet-dla-bazy-danych/source/SprzetDlaBazyDanych:chlodzenie}}
\sphinxAtStartPar
Intensywna praca serwera PostgreSQL generuje duże ilości ciepła. Wydajne chłodzenie powietrzne, a często nawet cieczowe jest potrzebne by utrzymać stabilność systemu i przedłużyć żywotność komponentów. W profesjonalnych serwerowniach stosuje się zaawansowane systemy klimatyzacji i kontroli termicznej.


\subsection{Sprzęt dla bazy danych SQLite}
\label{\detokenize{rozdzial2/Sprzet-dla-bazy-danych/source/SprzetDlaBazyDanych:sprzet-dla-bazy-danych-sqlite}}
\sphinxAtStartPar
SQLite to lekki, samodzielny silnik bazodanowy, nie wymagający uruchamiania oddzielnego serwera. Znajduje zastosowanie m.in. w aplikacjach mobilnych, przeglądarkach internetowych, systemach IoT czy oprogramowaniu wbudowanym.


\subsubsection{Procesor}
\label{\detokenize{rozdzial2/Sprzet-dla-bazy-danych/source/SprzetDlaBazyDanych:id1}}
\sphinxAtStartPar
SQLite działa lokalnie na urządzeniu użytkownika. Dla prostych operacji wystarczy procesor z jednym, albo dwoma rdzeniami. W bardziej wymagających zastosowaniach (np. filtrowanie dużych zbiorów danych) przyda się szybszy CPU. Wielowątkowość nie daje istotnych korzyści.


\subsubsection{Pamięć operacyjna}
\label{\detokenize{rozdzial2/Sprzet-dla-bazy-danych/source/SprzetDlaBazyDanych:id2}}
\sphinxAtStartPar
SQLite potrzebuje niewielkiej ilości pamięci RAM w wielu przypadkach wystarcza 256MB do 1GB. Jednak dla komfortowej pracy z większymi zbiorami danych warto zapewnić nieco więcej pamięci, czyli 2 GB lub więcej, szczególnie w aplikacjach desktopowych lub mobilnych.


\subsubsection{Przestrzeń dyskowa}
\label{\detokenize{rozdzial2/Sprzet-dla-bazy-danych/source/SprzetDlaBazyDanych:id3}}
\sphinxAtStartPar
Dane w SQLite zapisywane są w jednym pliku. Wydajność operacji zapisu/odczytu zależy od nośnika. Dyski SSD lub szybkie karty pamięci są preferowane. W przypadku urządzeń wbudowanych, kluczowe znaczenie ma trwałość nośnika, zwłaszcza przy częstym zapisie danych.


\subsubsection{Sieć internetowa}
\label{\detokenize{rozdzial2/Sprzet-dla-bazy-danych/source/SprzetDlaBazyDanych:id4}}
\sphinxAtStartPar
SQLite nie wymaga połączeń sieciowych \textendash{} działa lokalnie. W sytuacjach, gdzie dane są synchronizowane z serwerem lub przenoszone przez sieć (np. w aplikacjach mobilnych), znaczenie ma jakość połączenia (Wi\sphinxhyphen{}Fi, LTE), choć wpływa to bardziej na komfort użytkowania aplikacji niż na samą bazę.


\subsubsection{Zasilanie}
\label{\detokenize{rozdzial2/Sprzet-dla-bazy-danych/source/SprzetDlaBazyDanych:id5}}
\sphinxAtStartPar
W systemach mobilnych i IoT efektywne zarządzanie energią jest kluczowe. Aplikacje powinny ograniczać zbędne operacje odczytu i zapisu, by niepotrzebnie nie obciążać procesora i nie zużywać baterii. W zastosowaniach stacjonarnych problem ten zazwyczaj nie występuje.


\subsubsection{Chłodzenie}
\label{\detokenize{rozdzial2/Sprzet-dla-bazy-danych/source/SprzetDlaBazyDanych:id6}}
\sphinxAtStartPar
SQLite nie generuje dużego obciążenia cieplnego. W większości przypadków wystarczy pasywne chłodzenie w zamkniętych obudowach, lecz warto zadbać o minimalny przepływ powietrza.


\subsection{Podsumowanie}
\label{\detokenize{rozdzial2/Sprzet-dla-bazy-danych/source/SprzetDlaBazyDanych:podsumowanie}}
\sphinxAtStartPar
Zarówno PostgreSQL, jak i SQLite pełnią istotne role w ekosystemie baz danych, lecz ich wymagania sprzętowe są diametralnie różne. PostgreSQL, jako system serwerowy, wymaga zaawansowanego i wydajnego sprzętu: mocnych procesorów, dużej ilości RAM, szybkich dysków, niezawodnej sieci, zasilania i chłodzenia.
Z kolei SQLite działa doskonale na skromniejszych zasobach, stawiając na lekkość i prostotę implementacyjną.
Dostosowanie sprzętu do konkretnego silnika DBMS i charakterystyki aplikacji pozwala nie tylko na osiągnięcie optymalnej wydajności, ale też gwarantuje stabilność i bezpieczeństwo działania całego systemu.

\sphinxstepscope


\section{Sprawozdanie: Konfiguracja i Zarządzanie Bazą Danych}
\label{\detokenize{rozdzial2/Konfiguracja_baz_danych/Konfiguracja_baz_danych:sprawozdanie-konfiguracja-i-zarzadzanie-baza-danych}}\label{\detokenize{rozdzial2/Konfiguracja_baz_danych/Konfiguracja_baz_danych::doc}}\begin{quote}\begin{description}
\sphinxlineitem{Authors}\begin{itemize}
\item {} 
\sphinxAtStartPar
Piotr Domagała

\item {} 
\sphinxAtStartPar
Piotr Kotuła

\item {} 
\sphinxAtStartPar
Dawid Pasikowski

\end{itemize}

\end{description}\end{quote}


\subsection{1. Konfiguracja bazy danych}
\label{\detokenize{rozdzial2/Konfiguracja_baz_danych/Konfiguracja_baz_danych:konfiguracja-bazy-danych}}
\sphinxAtStartPar
Wprowadzenie do tematu konfiguracji bazy danych obejmuje podstawowe informacje na temat zarządzania i dostosowywania ustawień baz danych w systemach informatycznych. Konfiguracja ta jest kluczowa dla zapewnienia bezpieczeństwa, wydajności oraz stabilności działania aplikacji korzystających z bazy danych. Obejmuje m.in. określenie parametrów połączenia, zarządzanie użytkownikami, uprawnieniami oraz optymalizację działania systemu bazodanowego.


\subsection{2. Lokalizacja i struktura katalogów}
\label{\detokenize{rozdzial2/Konfiguracja_baz_danych/Konfiguracja_baz_danych:lokalizacja-i-struktura-katalogow}}
\sphinxAtStartPar
Każda baza danych przechowuje swoje pliki w określonych lokalizacjach systemowych, zależnie od używanego silnika. Przykładowe lokalizacje:
\begin{itemize}
\item {} 
\sphinxAtStartPar
\sphinxstylestrong{PostgreSQL}: \sphinxcode{\sphinxupquote{/var/lib/pgsql/data}}

\item {} 
\sphinxAtStartPar
\sphinxstylestrong{MySQL}: \sphinxcode{\sphinxupquote{/var/lib/mysql}}

\item {} 
\sphinxAtStartPar
\sphinxstylestrong{SQL Server}: \sphinxcode{\sphinxupquote{C:\textbackslash{}Program Files\textbackslash{}Microsoft SQL Server}}

\end{itemize}

\sphinxAtStartPar
Struktura katalogów obejmuje katalog główny bazy danych oraz podkatalogi na pliki danych, logi, kopie zapasowe i pliki konfiguracyjne.

\sphinxAtStartPar
\sphinxstylestrong{Przykład}: W dużych środowiskach produkcyjnych często stosuje się osobne dyski do przechowywania plików danych i logów transakcyjnych. Takie rozwiązanie pozwala na zwiększenie wydajności operacji zapisu oraz minimalizowanie ryzyka utraty danych.

\sphinxAtStartPar
\sphinxstylestrong{Dobra praktyka}: Zaleca się, aby katalogi z danymi i logami były regularnie monitorowane pod kątem dostępnego miejsca na dysku. Przepełnienie któregoś z nich może doprowadzić do zatrzymania pracy bazy danych.


\subsection{3. Katalog danych}
\label{\detokenize{rozdzial2/Konfiguracja_baz_danych/Konfiguracja_baz_danych:katalog-danych}}
\sphinxAtStartPar
Jest to miejsce, gdzie fizycznie przechowywane są wszystkie pliki związane z bazą danych, takie jak:
\begin{itemize}
\item {} 
\sphinxAtStartPar
Pliki tabel i indeksów

\item {} 
\sphinxAtStartPar
Dzienniki transakcji

\item {} 
\sphinxAtStartPar
Pliki tymczasowe

\end{itemize}

\sphinxAtStartPar
\sphinxstylestrong{Przykładowo}: W PostgreSQL katalog danych to \sphinxcode{\sphinxupquote{/var/lib/pgsql/data}}, gdzie znajdują się zarówno pliki z danymi, jak i główny plik konfiguracyjny \sphinxcode{\sphinxupquote{postgresql.conf}}.

\sphinxAtStartPar
\sphinxstylestrong{Wskazówka}: Dostęp do katalogu danych powinien być ograniczony tylko do uprawnionych użytkowników systemu, co zwiększa bezpieczeństwo i zapobiega przypadkowym lub celowym modyfikacjom plików bazy.


\subsection{4. Podział konfiguracji na podpliki}
\label{\detokenize{rozdzial2/Konfiguracja_baz_danych/Konfiguracja_baz_danych:podzial-konfiguracji-na-podpliki}}
\sphinxAtStartPar
Konfiguracja systemu bazodanowego może być rozbita na kilka mniejszych, wyspecjalizowanych plików, np.:
\begin{itemize}
\item {} 
\sphinxAtStartPar
\sphinxcode{\sphinxupquote{postgresql.conf}} \textendash{} główne ustawienia serwera

\item {} 
\sphinxAtStartPar
\sphinxcode{\sphinxupquote{pg\_hba.conf}} \textendash{} reguły autoryzacji i dostępu

\item {} 
\sphinxAtStartPar
\sphinxcode{\sphinxupquote{pg\_ident.conf}} \textendash{} mapowanie użytkowników systemowych na użytkowników PostgreSQL

\end{itemize}

\sphinxAtStartPar
\sphinxstylestrong{Przykład}: Jeśli administrator chce zmienić jedynie sposób autoryzacji użytkowników, edytuje tylko plik \sphinxcode{\sphinxupquote{pg\_hba.conf}}, bez ryzyka wprowadzenia niezamierzonych zmian w innych częściach konfiguracji.

\sphinxAtStartPar
\sphinxstylestrong{Dobra praktyka}: Rozdzielenie konfiguracji na podpliki ułatwia zarządzanie, pozwala szybciej lokalizować błędy i minimalizuje ryzyko konfliktów podczas aktualizacji lub wdrażania zmian.


\subsection{5. Katalog Konfiguracyjny}
\label{\detokenize{rozdzial2/Konfiguracja_baz_danych/Konfiguracja_baz_danych:katalog-konfiguracyjny}}
\sphinxAtStartPar
To miejsce przechowywania wszystkich plików konfiguracyjnych bazy danych, takich jak główny plik konfiguracyjny, pliki z ustawieniami użytkowników, uprawnień czy harmonogramów zadań.

\sphinxAtStartPar
Typowe lokalizacje to:
\begin{itemize}
\item {} 
\sphinxAtStartPar
\sphinxcode{\sphinxupquote{/etc}} (np. \sphinxcode{\sphinxupquote{my.cnf}} dla MySQL)

\item {} 
\sphinxAtStartPar
Katalog danych bazy (np. \sphinxcode{\sphinxupquote{/var/lib/pgsql/data}} dla PostgreSQL)

\end{itemize}

\sphinxAtStartPar
\sphinxstylestrong{Przykład}: W przypadku awarii systemu administrator może szybko przywrócić działanie bazy, kopiując wcześniej zapisane pliki konfiguracyjne z katalogu konfiguracyjnego.

\sphinxAtStartPar
\sphinxstylestrong{Wskazówka}: Regularne wykonywanie kopii zapasowych katalogu konfiguracyjnego jest kluczowe \textendash{} utrata tych plików może uniemożliwić uruchomienie bazy danych lub spowodować utratę ważnych ustawień systemowych.


\subsection{6. Katalog logów i struktura katalogów w PostgreSQL}
\label{\detokenize{rozdzial2/Konfiguracja_baz_danych/Konfiguracja_baz_danych:katalog-logow-i-struktura-katalogow-w-postgresql}}
\sphinxAtStartPar
\sphinxstylestrong{Katalog logów}
PostgreSQL zapisuje logi w różnych lokalizacjach, zależnie od systemu operacyjnego:
\begin{itemize}
\item {} 
\sphinxAtStartPar
Na Debianie/Ubuntu: \sphinxcode{\sphinxupquote{/var/log/postgresql}}

\item {} 
\sphinxAtStartPar
Na Red Hat/CentOS: \sphinxcode{\sphinxupquote{/var/lib/pgsql/\textless{}wersja\textgreater{}/data/pg\_log}}

\end{itemize}

\sphinxAtStartPar
\textgreater{} Uwaga: Aby zapisywać logi do pliku, należy upewnić się, że opcja \sphinxcode{\sphinxupquote{logging\_collector}} jest włączona w pliku \sphinxcode{\sphinxupquote{postgresql.conf}}.

\sphinxAtStartPar
\sphinxstylestrong{Struktura katalogów PostgreSQL}:

\begin{sphinxVerbatim}[commandchars=\\\{\}]
\PYG{n}{base}\PYG{o}{/}         \PYG{c+c1}{\PYGZsh{} dane użytkownika \textendash{} jedna podkatalog dla każdej bazy danych}
\PYG{k}{global}\PYG{o}{/}       \PYG{c+c1}{\PYGZsh{} dane wspólne dla wszystkich baz (np. użytkownicy)}
\PYG{n}{pg\PYGZus{}wal}\PYG{o}{/}       \PYG{c+c1}{\PYGZsh{} pliki WAL (Write\PYGZhy{}Ahead Logging)}
\PYG{n}{pg\PYGZus{}stat}\PYG{o}{/}      \PYG{c+c1}{\PYGZsh{} statystyki działania serwera}
\PYG{n}{pg\PYGZus{}log}\PYG{o}{/}       \PYG{c+c1}{\PYGZsh{} logi (jeśli skonfigurowane)}
\PYG{n}{pg\PYGZus{}tblspc}\PYG{o}{/}    \PYG{c+c1}{\PYGZsh{} dowiązania do tablespace’ów}
\PYG{n}{pg\PYGZus{}twophase}\PYG{o}{/}  \PYG{c+c1}{\PYGZsh{} dane dla transakcji dwufazowych}
\PYG{n}{postgresql}\PYG{o}{.}\PYG{n}{conf}  \PYG{c+c1}{\PYGZsh{} główny plik konfiguracyjny}
\PYG{n}{pg\PYGZus{}hba}\PYG{o}{.}\PYG{n}{conf}      \PYG{c+c1}{\PYGZsh{} kontrola dostępu}
\PYG{n}{pg\PYGZus{}ident}\PYG{o}{.}\PYG{n}{conf}    \PYG{c+c1}{\PYGZsh{} mapowanie użytkowników systemowych na bazodanowych}
\end{sphinxVerbatim}


\subsection{7. Przechowywanie i lokalizacja plików konfiguracyjnych}
\label{\detokenize{rozdzial2/Konfiguracja_baz_danych/Konfiguracja_baz_danych:przechowywanie-i-lokalizacja-plikow-konfiguracyjnych}}
\sphinxAtStartPar
Główne pliki konfiguracyjne:
\begin{itemize}
\item {} 
\sphinxAtStartPar
\sphinxcode{\sphinxupquote{postgresql.conf}} \textendash{} konfiguracja instancji PostgreSQL (parametry wydajności, logowania, lokalizacji itd.)

\item {} 
\sphinxAtStartPar
\sphinxcode{\sphinxupquote{pg\_hba.conf}} \textendash{} kontrola dostępu (adresy IP, użytkownicy, metody autoryzacji)

\item {} 
\sphinxAtStartPar
\sphinxcode{\sphinxupquote{pg\_ident.conf}} \textendash{} mapowanie użytkowników systemowych na użytkowników bazodanowych

\end{itemize}


\subsection{8. Podstawowe parametry konfiguracyjne}
\label{\detokenize{rozdzial2/Konfiguracja_baz_danych/Konfiguracja_baz_danych:podstawowe-parametry-konfiguracyjne}}
\sphinxAtStartPar
\sphinxstylestrong{Słuchanie połączeń:}

\begin{sphinxVerbatim}[commandchars=\\\{\}]
\PYG{n}{listen\PYGZus{}addresses} \PYG{o}{=} \PYG{l+s+s1}{\PYGZsq{}}\PYG{l+s+s1}{localhost}\PYG{l+s+s1}{\PYGZsq{}}
\PYG{n}{port} \PYG{o}{=} \PYG{l+m+mi}{5432}
\end{sphinxVerbatim}

\sphinxAtStartPar
\sphinxstylestrong{Pamięć i wydajność:}

\begin{sphinxVerbatim}[commandchars=\\\{\}]
\PYG{n}{shared\PYGZus{}buffers} \PYG{o}{=} \PYG{l+m+mi}{512}\PYG{n}{MB}         \PYG{c+c1}{\PYGZsh{} pamięć współdzielona}
\PYG{n}{work\PYGZus{}mem} \PYG{o}{=} \PYG{l+m+mi}{4}\PYG{n}{MB}                 \PYG{c+c1}{\PYGZsh{} pamięć na operacje sortowania/złączeń}
\PYG{n}{maintenance\PYGZus{}work\PYGZus{}mem} \PYG{o}{=} \PYG{l+m+mi}{64}\PYG{n}{MB}    \PYG{c+c1}{\PYGZsh{} dla operacji VACUUM, CREATE INDEX}
\end{sphinxVerbatim}

\sphinxAtStartPar
\sphinxstylestrong{Autovacuum:}

\begin{sphinxVerbatim}[commandchars=\\\{\}]
\PYG{n}{autovacuum} \PYG{o}{=} \PYG{n}{on}
\PYG{n}{autovacuum\PYGZus{}naptime} \PYG{o}{=} \PYG{l+m+mi}{1}\PYG{n+nb}{min}
\end{sphinxVerbatim}

\sphinxAtStartPar
\sphinxstylestrong{Konfiguracja pliku} \sphinxcode{\sphinxupquote{pg\_hba.conf}}:

\begin{sphinxVerbatim}[commandchars=\\\{\}]
\PYG{c+c1}{\PYGZsh{} TYPE  DATABASE  USER  ADDRESS         METHOD}
\PYG{n}{local}   \PYG{n+nb}{all}       \PYG{n+nb}{all}   \PYG{n}{md5}
\PYG{n}{host}    \PYG{n+nb}{all}       \PYG{n+nb}{all}   \PYG{l+m+mf}{192.168}\PYG{l+m+mf}{.0}\PYG{l+m+mf}{.0}\PYG{o}{/}\PYG{l+m+mi}{24}  \PYG{n}{md5}
\end{sphinxVerbatim}

\sphinxAtStartPar
\sphinxstylestrong{Konfiguracja pliku} \sphinxcode{\sphinxupquote{pg\_ident.conf}}:

\begin{sphinxVerbatim}[commandchars=\\\{\}]
\PYG{c+c1}{\PYGZsh{} MAPNAME      SYSTEM\PYGZhy{}USERNAME   PG\PYGZhy{}USERNAME}
\PYG{n}{local\PYGZus{}users}  \PYG{n}{ubuntu}            \PYG{n}{postgres}
\PYG{n}{local\PYGZus{}users}  \PYG{n}{jan\PYGZus{}kowalski}      \PYG{n}{janek\PYGZus{}db}
\end{sphinxVerbatim}

\sphinxAtStartPar
Można użyć tej mapy w pliku \sphinxcode{\sphinxupquote{pg\_hba.conf}}:

\begin{sphinxVerbatim}[commandchars=\\\{\}]
\PYG{n}{local}   \PYG{n+nb}{all}     \PYG{n+nb}{all}     \PYG{n}{peer} \PYG{n+nb}{map}\PYG{o}{=}\PYG{n}{local\PYGZus{}users}
\end{sphinxVerbatim}


\subsection{9. Wstęp teoretyczny}
\label{\detokenize{rozdzial2/Konfiguracja_baz_danych/Konfiguracja_baz_danych:wstep-teoretyczny}}
\sphinxAtStartPar
Systemy zarządzania bazą danych (DBMS \textendash{} \sphinxstyleemphasis{Database Management System}) umożliwiają tworzenie, modyfikowanie i zarządzanie danymi. Ułatwiają organizację danych, zapewniają integralność, bezpieczeństwo oraz możliwość jednoczesnego dostępu wielu użytkowników.


\subsubsection{9.1 Klasyfikacja systemów zarządzania bazą danych}
\label{\detokenize{rozdzial2/Konfiguracja_baz_danych/Konfiguracja_baz_danych:klasyfikacja-systemow-zarzadzania-baza-danych}}
\sphinxAtStartPar
Systemy DBMS można klasyfikować według:
\begin{itemize}
\item {} 
\sphinxAtStartPar
\sphinxstylestrong{Architektura działania:}
\sphinxhyphen{} \sphinxstyleemphasis{Klient\sphinxhyphen{}serwer} \textendash{} system działa jako niezależna usługa (np. PostgreSQL).
\sphinxhyphen{} \sphinxstyleemphasis{Osadzony (embedded)} \textendash{} baza danych jest integralną częścią aplikacji (np. SQLite).

\item {} 
\sphinxAtStartPar
\sphinxstylestrong{Rodzaj danych i funkcjonalność:}
\sphinxhyphen{} \sphinxstyleemphasis{Relacyjne (RDBMS)} \textendash{} oparte na tabelach, kluczach i SQL.
\sphinxhyphen{} \sphinxstyleemphasis{Nierelacyjne (NoSQL)} \textendash{} oparte na dokumentach, modelu klucz\sphinxhyphen{}wartość lub grafach.

\end{itemize}

\sphinxAtStartPar
Oba systemy \textendash{} \sphinxstylestrong{SQLite} oraz \sphinxstylestrong{PostgreSQL} \textendash{} należą do relacyjnych baz danych, lecz różnią się architekturą, wydajnością, konfiguracją i przeznaczeniem.


\subsubsection{9.2 SQLite}
\label{\detokenize{rozdzial2/Konfiguracja_baz_danych/Konfiguracja_baz_danych:sqlite}}
\sphinxAtStartPar
SQLite to lekka, bezserwerowa baza danych typu embedded, gdzie cała baza znajduje się w jednym pliku. Dzięki temu jest bardzo wygodna przy tworzeniu aplikacji lokalnych, mobilnych oraz projektów prototypowych.

\sphinxAtStartPar
\sphinxstylestrong{Cechy SQLite:}
\begin{itemize}
\item {} 
\sphinxAtStartPar
Brak osobnego procesu serwera \textendash{} baza działa w kontekście aplikacji.

\item {} 
\sphinxAtStartPar
Niskie wymagania systemowe \textendash{} brak potrzeby instalacji i konfiguracji.

\item {} 
\sphinxAtStartPar
Baza przechowywana jako pojedynczy plik (\sphinxstyleemphasis{.sqlite} lub \sphinxstyleemphasis{.db}).

\item {} 
\sphinxAtStartPar
Pełna obsługa SQL (z pewnymi ograniczeniami) \textendash{} wspiera standard SQL\sphinxhyphen{}92.

\item {} 
\sphinxAtStartPar
Ograniczona skalowalność przy wielu użytkownikach.

\end{itemize}

\sphinxAtStartPar
\sphinxstylestrong{Zastosowanie:}
\begin{itemize}
\item {} 
\sphinxAtStartPar
Aplikacje desktopowe (np. Firefox, VS Code).

\item {} 
\sphinxAtStartPar
Aplikacje mobilne (Android, iOS).

\item {} 
\sphinxAtStartPar
Małe i średnie systemy bazodanowe.

\end{itemize}


\subsubsection{9.3 PostgreSQL}
\label{\detokenize{rozdzial2/Konfiguracja_baz_danych/Konfiguracja_baz_danych:postgresql}}
\sphinxAtStartPar
PostgreSQL to zaawansowany system relacyjnej bazy danych typu klient\sphinxhyphen{}serwer, rozwijany jako projekt open\sphinxhyphen{}source. Zapewnia pełne wsparcie dla SQL oraz liczne rozszerzenia (np. typy przestrzenne, JSON).

\sphinxAtStartPar
\sphinxstylestrong{Cechy PostgreSQL:}
\begin{itemize}
\item {} 
\sphinxAtStartPar
Architektura klient\sphinxhyphen{}serwer \textendash{} działa jako oddzielny proces.

\item {} 
\sphinxAtStartPar
Wysoka skalowalność i niezawodność \textendash{} obsługuje wielu użytkowników, złożone zapytania, replikację.

\item {} 
\sphinxAtStartPar
Obsługa transakcji, MVCC, indeksowania oraz zarządzania uprawnieniami.

\item {} 
\sphinxAtStartPar
Rozszerzalność \textendash{} możliwość definiowania własnych typów danych, funkcji i procedur.

\end{itemize}

\sphinxAtStartPar
\sphinxstylestrong{Konfiguracja:}
Plikami konfiguracyjnymi są:
\begin{itemize}
\item {} 
\sphinxAtStartPar
\sphinxcode{\sphinxupquote{postgresql.conf}} \textendash{} ustawienia ogólne (port, ścieżki, pamięć, logi).

\item {} 
\sphinxAtStartPar
\sphinxcode{\sphinxupquote{pg\_hba.conf}} \textendash{} reguły autoryzacji.

\item {} 
\sphinxAtStartPar
\sphinxcode{\sphinxupquote{pg\_ident.conf}} \textendash{} mapowanie użytkowników systemowych na bazodanowych.

\end{itemize}

\sphinxAtStartPar
\sphinxstylestrong{Zastosowanie:}
\begin{itemize}
\item {} 
\sphinxAtStartPar
Systemy biznesowe, bankowe, analityczne.

\item {} 
\sphinxAtStartPar
Aplikacje webowe i serwery aplikacyjne.

\item {} 
\sphinxAtStartPar
Środowiska o wysokich wymaganiach bezpieczeństwa i kontroli dostępu.

\end{itemize}


\subsubsection{9.4 Cel użycia obu systemów}
\label{\detokenize{rozdzial2/Konfiguracja_baz_danych/Konfiguracja_baz_danych:cel-uzycia-obu-systemow}}
\sphinxAtStartPar
W ramach zajęć wykorzystano zarówno \sphinxstylestrong{SQLite} (dla szybkiego startu i analizy zapytań bez instalacji serwera), jak i \sphinxstylestrong{PostgreSQL} (dla nauki konfiguracji, zarządzania użytkownikami, uprawnieniami oraz obsługi złożonych operacji).


\subsection{10. Zarządzanie konfiguracją w PostgreSQL}
\label{\detokenize{rozdzial2/Konfiguracja_baz_danych/Konfiguracja_baz_danych:zarzadzanie-konfiguracja-w-postgresql}}
\sphinxAtStartPar
PostgreSQL oferuje rozbudowany i elastyczny mechanizm konfiguracji, umożliwiający precyzyjne dostosowanie działania bazy danych do potrzeb użytkownika oraz środowiska (lokalnego, deweloperskiego, testowego czy produkcyjnego).


\subsubsection{10.1 Pliki konfiguracyjne}
\label{\detokenize{rozdzial2/Konfiguracja_baz_danych/Konfiguracja_baz_danych:pliki-konfiguracyjne}}
\sphinxAtStartPar
Główne pliki konfiguracyjne PostgreSQL:
\begin{itemize}
\item {} 
\sphinxAtStartPar
\sphinxstylestrong{postgresql.conf} \textendash{} ustawienia dotyczące pamięci, sieci, logowania, autovacuum, planowania zapytań.

\item {} 
\sphinxAtStartPar
\sphinxstylestrong{pg\_hba.conf} \textendash{} definiuje metody uwierzytelniania i dostęp z określonych adresów.

\item {} 
\sphinxAtStartPar
\sphinxstylestrong{pg\_ident.conf} \textendash{} mapowanie nazw użytkowników systemowych na użytkowników PostgreSQL.

\end{itemize}

\sphinxAtStartPar
Pliki te zazwyczaj znajdują się w katalogu danych (np. \sphinxcode{\sphinxupquote{/var/lib/postgresql/15/main/}} lub \sphinxcode{\sphinxupquote{/etc/postgresql/15/main/}}).


\subsubsection{10.2 Przykładowe kluczowe parametry \sphinxstyleliteralintitle{\sphinxupquote{postgresql.conf}}}
\label{\detokenize{rozdzial2/Konfiguracja_baz_danych/Konfiguracja_baz_danych:przykladowe-kluczowe-parametry-postgresql-conf}}

\begin{savenotes}\sphinxattablestart
\sphinxthistablewithglobalstyle
\centering
\begin{tabulary}{\linewidth}[t]{TT}
\sphinxtoprule
\sphinxstyletheadfamily 
\sphinxAtStartPar
\sphinxstylestrong{Parametr}
&\sphinxstyletheadfamily 
\sphinxAtStartPar
\sphinxstylestrong{Opis}
\\
\sphinxmidrule
\sphinxtableatstartofbodyhook
\sphinxAtStartPar
shared\_buffers
&
\sphinxAtStartPar
Ilość pamięci RAM przeznaczona na bufor danych (rekomendacja: 25\textendash{}40\% RAM).
\\
\sphinxhline
\sphinxAtStartPar
work\_mem
&
\sphinxAtStartPar
Pamięć dla pojedynczej operacji zapytania (np. sortowania).
\\
\sphinxhline
\sphinxAtStartPar
maintenance\_work\_mem
&
\sphinxAtStartPar
Pamięć dla operacji administracyjnych (np. VACUUM, CREATE INDEX).
\\
\sphinxhline
\sphinxAtStartPar
effective\_cache\_size
&
\sphinxAtStartPar
Szacunkowa ilość pamięci dostępnej na cache systemu operacyjnego.
\\
\sphinxhline
\sphinxAtStartPar
max\_connections
&
\sphinxAtStartPar
Maksymalna liczba jednoczesnych połączeń z bazą danych.
\\
\sphinxhline
\sphinxAtStartPar
log\_directory
&
\sphinxAtStartPar
Katalog, w którym zapisywane są logi PostgreSQL.
\\
\sphinxhline
\sphinxAtStartPar
autovacuum
&
\sphinxAtStartPar
Włącza lub wyłącza automatyczne odświeżanie nieużywanych wierszy.
\\
\sphinxbottomrule
\end{tabulary}
\sphinxtableafterendhook\par
\sphinxattableend\end{savenotes}


\subsubsection{10.3 Sposoby zmiany konfiguracji}
\label{\detokenize{rozdzial2/Konfiguracja_baz_danych/Konfiguracja_baz_danych:sposoby-zmiany-konfiguracji}}\begin{enumerate}
\sphinxsetlistlabels{\arabic}{enumi}{enumii}{}{.}%
\item {} 
\sphinxAtStartPar
\sphinxstylestrong{Edycja pliku} \sphinxcode{\sphinxupquote{postgresql.conf}}

\sphinxAtStartPar
Zmiany są trwałe, ale wymagają restartu serwera (w niektórych przypadkach wystarczy reload).

\sphinxAtStartPar
\sphinxstylestrong{Przykład:}

\begin{sphinxVerbatim}[commandchars=\\\{\}]
\PYG{n}{shared\PYGZus{}buffers} \PYG{o}{=} \PYG{l+m+mi}{512}\PYG{n}{MB}
\PYG{n}{work\PYGZus{}mem} \PYG{o}{=} \PYG{l+m+mi}{64}\PYG{n}{MB}
\end{sphinxVerbatim}

\item {} 
\sphinxAtStartPar
\sphinxstylestrong{Dynamiczna zmiana poprzez SQL}

\sphinxAtStartPar
\sphinxstylestrong{Przykład:}

\begin{sphinxVerbatim}[commandchars=\\\{\}]
\PYG{n}{ALTER} \PYG{n}{SYSTEM} \PYG{n}{SET} \PYG{n}{work\PYGZus{}mem} \PYG{o}{=} \PYG{l+s+s1}{\PYGZsq{}}\PYG{l+s+s1}{64MB}\PYG{l+s+s1}{\PYGZsq{}}\PYG{p}{;}
\PYG{n}{SELECT} \PYG{n}{pg\PYGZus{}reload\PYGZus{}conf}\PYG{p}{(}\PYG{p}{)}\PYG{p}{;}  \PYG{c+c1}{\PYGZsh{} ładowanie zmian bez restartu}
\end{sphinxVerbatim}

\item {} 
\sphinxAtStartPar
\sphinxstylestrong{Tymczasowa zmiana dla jednej sesji}

\sphinxAtStartPar
\sphinxstylestrong{Przykład:}

\begin{sphinxVerbatim}[commandchars=\\\{\}]
\PYG{n}{SET} \PYG{n}{work\PYGZus{}mem} \PYG{o}{=} \PYG{l+s+s1}{\PYGZsq{}}\PYG{l+s+s1}{128MB}\PYG{l+s+s1}{\PYGZsq{}}\PYG{p}{;}
\end{sphinxVerbatim}

\end{enumerate}


\subsubsection{10.4 Sprawdzanie konfiguracji}
\label{\detokenize{rozdzial2/Konfiguracja_baz_danych/Konfiguracja_baz_danych:sprawdzanie-konfiguracji}}\begin{itemize}
\item {} 
\sphinxAtStartPar
Aby sprawdzić aktualną wartość parametru:

\begin{sphinxVerbatim}[commandchars=\\\{\}]
\PYG{n}{SHOW} \PYG{n}{work\PYGZus{}mem}\PYG{p}{;}
\end{sphinxVerbatim}

\item {} 
\sphinxAtStartPar
Pobranie szczegółowych informacji:

\begin{sphinxVerbatim}[commandchars=\\\{\}]
\PYG{n}{SELECT} \PYG{n}{name}\PYG{p}{,} \PYG{n}{setting}\PYG{p}{,} \PYG{n}{unit}\PYG{p}{,} \PYG{n}{context}\PYG{p}{,} \PYG{n}{source}
\PYG{n}{FROM} \PYG{n}{pg\PYGZus{}settings}
\PYG{n}{WHERE} \PYG{n}{name} \PYG{o}{=} \PYG{l+s+s1}{\PYGZsq{}}\PYG{l+s+s1}{work\PYGZus{}mem}\PYG{l+s+s1}{\PYGZsq{}}\PYG{p}{;}
\end{sphinxVerbatim}

\item {} 
\sphinxAtStartPar
Wylistowanie parametrów wymagających restartu serwera:

\begin{sphinxVerbatim}[commandchars=\\\{\}]
\PYG{n}{SELECT} \PYG{n}{name} \PYG{n}{FROM} \PYG{n}{pg\PYGZus{}settings} \PYG{n}{WHERE} \PYG{n}{context} \PYG{o}{=} \PYG{l+s+s1}{\PYGZsq{}}\PYG{l+s+s1}{postmaster}\PYG{l+s+s1}{\PYGZsq{}}\PYG{p}{;}
\end{sphinxVerbatim}

\end{itemize}


\subsubsection{10.5 Narzędzia pomocnicze}
\label{\detokenize{rozdzial2/Konfiguracja_baz_danych/Konfiguracja_baz_danych:narzedzia-pomocnicze}}\begin{itemize}
\item {} 
\sphinxAtStartPar
\sphinxstylestrong{pg\_ctl} \textendash{} narzędzie do zarządzania serwerem (start/stop/reload).

\item {} 
\sphinxAtStartPar
\sphinxstylestrong{psql} \textendash{} klient terminalowy PostgreSQL do wykonywania zapytań oraz operacji administracyjnych.

\item {} 
\sphinxAtStartPar
\sphinxstylestrong{pgAdmin} \textendash{} graficzne narzędzie do zarządzania bazą PostgreSQL (umożliwia edycję konfiguracji przez GUI).

\end{itemize}


\subsubsection{10.6 Kontrola dostępu i mechanizmy uwierzytelniania}
\label{\detokenize{rozdzial2/Konfiguracja_baz_danych/Konfiguracja_baz_danych:kontrola-dostepu-i-mechanizmy-uwierzytelniania}}
\sphinxAtStartPar
Konfiguracja umożliwia określenie, z jakich adresów i w jaki sposób można łączyć się z bazą:
\begin{itemize}
\item {} 
\sphinxAtStartPar
\sphinxstylestrong{Dostęp lokalny (localhost)} \textendash{} połączenia z tej samej maszyny.

\item {} 
\sphinxAtStartPar
\sphinxstylestrong{Dostęp z podsieci} \textendash{} administrator może wskazać konkretne podsieci IP (np. \sphinxcode{\sphinxupquote{192.168.0.0/24}}).

\item {} 
\sphinxAtStartPar
\sphinxstylestrong{Mechanizmy uwierzytelniania} \textendash{} np. \sphinxcode{\sphinxupquote{md5}}, \sphinxcode{\sphinxupquote{scram\sphinxhyphen{}sha\sphinxhyphen{}256}}, \sphinxcode{\sphinxupquote{peer}} (weryfikacja użytkownika systemowego) czy \sphinxcode{\sphinxupquote{trust}}.

\end{itemize}

\sphinxAtStartPar
Ważne, aby mechanizm \sphinxcode{\sphinxupquote{peer}} był odpowiednio skonfigurowany, gdyż umożliwia automatyczną autoryzację, jeśli nazwa użytkownika systemowego i bazy zgadza się.


\subsection{11. Planowanie}
\label{\detokenize{rozdzial2/Konfiguracja_baz_danych/Konfiguracja_baz_danych:planowanie}}
\sphinxAtStartPar
Planowanie w kontekście PostgreSQL oznacza optymalizację wykonania zapytań oraz efektywne zarządzanie zasobami.


\subsubsection{11.1 Co to jest planowanie zapytań?}
\label{\detokenize{rozdzial2/Konfiguracja_baz_danych/Konfiguracja_baz_danych:co-to-jest-planowanie-zapytan}}
\sphinxAtStartPar
Proces planowania zapytań obejmuje:
\begin{itemize}
\item {} 
\sphinxAtStartPar
Analizę składni i struktury zapytania SQL.

\item {} 
\sphinxAtStartPar
Przegląd dostępnych statystyk dotyczących tabel, indeksów i danych.

\item {} 
\sphinxAtStartPar
Dobór sposobu dostępu do danych (pełny skan, indeks, join, sortowanie).

\item {} 
\sphinxAtStartPar
Tworzenie planu wykonania, czyli sekwencji operacji potrzebnych do uzyskania wyniku.

\end{itemize}

\sphinxAtStartPar
Administrator może również kontrolować częstotliwość aktualizacji statystyk (np. \sphinxcode{\sphinxupquote{default\_statistics\_target}}, \sphinxcode{\sphinxupquote{autovacuum}}).


\subsubsection{11.2 Mechanizm planowania w PostgreSQL}
\label{\detokenize{rozdzial2/Konfiguracja_baz_danych/Konfiguracja_baz_danych:mechanizm-planowania-w-postgresql}}
\sphinxAtStartPar
PostgreSQL wykorzystuje kosztowy optymalizator; przy użyciu statystyk (liczby wierszy, rozkładu danych) szacuje „koszt” różnych metod wykonania zapytania, wybierając tę, która jest najtańsza pod względem czasu i zasobów.


\subsubsection{11.3 Statystyki i ich aktualizacja}
\label{\detokenize{rozdzial2/Konfiguracja_baz_danych/Konfiguracja_baz_danych:statystyki-i-ich-aktualizacja}}\begin{itemize}
\item {} 
\sphinxAtStartPar
Statystyki są tworzone przy pomocy polecenia \sphinxcode{\sphinxupquote{ANALYZE}} \textendash{} zbiera dane o rozkładzie wartości kolumn.

\item {} 
\sphinxAtStartPar
Mechanizm autovacuum odświeża statystyki automatycznie.

\end{itemize}

\sphinxAtStartPar
\sphinxstylestrong{Przykład:}

\begin{sphinxVerbatim}[commandchars=\\\{\}]
\PYG{n}{ANALYZE} \PYG{p}{[}\PYG{n}{nazwa\PYGZus{}tabeli}\PYG{p}{]}\PYG{p}{;}
\end{sphinxVerbatim}

\sphinxAtStartPar
W systemach o dużym obciążeniu planowanie uwzględnia również równoległość (parallel query).


\subsubsection{11.4 Typy planów wykonania}
\label{\detokenize{rozdzial2/Konfiguracja_baz_danych/Konfiguracja_baz_danych:typy-planow-wykonania}}
\sphinxAtStartPar
Przykładowe typy planów wykonania:
\begin{itemize}
\item {} 
\sphinxAtStartPar
\sphinxstylestrong{Seq Scan} \textendash{} pełny skan tabeli (gdy indeksy są niedostępne lub nieefektywne).

\item {} 
\sphinxAtStartPar
\sphinxstylestrong{Index Scan} \textendash{} wykorzystanie indeksu.

\item {} 
\sphinxAtStartPar
\sphinxstylestrong{Bitmap Index Scan} \textendash{} łączenie efektywności indeksów ze skanem sekwencyjnym.

\item {} 
\sphinxAtStartPar
\sphinxstylestrong{Nested Loop Join} \textendash{} efektywny join dla małych zbiorów.

\item {} 
\sphinxAtStartPar
\sphinxstylestrong{Hash Join} \textendash{} buduje tablicę hash dla dużych zbiorów.

\item {} 
\sphinxAtStartPar
\sphinxstylestrong{Merge Join} \textendash{} stosowany, gdy dane są posortowane.

\end{itemize}


\subsubsection{11.5 Jak sprawdzić plan zapytania?}
\label{\detokenize{rozdzial2/Konfiguracja_baz_danych/Konfiguracja_baz_danych:jak-sprawdzic-plan-zapytania}}
\sphinxAtStartPar
Aby zobaczyć plan wybrany przez PostgreSQL, można użyć:

\begin{sphinxVerbatim}[commandchars=\\\{\}]
\PYG{n}{EXPLAIN} \PYG{n}{ANALYZE} \PYG{n}{SELECT} \PYG{o}{*} \PYG{n}{FROM} \PYG{n}{tabela} \PYG{n}{WHERE} \PYG{n}{kolumna} \PYG{o}{=} \PYG{l+s+s1}{\PYGZsq{}}\PYG{l+s+s1}{wartość}\PYG{l+s+s1}{\PYGZsq{}}\PYG{p}{;}
\end{sphinxVerbatim}
\begin{itemize}
\item {} 
\sphinxAtStartPar
\sphinxcode{\sphinxupquote{EXPLAIN}} \textendash{} wyświetla plan bez wykonania zapytania.

\item {} 
\sphinxAtStartPar
\sphinxcode{\sphinxupquote{ANALYZE}} \textendash{} wykonuje zapytanie i podaje rzeczywiste czasy wykonania.

\end{itemize}

\sphinxAtStartPar
\sphinxstylestrong{Przykładowy wynik:}

\begin{sphinxVerbatim}[commandchars=\\\{\}]
\PYG{n}{Index} \PYG{n}{Scan} \PYG{n}{using} \PYG{n}{idx\PYGZus{}kolumna} \PYG{n}{on} \PYG{n}{tabela} \PYG{p}{(}\PYG{n}{cost}\PYG{o}{=}\PYG{l+m+mf}{0.29}\PYG{o}{.}\PYG{l+m+mf}{.8}\PYG{l+m+mf}{.56} \PYG{n}{rows}\PYG{o}{=}\PYG{l+m+mi}{3} \PYG{n}{width}\PYG{o}{=}\PYG{l+m+mi}{244}\PYG{p}{)}
\PYG{n}{Index} \PYG{n}{Cond}\PYG{p}{:} \PYG{p}{(}\PYG{n}{kolumna} \PYG{o}{=} \PYG{l+s+s1}{\PYGZsq{}}\PYG{l+s+s1}{wartość}\PYG{l+s+s1}{\PYGZsq{}}\PYG{p}{:}\PYG{p}{:}\PYG{n}{text}\PYG{p}{)}
\end{sphinxVerbatim}


\subsubsection{11.6 Parametry planowania i optymalizacji}
\label{\detokenize{rozdzial2/Konfiguracja_baz_danych/Konfiguracja_baz_danych:parametry-planowania-i-optymalizacji}}
\sphinxAtStartPar
W pliku \sphinxcode{\sphinxupquote{postgresql.conf}} można konfigurować m.in.:
\begin{itemize}
\item {} 
\sphinxAtStartPar
\sphinxcode{\sphinxupquote{random\_page\_cost}} \textendash{} koszt odczytu strony z dysku SSD/HDD.

\item {} 
\sphinxAtStartPar
\sphinxcode{\sphinxupquote{cpu\_tuple\_cost}} \textendash{} koszt przetwarzania pojedynczego wiersza.

\item {} 
\sphinxAtStartPar
\sphinxcode{\sphinxupquote{enable\_seqscan}}, \sphinxcode{\sphinxupquote{enable\_indexscan}}, \sphinxcode{\sphinxupquote{enable\_bitmapscan}} \textendash{} włączanie/wyłączanie konkretnych typów skanów.

\end{itemize}

\sphinxAtStartPar
Dostosowanie tych parametrów pozwala zoptymalizować planowanie zgodnie ze specyfiką sprzętu i obciążenia.


\subsection{12. Tabele \textendash{} rozmiar, planowanie i monitorowanie}
\label{\detokenize{rozdzial2/Konfiguracja_baz_danych/Konfiguracja_baz_danych:tabele-rozmiar-planowanie-i-monitorowanie}}

\subsubsection{12.1 Rozmiar tabeli}
\label{\detokenize{rozdzial2/Konfiguracja_baz_danych/Konfiguracja_baz_danych:rozmiar-tabeli}}
\sphinxAtStartPar
Rozmiar tabeli w PostgreSQL obejmuje dane (wiersze), strukturę, indeksy, dane TOAST oraz pliki statystyk. Do monitorowania rozmiaru stosuje się funkcje:
\begin{itemize}
\item {} 
\sphinxAtStartPar
\sphinxcode{\sphinxupquote{pg\_relation\_size()}} \textendash{} rozmiar tabeli lub pojedynczego indeksu.

\item {} 
\sphinxAtStartPar
\sphinxcode{\sphinxupquote{pg\_total\_relation\_size()}} \textendash{} całkowity rozmiar tabeli wraz z indeksami i TOAST.

\end{itemize}


\subsubsection{12.2 Planowanie rozmiaru i jego kontrola}
\label{\detokenize{rozdzial2/Konfiguracja_baz_danych/Konfiguracja_baz_danych:planowanie-rozmiaru-i-jego-kontrola}}
\sphinxAtStartPar
Podczas projektowania bazy danych należy oszacować potencjalny rozmiar tabel, biorąc pod uwagę liczbę wierszy i rozmiar pojedynczego rekordu. PostgreSQL nie posiada sztywnego limitu (poza ograniczeniami systemu plików i 32\sphinxhyphen{}bitowym limitem liczby stron). Parametr \sphinxcode{\sphinxupquote{fillfactor}} może być stosowany do optymalizacji częstotliwości operacji UPDATE i VACUUM.


\subsubsection{12.3 Monitorowanie rozmiaru tabel}
\label{\detokenize{rozdzial2/Konfiguracja_baz_danych/Konfiguracja_baz_danych:monitorowanie-rozmiaru-tabel}}
\sphinxAtStartPar
\sphinxstylestrong{Przykład zapytania:}

\begin{sphinxVerbatim}[commandchars=\\\{\}]
\PYG{n}{SELECT} \PYG{n}{pg\PYGZus{}size\PYGZus{}pretty}\PYG{p}{(}\PYG{n}{pg\PYGZus{}total\PYGZus{}relation\PYGZus{}size}\PYG{p}{(}\PYG{l+s+s1}{\PYGZsq{}}\PYG{l+s+s1}{nazwa\PYGZus{}tabeli}\PYG{l+s+s1}{\PYGZsq{}}\PYG{p}{)}\PYG{p}{)}\PYG{p}{;}
\end{sphinxVerbatim}

\sphinxAtStartPar
Inne funkcje:
\begin{itemize}
\item {} 
\sphinxAtStartPar
\sphinxcode{\sphinxupquote{pg\_relation\_size}} \textendash{} rozmiar samej tabeli.

\item {} 
\sphinxAtStartPar
\sphinxcode{\sphinxupquote{pg\_indexes\_size}} \textendash{} rozmiar indeksów.

\item {} 
\sphinxAtStartPar
\sphinxcode{\sphinxupquote{pg\_table\_size}} \textendash{} zwraca łączny rozmiar tabeli wraz z TOAST.

\end{itemize}


\subsubsection{12.4 Planowanie na poziomie tabel}
\label{\detokenize{rozdzial2/Konfiguracja_baz_danych/Konfiguracja_baz_danych:planowanie-na-poziomie-tabel}}
\sphinxAtStartPar
Administrator może wpływać na fizyczne rozmieszczenie danych poprzez:
\begin{itemize}
\item {} 
\sphinxAtStartPar
\sphinxstylestrong{Tablespaces} \textendash{} przenoszenie tabel lub indeksów na inne dyski/partycje.

\item {} 
\sphinxAtStartPar
\sphinxstylestrong{Podział tabel (partitioning)} \textendash{} rozbijanie dużych tabel na mniejsze części.

\end{itemize}


\subsubsection{12.5 Monitorowanie stanu tabel}
\label{\detokenize{rozdzial2/Konfiguracja_baz_danych/Konfiguracja_baz_danych:monitorowanie-stanu-tabel}}
\sphinxAtStartPar
Monitorowanie obejmuje:
\begin{itemize}
\item {} 
\sphinxAtStartPar
Śledzenie fragmentacji danych.

\item {} 
\sphinxAtStartPar
Kontrolę wzrostu tabel i indeksów.

\item {} 
\sphinxAtStartPar
Statystyki dotyczące operacji odczytów i zapisów.

\end{itemize}

\sphinxAtStartPar
Narzędzia i widoki systemowe:
\begin{itemize}
\item {} 
\sphinxAtStartPar
\sphinxcode{\sphinxupquote{pg\_stat\_all\_tables}}

\item {} 
\sphinxAtStartPar
\sphinxcode{\sphinxupquote{pg\_stat\_user\_tables}}

\item {} 
\sphinxAtStartPar
\sphinxcode{\sphinxupquote{pg\_stat\_activity}}

\end{itemize}


\subsubsection{12.6 Konserwacja i optymalizacja tabel}
\label{\detokenize{rozdzial2/Konfiguracja_baz_danych/Konfiguracja_baz_danych:konserwacja-i-optymalizacja-tabel}}
\sphinxAtStartPar
Regularne uruchamianie poleceń:
\begin{itemize}
\item {} 
\sphinxAtStartPar
\sphinxstylestrong{VACUUM} \textendash{} usuwa martwe wiersze, zapobiegając nadmiernej fragmentacji.

\item {} 
\sphinxAtStartPar
\sphinxstylestrong{ANALYZE} \textendash{} aktualizuje statystyki, ułatwiając optymalizację zapytań.

\end{itemize}

\sphinxAtStartPar
Dla bardzo dużych tabel można stosować \sphinxcode{\sphinxupquote{VACUUM FULL}} lub reorganizację danych, aby odzyskać przestrzeń.


\subsection{13. Rozmiar pojedynczych tabel, rozmiar wszystkich tabel, indeksów tabeli}
\label{\detokenize{rozdzial2/Konfiguracja_baz_danych/Konfiguracja_baz_danych:rozmiar-pojedynczych-tabel-rozmiar-wszystkich-tabel-indeksow-tabeli}}
\sphinxAtStartPar
Efektywne zarządzanie rozmiarem tabel oraz ich indeksów ma kluczowe znaczenie dla wydajności systemu.


\subsubsection{13.1 Rozmiar pojedynczej tabeli}
\label{\detokenize{rozdzial2/Konfiguracja_baz_danych/Konfiguracja_baz_danych:rozmiar-pojedynczej-tabeli}}
\sphinxAtStartPar
Do pozyskania informacji o rozmiarze konkretnej tabeli służą funkcje:
\begin{itemize}
\item {} 
\sphinxAtStartPar
\sphinxcode{\sphinxupquote{pg\_relation\_size(\textquotesingle{}nazwa\_tabeli\textquotesingle{})}} \textendash{} rozmiar danych tabeli (w bajtach).

\item {} 
\sphinxAtStartPar
\sphinxcode{\sphinxupquote{pg\_table\_size(\textquotesingle{}nazwa\_tabeli\textquotesingle{})}} \textendash{} rozmiar danych tabeli wraz z danymi TOAST.

\item {} 
\sphinxAtStartPar
\sphinxcode{\sphinxupquote{pg\_total\_relation\_size(\textquotesingle{}nazwa\_tabeli\textquotesingle{})}} \textendash{} całkowity rozmiar tabeli wraz z indeksami i TOAST.

\end{itemize}

\sphinxAtStartPar
\sphinxstylestrong{Przykład zapytania:}

\begin{sphinxVerbatim}[commandchars=\\\{\}]
\PYG{n}{SELECT}
  \PYG{n}{pg\PYGZus{}size\PYGZus{}pretty}\PYG{p}{(}\PYG{n}{pg\PYGZus{}relation\PYGZus{}size}\PYG{p}{(}\PYG{l+s+s1}{\PYGZsq{}}\PYG{l+s+s1}{nazwa\PYGZus{}tabeli}\PYG{l+s+s1}{\PYGZsq{}}\PYG{p}{)}\PYG{p}{)} \PYG{n}{AS} \PYG{n}{data\PYGZus{}size}\PYG{p}{,}
  \PYG{n}{pg\PYGZus{}size\PYGZus{}pretty}\PYG{p}{(}\PYG{n}{pg\PYGZus{}indexes\PYGZus{}size}\PYG{p}{(}\PYG{l+s+s1}{\PYGZsq{}}\PYG{l+s+s1}{nazwa\PYGZus{}tabeli}\PYG{l+s+s1}{\PYGZsq{}}\PYG{p}{)}\PYG{p}{)} \PYG{n}{AS} \PYG{n}{indexes\PYGZus{}size}\PYG{p}{,}
  \PYG{n}{pg\PYGZus{}size\PYGZus{}pretty}\PYG{p}{(}\PYG{n}{pg\PYGZus{}total\PYGZus{}relation\PYGZus{}size}\PYG{p}{(}\PYG{l+s+s1}{\PYGZsq{}}\PYG{l+s+s1}{nazwa\PYGZus{}tabeli}\PYG{l+s+s1}{\PYGZsq{}}\PYG{p}{)}\PYG{p}{)} \PYG{n}{AS} \PYG{n}{total\PYGZus{}size}\PYG{p}{;}
\end{sphinxVerbatim}


\subsubsection{13.2 Rozmiar wszystkich tabel w bazie}
\label{\detokenize{rozdzial2/Konfiguracja_baz_danych/Konfiguracja_baz_danych:rozmiar-wszystkich-tabel-w-bazie}}
\sphinxAtStartPar
Zapytanie pozwalające wylistować wszystkie tabele i ich rozmiary:

\begin{sphinxVerbatim}[commandchars=\\\{\}]
\PYG{n}{SELECT}
  \PYG{n}{schemaname}\PYG{p}{,}
  \PYG{n}{relname} \PYG{n}{AS} \PYG{n}{table\PYGZus{}name}\PYG{p}{,}
  \PYG{n}{pg\PYGZus{}size\PYGZus{}pretty}\PYG{p}{(}\PYG{n}{pg\PYGZus{}total\PYGZus{}relation\PYGZus{}size}\PYG{p}{(}\PYG{n}{relid}\PYG{p}{)}\PYG{p}{)} \PYG{n}{AS} \PYG{n}{total\PYGZus{}size}
\PYG{n}{FROM}
  \PYG{n}{pg\PYGZus{}catalog}\PYG{o}{.}\PYG{n}{pg\PYGZus{}statio\PYGZus{}user\PYGZus{}tables}
\PYG{n}{ORDER} \PYG{n}{BY}
  \PYG{n}{pg\PYGZus{}total\PYGZus{}relation\PYGZus{}size}\PYG{p}{(}\PYG{n}{relid}\PYG{p}{)} \PYG{n}{DESC}\PYG{p}{;}
\end{sphinxVerbatim}


\subsubsection{13.3 Rozmiar indeksów tabeli}
\label{\detokenize{rozdzial2/Konfiguracja_baz_danych/Konfiguracja_baz_danych:rozmiar-indeksow-tabeli}}
\sphinxAtStartPar
Funkcja:

\begin{sphinxVerbatim}[commandchars=\\\{\}]
\PYG{n}{pg\PYGZus{}indexes\PYGZus{}size}\PYG{p}{(}\PYG{l+s+s1}{\PYGZsq{}}\PYG{l+s+s1}{nazwa\PYGZus{}tabeli}\PYG{l+s+s1}{\PYGZsq{}}\PYG{p}{)}
\end{sphinxVerbatim}

\sphinxAtStartPar
Pozwala sprawdzić rozmiar wszystkich indeksów przypisanych do danej tabeli. Monitorowanie indeksów pomaga w podejmowaniu decyzji o ich przebudowie lub usunięciu.


\subsubsection{13.4 Znaczenie rozmiarów}
\label{\detokenize{rozdzial2/Konfiguracja_baz_danych/Konfiguracja_baz_danych:znaczenie-rozmiarow}}
\sphinxAtStartPar
Duże tabele i indeksy mogą powodować:
\begin{itemize}
\item {} 
\sphinxAtStartPar
Wolniejsze operacje zapisu i odczytu.

\item {} 
\sphinxAtStartPar
Wydłużony czas tworzenia kopii zapasowych.

\item {} 
\sphinxAtStartPar
Większe wymagania przestrzeni dyskowej.

\end{itemize}

\sphinxAtStartPar
Regularne monitorowanie rozmiaru umożliwia planowanie działań optymalizacyjnych i konserwacyjnych.


\subsection{14. Rozmiar}
\label{\detokenize{rozdzial2/Konfiguracja_baz_danych/Konfiguracja_baz_danych:rozmiar}}
\sphinxAtStartPar
Pojęcie „rozmiar” odnosi się do przestrzeni dyskowej zajmowanej przez elementy bazy danych \textendash{} tabele, indeksy, pliki TOAST, a także całe bazy danych lub schematy.


\subsubsection{14.1 Rodzaje rozmiarów w PostgreSQL}
\label{\detokenize{rozdzial2/Konfiguracja_baz_danych/Konfiguracja_baz_danych:rodzaje-rozmiarow-w-postgresql}}\begin{itemize}
\item {} 
\sphinxAtStartPar
\sphinxstylestrong{Rozmiar pojedynczego obiektu} (tabeli, indeksu):
Funkcje takie jak \sphinxcode{\sphinxupquote{pg\_relation\_size()}}, \sphinxcode{\sphinxupquote{pg\_table\_size()}}, \sphinxcode{\sphinxupquote{pg\_indexes\_size()}} oraz \sphinxcode{\sphinxupquote{pg\_total\_relation\_size()}}.

\item {} 
\sphinxAtStartPar
\sphinxstylestrong{Rozmiar schematu lub bazy danych}:
Funkcje \sphinxcode{\sphinxupquote{pg\_namespace\_size(\textquotesingle{}nazwa\_schematu\textquotesingle{})}} oraz \sphinxcode{\sphinxupquote{pg\_database\_size(\textquotesingle{}nazwa\_bazy\textquotesingle{})}}.

\item {} 
\sphinxAtStartPar
\sphinxstylestrong{Rozmiar plików TOAST}:
Duże wartości (np. teksty, obrazy) są przenoszone do struktur TOAST, których rozmiar wliczany jest do rozmiaru tabeli, choć można go analizować osobno.

\end{itemize}


\subsubsection{14.2 Monitorowanie i kontrola rozmiaru}
\label{\detokenize{rozdzial2/Konfiguracja_baz_danych/Konfiguracja_baz_danych:monitorowanie-i-kontrola-rozmiaru}}
\sphinxAtStartPar
Administratorzy baz danych powinni regularnie monitorować rozmiar baz danych i jej obiektów, aby:
\begin{itemize}
\item {} 
\sphinxAtStartPar
Zapobiegać przekroczeniu limitów przestrzeni dyskowej.

\item {} 
\sphinxAtStartPar
Wcześniej wykrywać problemy z fragmentacją.

\item {} 
\sphinxAtStartPar
Planować archiwizację lub czyszczenie danych.

\end{itemize}

\sphinxAtStartPar
Do monitoringu można wykorzystać zapytania SQL lub narzędzia zewnętrzne (np. pgAdmin, pgBadger).


\subsubsection{14.3 Optymalizacja rozmiaru}
\label{\detokenize{rozdzial2/Konfiguracja_baz_danych/Konfiguracja_baz_danych:optymalizacja-rozmiaru}}
\sphinxAtStartPar
Działania optymalizacyjne obejmują:
\begin{itemize}
\item {} 
\sphinxAtStartPar
\sphinxstylestrong{Reorganizację i VACUUM}: odzyskiwanie przestrzeni po usuniętych lub zaktualizowanych rekordach oraz poprawa statystyk.

\item {} 
\sphinxAtStartPar
\sphinxstylestrong{Partycjonowanie tabel}: dzielenie dużych tabel na mniejsze, co ułatwia zarządzanie.

\item {} 
\sphinxAtStartPar
\sphinxstylestrong{Ograniczenia i typy danych}: odpowiedni dobór typów danych (np. \sphinxcode{\sphinxupquote{varchar(n)}} zamiast \sphinxcode{\sphinxupquote{text}}) oraz stosowanie ograniczeń (np. CHECK) zmniejsza rozmiar danych.

\end{itemize}


\subsubsection{14.4 Znaczenie zarządzania rozmiarem}
\label{\detokenize{rozdzial2/Konfiguracja_baz_danych/Konfiguracja_baz_danych:znaczenie-zarzadzania-rozmiarem}}
\sphinxAtStartPar
Niewłaściwe zarządzanie przestrzenią dyskową może prowadzić do:
\begin{itemize}
\item {} 
\sphinxAtStartPar
Spowolnienia działania bazy.

\item {} 
\sphinxAtStartPar
Problemów z backupem i odtwarzaniem.

\item {} 
\sphinxAtStartPar
Wzrostu kosztów utrzymania infrastruktury.

\end{itemize}


\subsection{Podsumowanie}
\label{\detokenize{rozdzial2/Konfiguracja_baz_danych/Konfiguracja_baz_danych:podsumowanie}}
\sphinxAtStartPar
Zarządzanie konfiguracją bazy danych PostgreSQL, optymalizacja zapytań oraz monitorowanie i konserwacja tabel stanowią fundament skutecznego zarządzania systemem bazodanowym. Prawidłowe podejście do tych elementów zapewnia wysoką wydajność, niezawodność i skalowalność systemu.

\sphinxAtStartPar
—

\sphinxstepscope


\section{Kopie zapasowe i odzyskiwanie danych w PostgreSQL}
\label{\detokenize{rozdzial2/Kopie_zapasowe_i_odzyskiwanie_danych/kopie_zapasowe_i_odzyskiwanie_danych:kopie-zapasowe-i-odzyskiwanie-danych-w-postgresql}}\label{\detokenize{rozdzial2/Kopie_zapasowe_i_odzyskiwanie_danych/kopie_zapasowe_i_odzyskiwanie_danych::doc}}\begin{quote}\begin{description}
\sphinxlineitem{Autorzy}
\sphinxAtStartPar
Miłosz Śmieja Szymon Piskorz Mateusz Wasilewicz

\end{description}\end{quote}


\subsection{Wprowadzenie}
\label{\detokenize{rozdzial2/Kopie_zapasowe_i_odzyskiwanie_danych/kopie_zapasowe_i_odzyskiwanie_danych:wprowadzenie}}
\sphinxAtStartPar
System zarządzania bazą danych PostgreSQL oferuje kompleksowy zestaw narzędzi i mechanizmów służących do tworzenia kopii zapasowych oraz odzyskiwania danych. Skuteczne zarządzanie kopiami zapasowymi stanowi fundament bezpieczeństwa danych i ciągłości działania systemów bazodanowych.

\sphinxAtStartPar
PostgreSQL dostarcza zarówno mechanizmy wbudowane, jak i możliwość integracji z zewnętrznymi narzędziami automatyzacji.


\subsection{Mechanizmy wbudowane do tworzenia kopii zapasowych całego systemu PostgreSQL}
\label{\detokenize{rozdzial2/Kopie_zapasowe_i_odzyskiwanie_danych/kopie_zapasowe_i_odzyskiwanie_danych:mechanizmy-wbudowane-do-tworzenia-kopii-zapasowych-calego-systemu-postgresql}}
\sphinxAtStartPar
PostgreSQL oferuje kilka mechanizmów tworzenia kopii zapasowych na poziomie całego systemu, które zapewniają kompleksową ochronę wszystkich baz danych w klastrze.


\subsubsection{pg\_basebackup}
\label{\detokenize{rozdzial2/Kopie_zapasowe_i_odzyskiwanie_danych/kopie_zapasowe_i_odzyskiwanie_danych:pg-basebackup}}
\sphinxAtStartPar
\sphinxstylestrong{pg\_basebackup} stanowi podstawowe narzędzie do tworzenia fizycznych kopii zapasowych całego klastra PostgreSQL.

\sphinxAtStartPar
Kluczowe cechy:
\begin{itemize}
\item {} 
\sphinxAtStartPar
Działa w trybie online \sphinxhyphen{} możliwość wykonywania kopii zapasowych bez zatrzymywania działania serwera

\item {} 
\sphinxAtStartPar
Tworzy dokładną kopię wszystkich plików danych

\item {} 
\sphinxAtStartPar
Zawiera pliki konfiguracyjne, dzienniki transakcji oraz wszystkie bazy danych w klastrze

\end{itemize}


\subsubsection{Continuous Archiving (Point\sphinxhyphen{}in\sphinxhyphen{}Time Recovery)}
\label{\detokenize{rozdzial2/Kopie_zapasowe_i_odzyskiwanie_danych/kopie_zapasowe_i_odzyskiwanie_danych:continuous-archiving-point-in-time-recovery}}
\sphinxAtStartPar
\sphinxstylestrong{Continuous Archiving} reprezentuje zaawansowany mechanizm tworzenia ciągłych kopii zapasowych poprzez archiwizację dzienników WAL (Write\sphinxhyphen{}Ahead Logging).

\sphinxAtStartPar
Zalety:
\begin{itemize}
\item {} 
\sphinxAtStartPar
Umożliwia odtworzenie stanu bazy danych w dowolnym momencie czasowym

\item {} 
\sphinxAtStartPar
Szczególnie wartościowe w środowiskach produkcyjnych wymagających minimalnej utraty danych

\item {} 
\sphinxAtStartPar
Zapewnia wysoką granularność odzyskiwania danych

\end{itemize}


\subsubsection{Streaming Replication}
\label{\detokenize{rozdzial2/Kopie_zapasowe_i_odzyskiwanie_danych/kopie_zapasowe_i_odzyskiwanie_danych:streaming-replication}}
\sphinxAtStartPar
\sphinxstylestrong{Streaming Replication} może służyć jako mechanizm kopii zapasowych poprzez utrzymywanie synchronicznych lub asynchronicznych replik głównej bazy danych.

\sphinxAtStartPar
Funkcjonalności:
\begin{itemize}
\item {} 
\sphinxAtStartPar
Repliki funkcjonują jako kopie zapasowe w czasie rzeczywistym

\item {} 
\sphinxAtStartPar
Oferuje możliwość szybkiego przełączenia w przypadku awarii systemu głównego

\item {} 
\sphinxAtStartPar
Wspiera zarówno tryb synchroniczny, jak i asynchroniczny

\end{itemize}


\subsubsection{File System Level Backup}
\label{\detokenize{rozdzial2/Kopie_zapasowe_i_odzyskiwanie_danych/kopie_zapasowe_i_odzyskiwanie_danych:file-system-level-backup}}
\sphinxAtStartPar
\sphinxstylestrong{File System Level Backup} polega na tworzeniu kopii zapasowych na poziomie systemu plików.

\sphinxAtStartPar
Wymagania:
\begin{itemize}
\item {} 
\sphinxAtStartPar
Zatrzymanie serwera PostgreSQL lub zapewnienie spójności

\item {} 
\sphinxAtStartPar
Wykorzystanie mechanizmów snapshot systemu plików:
\begin{itemize}
\item {} 
\sphinxAtStartPar
LVM snapshots

\item {} 
\sphinxAtStartPar
ZFS snapshots

\end{itemize}

\end{itemize}


\subsection{Mechanizmy wbudowane do tworzenia kopii zapasowych poszczególnych baz danych}
\label{\detokenize{rozdzial2/Kopie_zapasowe_i_odzyskiwanie_danych/kopie_zapasowe_i_odzyskiwanie_danych:mechanizmy-wbudowane-do-tworzenia-kopii-zapasowych-poszczegolnych-baz-danych}}
\sphinxAtStartPar
PostgreSQL dostarcza precyzyjne narzędzia umożliwiające tworzenie kopii zapasowych pojedynczych baz danych lub ich wybranych elementów.


\subsubsection{pg\_dump}
\label{\detokenize{rozdzial2/Kopie_zapasowe_i_odzyskiwanie_danych/kopie_zapasowe_i_odzyskiwanie_danych:pg-dump}}
\sphinxAtStartPar
\sphinxstylestrong{pg\_dump} stanowi najczęściej wykorzystywane narzędzie do tworzenia logicznych kopii zapasowych pojedynczych baz danych.

\sphinxAtStartPar
Charakterystyka:
\begin{itemize}
\item {} 
\sphinxAtStartPar
Tworzy skrypt SQL zawierający wszystkie polecenia niezbędne do odtworzenia struktury bazy danych oraz jej danych

\item {} 
\sphinxAtStartPar
Oferuje liczne opcje konfiguracji:
\begin{itemize}
\item {} 
\sphinxAtStartPar
Możliwość wyboru formatu wyjściowego

\item {} 
\sphinxAtStartPar
Filtrowanie obiektów

\item {} 
\sphinxAtStartPar
Kontrola nad poziomem szczegółowości kopii zapasowej

\end{itemize}

\end{itemize}


\subsubsection{pg\_dumpall}
\label{\detokenize{rozdzial2/Kopie_zapasowe_i_odzyskiwanie_danych/kopie_zapasowe_i_odzyskiwanie_danych:pg-dumpall}}
\sphinxAtStartPar
\sphinxstylestrong{pg\_dumpall} rozszerza funkcjonalność \sphinxcode{\sphinxupquote{pg\_dump}} o możliwość tworzenia kopii zapasowych wszystkich baz danych w klastrze.

\sphinxAtStartPar
Dodatkowe funkcje:
\begin{itemize}
\item {} 
\sphinxAtStartPar
Backup obiektów globalnych:
\begin{itemize}
\item {} 
\sphinxAtStartPar
Role użytkowników

\item {} 
\sphinxAtStartPar
Tablespaces

\item {} 
\sphinxAtStartPar
Ustawienia konfiguracyjne na poziomie klastra

\end{itemize}

\end{itemize}


\subsubsection{COPY command}
\label{\detokenize{rozdzial2/Kopie_zapasowe_i_odzyskiwanie_danych/kopie_zapasowe_i_odzyskiwanie_danych:copy-command}}
\sphinxAtStartPar
\sphinxstylestrong{COPY command} umożliwia eksport danych z poszczególnych tabel do plików w różnych formatach.

\sphinxAtStartPar
Obsługiwane formaty:
\begin{itemize}
\item {} 
\sphinxAtStartPar
CSV

\item {} 
\sphinxAtStartPar
Text

\item {} 
\sphinxAtStartPar
Binary

\end{itemize}

\sphinxAtStartPar
Zastosowania:
\begin{itemize}
\item {} 
\sphinxAtStartPar
Tworzenie selektywnych kopii zapasowych dużych tabel

\item {} 
\sphinxAtStartPar
Migracje danych

\end{itemize}


\subsubsection{pg\_dump z opcjami selektywnymi}
\label{\detokenize{rozdzial2/Kopie_zapasowe_i_odzyskiwanie_danych/kopie_zapasowe_i_odzyskiwanie_danych:pg-dump-z-opcjami-selektywnymi}}
\sphinxAtStartPar
\sphinxstylestrong{pg\_dump z opcjami selektywnymi} pozwala na tworzenie kopii zapasowych wybranych obiektów bazy danych.

\sphinxAtStartPar
Możliwości filtrowania:
\begin{itemize}
\item {} 
\sphinxAtStartPar
Konkretne tabele

\item {} 
\sphinxAtStartPar
Schematy

\item {} 
\sphinxAtStartPar
Sekwencje

\end{itemize}

\sphinxAtStartPar
Funkcjonalność ta jest nieoceniona w scenariuszach wymagających granularnej kontroli nad procesem tworzenia kopii zapasowych.


\subsection{Odzyskiwanie usuniętych lub uszkodzonych danych}
\label{\detokenize{rozdzial2/Kopie_zapasowe_i_odzyskiwanie_danych/kopie_zapasowe_i_odzyskiwanie_danych:odzyskiwanie-usunietych-lub-uszkodzonych-danych}}
\sphinxAtStartPar
PostgreSQL oferuje różnorodne mechanizmy odzyskiwania danych w zależności od rodzaju i zakresu uszkodzeń.


\subsubsection{Odzyskiwanie z kopii logicznych}
\label{\detokenize{rozdzial2/Kopie_zapasowe_i_odzyskiwanie_danych/kopie_zapasowe_i_odzyskiwanie_danych:odzyskiwanie-z-kopii-logicznych}}
\sphinxAtStartPar
\sphinxstylestrong{Odzyskiwanie z kopii logicznych} wykonanych przy użyciu \sphinxcode{\sphinxupquote{pg\_dump}} realizowane jest poprzez \sphinxcode{\sphinxupquote{psql}} lub \sphinxcode{\sphinxupquote{pg\_restore}}.

\sphinxAtStartPar
Proces odzyskiwania:
\begin{itemize}
\item {} 
\sphinxAtStartPar
Wykonanie skryptów SQL

\item {} 
\sphinxAtStartPar
Przywrócenie plików dump w odpowiednim formacie

\end{itemize}

\sphinxAtStartPar
Zaawansowane opcje pg\_restore:
\begin{itemize}
\item {} 
\sphinxAtStartPar
Selektywne przywracanie obiektów

\item {} 
\sphinxAtStartPar
Równoległe przetwarzanie

\item {} 
\sphinxAtStartPar
Kontrola nad kolejnością przywracania

\end{itemize}


\subsubsection{Point\sphinxhyphen{}in\sphinxhyphen{}Time Recovery (PITR)}
\label{\detokenize{rozdzial2/Kopie_zapasowe_i_odzyskiwanie_danych/kopie_zapasowe_i_odzyskiwanie_danych:point-in-time-recovery-pitr}}
\sphinxAtStartPar
\sphinxstylestrong{Point\sphinxhyphen{}in\sphinxhyphen{}Time Recovery (PITR)} umożliwia przywrócenie bazy danych do konkretnego momentu w czasie.

\sphinxAtStartPar
Wykorzystywane komponenty:
\begin{itemize}
\item {} 
\sphinxAtStartPar
Kombinacja kopii bazowej

\item {} 
\sphinxAtStartPar
Archiwalne dzienniki WAL

\end{itemize}

\sphinxAtStartPar
Zastosowania:
\begin{itemize}
\item {} 
\sphinxAtStartPar
Cofnięcie zmian do momentu poprzedzającego wystąpienie błędu

\item {} 
\sphinxAtStartPar
Odzyskiwanie po uszkodzeniu danych

\end{itemize}

\begin{sphinxadmonition}{note}{Informacja:}
\sphinxAtStartPar
PITR jest szczególnie wartościowy w przypadkach, gdy konieczne jest cofnięcie zmian do momentu poprzedzającego wystąpienie błędu lub uszkodzenia.
\end{sphinxadmonition}


\subsubsection{Odzyskiwanie tabel z tablespaces}
\label{\detokenize{rozdzial2/Kopie_zapasowe_i_odzyskiwanie_danych/kopie_zapasowe_i_odzyskiwanie_danych:odzyskiwanie-tabel-z-tablespaces}}
\sphinxAtStartPar
\sphinxstylestrong{Odzyskiwanie tabel z tablespaces} może wymagać specjalnych procedur w przypadku uszkodzenia przestrzeni tabel.

\sphinxAtStartPar
Możliwości PostgreSQL:
\begin{itemize}
\item {} 
\sphinxAtStartPar
Odtworzenie tablespaces

\item {} 
\sphinxAtStartPar
Przeniesienie tabel między różnymi lokalizacjami

\item {} 
\sphinxAtStartPar
Odzyskiwanie danych nawet w przypadku częściowego uszkodzenia systemu plików

\end{itemize}


\subsubsection{Transaction log replay}
\label{\detokenize{rozdzial2/Kopie_zapasowe_i_odzyskiwanie_danych/kopie_zapasowe_i_odzyskiwanie_danych:transaction-log-replay}}
\sphinxAtStartPar
\sphinxstylestrong{Transaction log replay} wykorzystuje dzienniki WAL do odtworzenia zmian wprowadzonych po utworzeniu kopii zapasowej.

\sphinxAtStartPar
Charakterystyka:
\begin{itemize}
\item {} 
\sphinxAtStartPar
Automatycznie wykorzystywany podczas standardowych procedur odzyskiwania

\item {} 
\sphinxAtStartPar
Możliwość ręcznej kontroli w szczególnych sytuacjach

\end{itemize}


\subsubsection{Odzyskiwanie na poziomie klastra}
\label{\detokenize{rozdzial2/Kopie_zapasowe_i_odzyskiwanie_danych/kopie_zapasowe_i_odzyskiwanie_danych:odzyskiwanie-na-poziomie-klastra}}
\sphinxAtStartPar
\sphinxstylestrong{Odzyskiwanie na poziomie klastra} przy wykorzystaniu \sphinxcode{\sphinxupquote{pg\_basebackup}} wymaga przywrócenia wszystkich plików klastra oraz odpowiedniej konfiguracji parametrów recovery.

\sphinxAtStartPar
Zakres procesu:
\begin{itemize}
\item {} 
\sphinxAtStartPar
Odtworzenie całego środowiska PostgreSQL

\item {} 
\sphinxAtStartPar
Konfiguracja ról i uprawnień

\item {} 
\sphinxAtStartPar
Przywrócenie ustawień systemowych

\end{itemize}


\subsection{Dedykowane oprogramowanie i skrypty zewnętrzne do automatyzacji}
\label{\detokenize{rozdzial2/Kopie_zapasowe_i_odzyskiwanie_danych/kopie_zapasowe_i_odzyskiwanie_danych:dedykowane-oprogramowanie-i-skrypty-zewnetrzne-do-automatyzacji}}
\sphinxAtStartPar
Automatyzacja procesów tworzenia kopii zapasowych stanowi kluczowy element profesjonalnego zarządzania bazami danych PostgreSQL.


\subsubsection{pgBackRest}
\label{\detokenize{rozdzial2/Kopie_zapasowe_i_odzyskiwanie_danych/kopie_zapasowe_i_odzyskiwanie_danych:pgbackrest}}
\sphinxAtStartPar
\sphinxstylestrong{pgBackRest} reprezentuje kompleksowe rozwiązanie do zarządzania kopiami zapasowymi PostgreSQL.

\sphinxAtStartPar
Zaawansowane funkcje:
\begin{itemize}
\item {} 
\sphinxAtStartPar
Incremental i differential backups

\item {} 
\sphinxAtStartPar
Kompresja danych

\item {} 
\sphinxAtStartPar
Szyfrowanie

\item {} 
\sphinxAtStartPar
Weryfikacja integralności kopii

\item {} 
\sphinxAtStartPar
Możliwość przechowywania kopii w chmurze

\item {} 
\sphinxAtStartPar
Automatyzacja procesów zarządzania kopiami zapasowymi

\item {} 
\sphinxAtStartPar
Uproszczone procedury odzyskiwania

\end{itemize}

\begin{sphinxadmonition}{important}{Ważne:}
\sphinxAtStartPar
pgBackRest automatyzuje wiele procesów związanych z zarządzaniem kopiami zapasowymi i znacznie upraszcza procedury odzyskiwania.
\end{sphinxadmonition}


\subsubsection{Barman (Backup and Recovery Manager)}
\label{\detokenize{rozdzial2/Kopie_zapasowe_i_odzyskiwanie_danych/kopie_zapasowe_i_odzyskiwanie_danych:barman-backup-and-recovery-manager}}
\sphinxAtStartPar
\sphinxstylestrong{Barman} stanowi dedykowane narzędzie stworzone przez 2ndQuadrant do zarządzania kopiami zapasowymi PostgreSQL w środowiskach enterprise.

\sphinxAtStartPar
Kluczowe funkcjonalności:
\begin{itemize}
\item {} 
\sphinxAtStartPar
Centralne zarządzanie kopiami zapasowymi wielu serwerów PostgreSQL

\item {} 
\sphinxAtStartPar
Monitoring procesów backup

\item {} 
\sphinxAtStartPar
Automatyczne testowanie procedur recovery

\item {} 
\sphinxAtStartPar
Integracja z narzędziami monitorowania

\end{itemize}


\subsubsection{WAL\sphinxhyphen{}E i WAL\sphinxhyphen{}G}
\label{\detokenize{rozdzial2/Kopie_zapasowe_i_odzyskiwanie_danych/kopie_zapasowe_i_odzyskiwanie_danych:wal-e-i-wal-g}}
\sphinxAtStartPar
\sphinxstylestrong{WAL\sphinxhyphen{}E i WAL\sphinxhyphen{}G} specjalizują się w archiwizacji dzienników WAL w środowiskach chmurowych.

\sphinxAtStartPar
Oferowane funkcje:
\begin{itemize}
\item {} 
\sphinxAtStartPar
Efektywna kompresja

\item {} 
\sphinxAtStartPar
Szyfrowanie danych

\item {} 
\sphinxAtStartPar
Przechowywanie kopii zapasowych w serwisach chmurowych:
\begin{itemize}
\item {} 
\sphinxAtStartPar
Amazon S3

\item {} 
\sphinxAtStartPar
Google Cloud Storage

\item {} 
\sphinxAtStartPar
Azure Blob Storage

\end{itemize}

\end{itemize}


\subsubsection{Skrypty shell i cron jobs}
\label{\detokenize{rozdzial2/Kopie_zapasowe_i_odzyskiwanie_danych/kopie_zapasowe_i_odzyskiwanie_danych:skrypty-shell-i-cron-jobs}}
\sphinxAtStartPar
\sphinxstylestrong{Skrypty shell i cron jobs} stanowią tradycyjne podejście do automatyzacji kopii zapasowych.

\sphinxAtStartPar
Możliwości automatyzacji:
\begin{itemize}
\item {} 
\sphinxAtStartPar
Wykonywanie \sphinxcode{\sphinxupquote{pg\_dump}} i \sphinxcode{\sphinxupquote{pg\_basebackup}}

\item {} 
\sphinxAtStartPar
Zarządzanie cyklem życia kopii zapasowych

\item {} 
\sphinxAtStartPar
Rotacja i czyszczenie starych kopii

\end{itemize}

\begin{sphinxadmonition}{tip}{Wskazówka:}
\sphinxAtStartPar
Właściwie napisane skrypty mogą automatyzować wykonywanie pg\_dump, pg\_basebackup oraz zarządzanie cyklem życia kopii zapasowych, w tym rotację i czyszczenie starych kopii.
\end{sphinxadmonition}


\subsubsection{Narzędzia automatyzacji infrastruktury}
\label{\detokenize{rozdzial2/Kopie_zapasowe_i_odzyskiwanie_danych/kopie_zapasowe_i_odzyskiwanie_danych:narzedzia-automatyzacji-infrastruktury}}
\sphinxAtStartPar
\sphinxstylestrong{Ansible, Puppet, Chef} jako narzędzia automatyzacji infrastruktury mogą być wykorzystywane do zarządzania konfiguracją procesów backup na większą skalę.

\sphinxAtStartPar
Korzyści:
\begin{itemize}
\item {} 
\sphinxAtStartPar
Standaryzacja procedur backup w środowiskach wieloserwerowych

\item {} 
\sphinxAtStartPar
Zapewnienie konsystentności konfiguracji

\item {} 
\sphinxAtStartPar
Skalowalne zarządzanie infrastrukturą

\end{itemize}


\subsubsection{Monitoring i alertowanie}
\label{\detokenize{rozdzial2/Kopie_zapasowe_i_odzyskiwanie_danych/kopie_zapasowe_i_odzyskiwanie_danych:monitoring-i-alertowanie}}
\sphinxAtStartPar
\sphinxstylestrong{Prometheus i Grafana} w połączeniu z \sphinxcode{\sphinxupquote{postgres\_exporter}} umożliwiają monitoring procesów backup oraz alertowanie w przypadku niepowodzeń.

\sphinxAtStartPar
Zakres monitorowania:
\begin{itemize}
\item {} 
\sphinxAtStartPar
Śledzenie czasu wykonywania kopii

\item {} 
\sphinxAtStartPar
Monitorowanie rozmiaru kopii zapasowych

\item {} 
\sphinxAtStartPar
Wskaźnik sukcesu procesów backup

\item {} 
\sphinxAtStartPar
Alertowanie w czasie rzeczywistym

\end{itemize}


\subsection{Podsumowanie}
\label{\detokenize{rozdzial2/Kopie_zapasowe_i_odzyskiwanie_danych/kopie_zapasowe_i_odzyskiwanie_danych:podsumowanie}}
\sphinxAtStartPar
Skuteczne zarządzanie kopiami zapasowymi w PostgreSQL wymaga kombinacji mechanizmów wbudowanych oraz zewnętrznych narzędzi automatyzacji. Wybór odpowiedniej strategii backup zależy od specyficznych wymagań organizacji, w tym:
\begin{itemize}
\item {} 
\sphinxAtStartPar
\sphinxstylestrong{RTO (Recovery Time Objective)} \sphinxhyphen{} maksymalny akceptowalny czas odzyskiwania

\item {} 
\sphinxAtStartPar
\sphinxstylestrong{RPO (Recovery Point Objective)} \sphinxhyphen{} maksymalna akceptowalna utrata danych

\item {} 
\sphinxAtStartPar
Dostępne zasoby

\item {} 
\sphinxAtStartPar
Złożoność środowiska

\end{itemize}


\subsubsection{Kluczowe wnioski}
\label{\detokenize{rozdzial2/Kopie_zapasowe_i_odzyskiwanie_danych/kopie_zapasowe_i_odzyskiwanie_danych:kluczowe-wnioski}}
\sphinxAtStartPar
\sphinxstylestrong{Mechanizmy wbudowane} PostgreSQL, takie jak \sphinxcode{\sphinxupquote{pg\_dump}}, \sphinxcode{\sphinxupquote{pg\_basebackup}} czy PITR, oferują solidne podstawy dla większości scenariuszy backup i recovery.

\sphinxAtStartPar
\sphinxstylestrong{W środowiskach produkcyjnych} o wysokich wymaganiach dotyczących dostępności i niezawodności, integracja z dedykowanymi narzędziami takimi jak pgBackRest czy Barman staje się niezbędna.


\subsubsection{Najważniejsze zalecenia}
\label{\detokenize{rozdzial2/Kopie_zapasowe_i_odzyskiwanie_danych/kopie_zapasowe_i_odzyskiwanie_danych:najwazniejsze-zalecenia}}
\begin{sphinxadmonition}{warning}{Ostrzeżenie:}
\sphinxAtStartPar
Kluczowym elementem każdej strategii backup jest regularne testowanie procedur odzyskiwania danych. Kopie zapasowe mają wartość tylko wtedy, gdy można z nich skutecznie odzyskać dane w sytuacji kryzysowej.
\end{sphinxadmonition}

\sphinxAtStartPar
\sphinxstylestrong{Kompleksowa strategia backup} powinna obejmować:
\begin{enumerate}
\sphinxsetlistlabels{\arabic}{enumi}{enumii}{}{.}%
\item {} 
\sphinxAtStartPar
Tworzenie kopii zapasowych

\item {} 
\sphinxAtStartPar
Regularne testy restore

\item {} 
\sphinxAtStartPar
Dokumentację procedur

\item {} 
\sphinxAtStartPar
Szkolenie personelu odpowiedzialnego za zarządzanie bazami danych

\end{enumerate}

\sphinxstepscope


\section{Kontrola i konserwacja baz danych}
\label{\detokenize{rozdzial2/Kontrola_i_konserwacja/kontrola_i_konserwacja:kontrola-i-konserwacja-baz-danych}}\label{\detokenize{rozdzial2/Kontrola_i_konserwacja/kontrola_i_konserwacja::doc}}

\subsection{Wprowadzenie}
\label{\detokenize{rozdzial2/Kontrola_i_konserwacja/kontrola_i_konserwacja:wprowadzenie}}
\sphinxAtStartPar
Autor: Bartłomiej Czyż

\sphinxAtStartPar
Systemy baz danych są niezwykle ważnym elementem infrastruktury informatycznej współczesnych organizacji. Umożliwiają przechowywanie, zarządzanie i analizę danych w sposób bezpieczny oraz wydajny. Aby zapewnić ich niezawodność, integralność i wysoką dostępność, konieczne jest prowadzenie regularnych działań z zakresu kontroli i konserwacji. Działania te można podzielić na część fizyczną oraz część programową, a sposób ich przeprowadzania różni się w zależności od rodzaju i architektury używanej bazy danych.


\subsection{Podział konserwacji baz danych}
\label{\detokenize{rozdzial2/Kontrola_i_konserwacja/kontrola_i_konserwacja:podzial-konserwacji-baz-danych}}
\sphinxAtStartPar
Autor: Bartłomiej Czyż


\subsubsection{Konserwacja fizyczna}
\label{\detokenize{rozdzial2/Kontrola_i_konserwacja/kontrola_i_konserwacja:konserwacja-fizyczna}}
\sphinxAtStartPar
Konserwacja fizyczna obejmuje wszystkie działania związane z infrastrukturą sprzętową i zasobami systemowymi, na których działa baza danych. Do najważniejszych elementów tej konserwacji należą:
\begin{itemize}
\item {} 
\sphinxAtStartPar
Monitorowanie stanu dysków twardych \textendash{} pozostała przestrzeń na dyskach, zużycie dysków oraz fragmentacja danych,

\item {} 
\sphinxAtStartPar
Zabezpieczenie fizyczne serwerów \textendash{} kontrola dostępu, ochrona przeciwpożarowa, klimatyzacja,

\item {} 
\sphinxAtStartPar
Zasilanie awaryjne (UPS) \sphinxhyphen{} zabezpieczenie bazy przed skutkami nagłego zaniku zasilania,

\item {} 
\sphinxAtStartPar
Monitoring stanu sieci \textendash{} wydajność i stabilność połączenia między bazą a klientami,

\item {} 
\sphinxAtStartPar
Tworzenie kopii zapasowych na nośnikach fizycznych \textendash{} np. dyskach zewnętrznych czy taśmach LTO.

\end{itemize}


\subsubsection{Konserwacja programowa}
\label{\detokenize{rozdzial2/Kontrola_i_konserwacja/kontrola_i_konserwacja:konserwacja-programowa}}
\sphinxAtStartPar
Konserwacja programowa odnosi się do czynności wykonywanych na poziomie oprogramowania i logiki działania systemu bazy danych. Obejmuje:
\begin{itemize}
\item {} 
\sphinxAtStartPar
Zarządzanie użytkownikami i ich uprawnieniami,

\item {} 
\sphinxAtStartPar
Optymalizację zapytań SQL,

\item {} 
\sphinxAtStartPar
Aktualizację oprogramowania bazodanowego (np. MySQL, PostgreSQL),

\item {} 
\sphinxAtStartPar
Defragmentację indeksów,

\item {} 
\sphinxAtStartPar
Weryfikację integralności danych i naprawę uszkodzonych rekordów,

\item {} 
\sphinxAtStartPar
Automatyczne zadania konserwacyjne (cron, schedulery),

\item {} 
\sphinxAtStartPar
Reduplikację i redundancję \sphinxhyphen{} konfiguracja serwerów zapasowych.

\end{itemize}


\subsection{Różnice konserwacyjne w zależności od rodzaju bazy danych}
\label{\detokenize{rozdzial2/Kontrola_i_konserwacja/kontrola_i_konserwacja:roznice-konserwacyjne-w-zaleznosci-od-rodzaju-bazy-danych}}
\sphinxAtStartPar
Autor: Bartłomiej Czyż


\subsubsection{PostgreSQL}
\label{\detokenize{rozdzial2/Kontrola_i_konserwacja/kontrola_i_konserwacja:postgresql}}
\sphinxAtStartPar
PostgreSQL to zaawansowany system RDBMS, znany z silnego wsparcia dla różnych typów danych i transakcyjności.
\begin{enumerate}
\sphinxsetlistlabels{\arabic}{enumi}{enumii}{}{.}%
\item {} 
\sphinxAtStartPar
Fizyczna konserwacja:
\begin{itemize}
\item {} 
\sphinxAtStartPar
Złożona struktura katalogów danych (base, pg\_wal, pg\_tblspc) \textendash{} wymaga regularnego monitoringu,

\item {} 
\sphinxAtStartPar
Możliwość wykorzystania narzędzia pg\_basebackup do tworzenia pełnych kopii fizycznych.

\end{itemize}

\item {} 
\sphinxAtStartPar
Programowa konserwacja:
\begin{itemize}
\item {} 
\sphinxAtStartPar
Automatyczne zadania VACUUM, ANALYZE \textendash{} zapewniają odzyskiwanie przestrzeni po usunięciu rekordów,

\item {} 
\sphinxAtStartPar
Możliwość używania pg\_repack do defragmentacji bez przestojów,

\item {} 
\sphinxAtStartPar
Silne wsparcie dla replikacji strumieniowej i klastrów wysokiej dostępności (HA).

\end{itemize}

\end{enumerate}


\subsubsection{MySQL}
\label{\detokenize{rozdzial2/Kontrola_i_konserwacja/kontrola_i_konserwacja:mysql}}
\sphinxAtStartPar
MySQL jest obecnie jedną z najpopularniejszych relacyjnych baz danych, szeroko stosowana w aplikacjach webowych.
\begin{enumerate}
\sphinxsetlistlabels{\arabic}{enumi}{enumii}{}{.}%
\item {} 
\sphinxAtStartPar
Fizyczna konserwacja:
\begin{itemize}
\item {} 
\sphinxAtStartPar
Wymaga monitorowania plików .ibd (w przypadku silknika InnoDB), które mogą znacznie rosnąć,

\item {} 
\sphinxAtStartPar
Backup danych realizowany poprzez mysqldump lub system replikacji binlogów.

\end{itemize}

\item {} 
\sphinxAtStartPar
Programowa konserwacja:
\begin{itemize}
\item {} 
\sphinxAtStartPar
Regularne sprawdzanie indeksów (ANALYZE TABLE, OPTIMIZE TABLE),

\item {} 
\sphinxAtStartPar
Używanie narzędzi typu mysqlcheck do weryfikacji i naprawy tabel,

\item {} 
\sphinxAtStartPar
Konfiguracja pliku my.cnf w celu dostosowania do wymagań aplikacji.

\end{itemize}

\end{enumerate}


\subsubsection{SQLite (np. LightSQL)}
\label{\detokenize{rozdzial2/Kontrola_i_konserwacja/kontrola_i_konserwacja:sqlite-np-lightsql}}
\sphinxAtStartPar
SQLite, używana w aplikacjach mobilnych i desktopowych, różni się znacznie od serwerowych baz danych.
\begin{enumerate}
\sphinxsetlistlabels{\arabic}{enumi}{enumii}{}{.}%
\item {} 
\sphinxAtStartPar
Fizyczna konserwacja:
\begin{itemize}
\item {} 
\sphinxAtStartPar
Brak klasycznego serwera \textendash{} baza to pojedynczy plik .db,

\item {} 
\sphinxAtStartPar
Konieczność regularnego kopiowania pliku bazy danych jako backup.

\end{itemize}

\item {} 
\sphinxAtStartPar
Programowa konserwacja:
\begin{itemize}
\item {} 
\sphinxAtStartPar
Użycie polecenia VACUUM do defragmentacji i zmniejszenia rozmiaru pliku,

\item {} 
\sphinxAtStartPar
Ograniczone możliwości równoczesnego dostępu \textendash{} wymaga uwagi w aplikacjach wielowątkowych,

\item {} 
\sphinxAtStartPar
Nie wymaga osobnych usług do zarządzania \textendash{} działa bezpośrednio w aplikacji.

\end{itemize}

\end{enumerate}


\subsubsection{Microsoft SQL Server}
\label{\detokenize{rozdzial2/Kontrola_i_konserwacja/kontrola_i_konserwacja:microsoft-sql-server}}
\sphinxAtStartPar
System korporacyjny, szeroko wykorzystywany w dużych organizacjach.
\begin{enumerate}
\sphinxsetlistlabels{\arabic}{enumi}{enumii}{}{.}%
\item {} 
\sphinxAtStartPar
Fizyczna konserwacja:
\begin{itemize}
\item {} 
\sphinxAtStartPar
Obsługuje macierze RAID i pamięci masowe SAN,

\item {} 
\sphinxAtStartPar
Regularne kopie pełne, różnicowe i dzienniki transakcyjne.

\end{itemize}

\item {} 
\sphinxAtStartPar
Programowa konserwacja:
\begin{itemize}
\item {} 
\sphinxAtStartPar
Zaawansowany SQL Server Agent \textendash{} możliwość harmonogramowania zadań,

\item {} 
\sphinxAtStartPar
Narzędzia do monitorowania stanu instancji (SQL Profiler, Database Tuning Advisor),

\item {} 
\sphinxAtStartPar
Wsparcie dla Always On Availability Groups dla wysokiej dostępności.

\end{itemize}

\end{enumerate}


\subsection{Planowanie konserwacji bazy danych}
\label{\detokenize{rozdzial2/Kontrola_i_konserwacja/kontrola_i_konserwacja:planowanie-konserwacji-bazy-danych}}
\sphinxAtStartPar
Autor: Piotr Mikołajczyk

\sphinxAtStartPar
Konserwację bazy danych należy przeprowadzać regularnie, np. co tydzień lub co miesiąc. Nie powinna mieć miejsca w godzinach szczytu. Przeprowadzenie konserwacji może również okazać się koniecznie po wykryciu błędu lub wystąpieniu awarii.

\sphinxAtStartPar
Konserwacja może obejmować m.in. zmianę parametrów konfiguracji bazy, przeprowadzenie procesu VACUUM, zmianę uprawnien użytkowników, aktualizacje systemowe i wykonanie backupów lub przywrócenie danych.

\sphinxAtStartPar
Działanie te muszą zostać przeprowadzone w czasie, gdy mamy pewność, że żaden klient nie będzie podłączony, nie będą przeprowadzane żadne transakcje. Użytkownicy powinni być uprzednio poinformowani o czasie przeprowadzenia konserwacji. Mimo to, należy wcześniej sprawdzić, czy nie ma aktywnych sesji.


\subsection{Uruchamianie, zatrzymywanie i restartowanie serwera bazy danych}
\label{\detokenize{rozdzial2/Kontrola_i_konserwacja/kontrola_i_konserwacja:uruchamianie-zatrzymywanie-i-restartowanie-serwera-bazy-danych}}
\sphinxAtStartPar
Autor: Piotr Mikołajczyk

\sphinxAtStartPar
Działania, takie jak aktualizacja oprogramowania, instalacja rozszerzeń, wprowadzenie pewnych zmian w plikach konfiguracyjnych, migracja danych, wykonanie backupów bazy, wymagają zrestartowania, zatrzymania bądź ponownego uruchomienia serwera bazy danych.


\subsubsection{Uruchamianie}
\label{\detokenize{rozdzial2/Kontrola_i_konserwacja/kontrola_i_konserwacja:uruchamianie}}
\sphinxAtStartPar
Linux:

\begin{sphinxVerbatim}[commandchars=\\\{\}]
sudo\PYG{+w}{ }systemctl\PYG{+w}{ }start\PYG{+w}{ }postgresql
\end{sphinxVerbatim}

\sphinxAtStartPar
Windows CMD:

\begin{sphinxVerbatim}[commandchars=\\\{\}]
net start postgresql\PYGZhy{}x64\PYGZhy{}15
\end{sphinxVerbatim}

\sphinxAtStartPar
Windows PowerShell

\begin{sphinxVerbatim}[commandchars=\\\{\}]
\PYG{n+nb}{Start\PYGZhy{}Service} \PYG{n}{\PYGZhy{}Name} \PYG{n}{postgresql}\PYG{n}{\PYGZhy{}x64}\PYG{p}{\PYGZhy{}}\PYG{n}{15}
\end{sphinxVerbatim}


\subsubsection{Zatrzymywanie}
\label{\detokenize{rozdzial2/Kontrola_i_konserwacja/kontrola_i_konserwacja:zatrzymywanie}}
\sphinxAtStartPar
Linux:

\begin{sphinxVerbatim}[commandchars=\\\{\}]
sudo\PYG{+w}{ }systemctl\PYG{+w}{ }stop\PYG{+w}{ }postgresql
\end{sphinxVerbatim}

\sphinxAtStartPar
Windows CMD:

\begin{sphinxVerbatim}[commandchars=\\\{\}]
net stop postgresql\PYGZhy{}x64\PYGZhy{}15
\end{sphinxVerbatim}

\sphinxAtStartPar
Windows PowerShell

\begin{sphinxVerbatim}[commandchars=\\\{\}]
\PYG{n+nb}{Stop\PYGZhy{}Service} \PYG{n}{\PYGZhy{}Name} \PYG{n}{postgresql}\PYG{n}{\PYGZhy{}x64}\PYG{p}{\PYGZhy{}}\PYG{n}{15}
\end{sphinxVerbatim}


\subsubsection{Restartowanie}
\label{\detokenize{rozdzial2/Kontrola_i_konserwacja/kontrola_i_konserwacja:restartowanie}}
\sphinxAtStartPar
Linux:

\begin{sphinxVerbatim}[commandchars=\\\{\}]
sudo\PYG{+w}{ }systemctl\PYG{+w}{ }restart\PYG{+w}{ }postgresql
\end{sphinxVerbatim}

\sphinxAtStartPar
W CMD nie istnieje osobne polecenie restartowania. Należy zatrzymać serwer, a następnie uruchomić go ponownie.

\sphinxAtStartPar
Windows PowerShell

\begin{sphinxVerbatim}[commandchars=\\\{\}]
\PYG{n+nb}{Restart\PYGZhy{}Service} \PYG{n}{\PYGZhy{}Name} \PYG{n}{postgresql}\PYG{n}{\PYGZhy{}x64}\PYG{p}{\PYGZhy{}}\PYG{n}{15}
\end{sphinxVerbatim}

\sphinxAtStartPar
Polecenia CMD mogą zostać również użyte w PowerShell.


\subsection{Zarządzanie połączeniami użytkowników}
\label{\detokenize{rozdzial2/Kontrola_i_konserwacja/kontrola_i_konserwacja:zarzadzanie-polaczeniami-uzytkownikow}}
\sphinxAtStartPar
Autor: Piotr Mikołajczyk

\sphinxAtStartPar
Oprócz sytuacji, gdy trzeba zamknąć dostęp do bazy danych na czas konserwacji, połączenia użytkowników należy ograniczyć także wtedy, gdy sesja użytkownika została zawieszona lub zbyt wiele połączeń skutkuje nadmiernym zużyciem pamięci i mocy obliczeniowej, uniemożliwiając nawiązywanie nowych połączeń i spowolniając działanie serwera.


\subsubsection{Ograniczanie użytkowników}
\label{\detokenize{rozdzial2/Kontrola_i_konserwacja/kontrola_i_konserwacja:ograniczanie-uzytkownikow}}
\sphinxAtStartPar
Istnieje kilka sposobów ograniczenia dostępu użytkownika:
\begin{itemize}
\item {} 
\sphinxAtStartPar
Odebranie użytkownikowi prawa dostępu do bazy:
\begin{quote}

\begin{sphinxVerbatim}[commandchars=\\\{\}]
\PYG{k}{REVOKE}\PYG{+w}{ }\PYG{k}{CONNECT}\PYG{+w}{ }\PYG{k}{ON}\PYG{+w}{ }\PYG{k}{DATABASE}\PYG{+w}{ }\PYG{n}{baza}\PYG{+w}{ }\PYG{k}{FROM}\PYG{+w}{ }\PYG{k}{user}\PYG{p}{;}
\end{sphinxVerbatim}
\end{quote}

\item {} 
\sphinxAtStartPar
Limit liczby jednoczesnych połączeń:
\begin{quote}

\begin{sphinxVerbatim}[commandchars=\\\{\}]
\PYG{k}{ALTER}\PYG{+w}{ }\PYG{k}{ROLE}\PYG{+w}{ }\PYG{k}{user}\PYG{+w}{ }\PYG{k}{CONNECTION}\PYG{+w}{ }\PYG{k}{LIMIT}\PYG{+w}{ }\PYG{l+m+mi}{3}\PYG{p}{;}
\end{sphinxVerbatim}
\end{quote}

\end{itemize}


\subsubsection{Ręczne rozłączanie użytkowników}
\label{\detokenize{rozdzial2/Kontrola_i_konserwacja/kontrola_i_konserwacja:reczne-rozlaczanie-uzytkownikow}}
\sphinxAtStartPar
Według nazwy danego użytkownika:
\begin{quote}

\begin{sphinxVerbatim}[commandchars=\\\{\}]
\PYG{k}{SELECT}\PYG{+w}{ }\PYG{n}{pg\PYGZus{}terminate\PYGZus{}backend}\PYG{p}{(}\PYG{n}{pid}\PYG{p}{)}
\PYG{k}{FROM}\PYG{+w}{ }\PYG{n}{pg\PYGZus{}stat\PYGZus{}activity}
\PYG{k}{WHERE}\PYG{+w}{ }\PYG{n}{usename}\PYG{+w}{ }\PYG{o}{=}\PYG{+w}{ }\PYG{l+s+s1}{\PYGZsq{}user\PYGZsq{}}\PYG{p}{;}
\end{sphinxVerbatim}
\end{quote}

\sphinxAtStartPar
Według PID (np. 12340):
\begin{quote}

\begin{sphinxVerbatim}[commandchars=\\\{\}]
\PYG{k}{SELECT}\PYG{+w}{ }\PYG{n}{pg\PYGZus{}terminate\PYGZus{}backend}\PYG{p}{(}\PYG{l+m+mi}{12340}\PYG{p}{)}\PYG{p}{;}
\end{sphinxVerbatim}
\end{quote}


\subsubsection{Automatyczne rozłączanie użytkowników}
\label{\detokenize{rozdzial2/Kontrola_i_konserwacja/kontrola_i_konserwacja:automatyczne-rozlaczanie-uzytkownikow}}
\sphinxAtStartPar
Sesja użytkownika lub jego zapytania mogą zostać rozłączone automatycznie, jeśli wprowadzimy pewne ograniczenia czasowe:
\begin{itemize}
\item {} 
\sphinxAtStartPar
Rozłączenie sesji po przekroczeniu limitu czasu bezczynności podczas zapytania:
\begin{itemize}
\item {} 
\sphinxAtStartPar
dla bieżącej sesji:
\begin{quote}

\begin{sphinxVerbatim}[commandchars=\\\{\}]
\PYG{k}{SET}\PYG{+w}{ }\PYG{n}{idle\PYGZus{}in\PYGZus{}transaction\PYGZus{}session\PYGZus{}timeout}\PYG{+w}{ }\PYG{o}{=}\PYG{+w}{ }\PYG{l+s+s1}{\PYGZsq{}5min\PYGZsq{}}\PYG{p}{;}
\end{sphinxVerbatim}
\end{quote}

\item {} 
\sphinxAtStartPar
dla danego użytkownika:
\begin{quote}

\begin{sphinxVerbatim}[commandchars=\\\{\}]
\PYG{k}{ALTER}\PYG{+w}{ }\PYG{k}{ROLE}\PYG{+w}{ }\PYG{k}{user}\PYG{+w}{ }\PYG{k}{SET}\PYG{+w}{ }\PYG{n}{idle\PYGZus{}in\PYGZus{}transaction\PYGZus{}session\PYGZus{}timeout}\PYG{+w}{ }\PYG{o}{=}\PYG{+w}{ }\PYG{l+s+s1}{\PYGZsq{}5min\PYGZsq{}}\PYG{p}{;}
\end{sphinxVerbatim}
\end{quote}

\end{itemize}

\item {} 
\sphinxAtStartPar
Limit czasu zapytania:
\begin{quote}

\begin{sphinxVerbatim}[commandchars=\\\{\}]
\PYG{k}{ALTER}\PYG{+w}{ }\PYG{k}{ROLE}\PYG{+w}{ }\PYG{k}{user}\PYG{+w}{ }\PYG{k}{SET}\PYG{+w}{ }\PYG{n}{statement\PYGZus{}timeout}\PYG{+w}{ }\PYG{o}{=}\PYG{+w}{ }\PYG{l+s+s1}{\PYGZsq{}30s\PYGZsq{}}\PYG{p}{;}
\end{sphinxVerbatim}
\end{quote}

\end{itemize}


\subsubsection{Zapobieganie nowym połączeniom}
\label{\detokenize{rozdzial2/Kontrola_i_konserwacja/kontrola_i_konserwacja:zapobieganie-nowym-polaczeniom}}
\sphinxAtStartPar
Zablokowanie logowania konkretnego użytkownika:
\begin{quote}

\begin{sphinxVerbatim}[commandchars=\\\{\}]
\PYG{k}{ALTER}\PYG{+w}{ }\PYG{k}{ROLE}\PYG{+w}{ }\PYG{k}{user}\PYG{+w}{ }\PYG{n}{NOLOGIN}\PYG{p}{;}
\end{sphinxVerbatim}

\sphinxAtStartPar
Odblokowanie:

\begin{sphinxVerbatim}[commandchars=\\\{\}]
\PYG{k}{ALTER}\PYG{+w}{ }\PYG{k}{ROLE}\PYG{+w}{ }\PYG{k}{user}\PYG{+w}{ }\PYG{n}{LOGIN}\PYG{p}{;}
\end{sphinxVerbatim}
\end{quote}

\sphinxAtStartPar
Blokowanie nowych połączeń do bazy danych:
\begin{quote}

\begin{sphinxVerbatim}[commandchars=\\\{\}]
\PYG{k}{REVOKE}\PYG{+w}{ }\PYG{k}{CONNECT}\PYG{+w}{ }\PYG{k}{ON}\PYG{+w}{ }\PYG{k}{DATABASE}\PYG{+w}{ }\PYG{n}{baza}\PYG{+w}{ }\PYG{k}{FROM}\PYG{+w}{ }\PYG{k}{PUBLIC}\PYG{p}{;}
\end{sphinxVerbatim}

\sphinxAtStartPar
PUBLIC oznacza wszystkich użytkowników. Nadal połączeni użytkownicy nie są rozłączani.
\end{quote}


\subsection{Proces VACUUM}
\label{\detokenize{rozdzial2/Kontrola_i_konserwacja/kontrola_i_konserwacja:proces-vacuum}}
\sphinxAtStartPar
Autor: Piotr Mikołajczyk

\sphinxAtStartPar
DELETE nie usuwa rekordów z tabeli, jedynie oznacza je jako martwe. Podobnie UPDATE pozostawia stare wersje zaktualizowanych krotek.

\sphinxAtStartPar
Proces VACUUM przeszukuje tabele i indeksy, szukając martwych wierszy, które można fizycznie usunąć lub oznaczyć do nadpisania.

\sphinxAtStartPar
Może zostać przeprowadzony na kilka sposobów:

\begin{sphinxVerbatim}[commandchars=\\\{\}]
\PYG{k}{VACUUM}\PYG{p}{;}
\end{sphinxVerbatim}

\sphinxAtStartPar
Usuwa martwe krotki, ale nie odzyskuje miejsca z dysku, a jedynie udostępnia je dla przyszłych danych,

\begin{sphinxVerbatim}[commandchars=\\\{\}]
\PYG{k}{VACUUM}\PYG{+w}{ }\PYG{k}{FULL}\PYG{p}{;}
\end{sphinxVerbatim}

\sphinxAtStartPar
Kompaktuje tabelę do nowego pliku, zwalnia miejsce w pamięci,

\begin{sphinxVerbatim}[commandchars=\\\{\}]
\PYG{k}{VACUUM}\PYG{+w}{ }\PYG{k}{ANALYZE}
\end{sphinxVerbatim}

\sphinxAtStartPar
Usuwa martwe krotki i przeprowadza aktualizację statystyk, nie odzyskuje miejsca.


\subsubsection{Autovacuum}
\label{\detokenize{rozdzial2/Kontrola_i_konserwacja/kontrola_i_konserwacja:autovacuum}}
\sphinxAtStartPar
Autovacuum działa w tle, automatycznie wykonując VACUUM na odpowiednich tabelach. Dzięki niemu nie trzeba ręcznie uruchamiać VACUUM po każdej modyfikacji tabeli. Autovacuum posiada wiele parametrów, od których zależy kiedy wykonany zostanie proces, między innymi:
\begin{itemize}
\item {} 
\sphinxAtStartPar
autovacuum \sphinxhyphen{} parametr logiczny, decyduje, czy serwer będzie uruchamiał launcher procesu autovacuum,

\item {} 
\sphinxAtStartPar
autovacuum\_max\_workers \sphinxhyphen{} liczba całkowita, określa maksymalną ilość procesów autovacuum mogących działać w tym samym czasie, domyślnie 3,

\item {} 
\sphinxAtStartPar
autovacuum\_vacuum\_threshold \sphinxhyphen{} liczba całkowita, określa ile wierszy w jednej tabeli musi zostać usunięte lub zmienione, aby wywołano VACUUM, domyślnie 50,

\item {} 
\sphinxAtStartPar
autovacuum\_vacuum\_scale\_factor \sphinxhyphen{} liczba zmiennoprzecinkowa, jaki procent tabeli musi zostać zmieniony aby wywołano VACUUM, domyślna wartość to 0.2 (20\%).

\end{itemize}

\sphinxAtStartPar
Analogiczne parametry warunkują również wywołanie ANALYZE, na przykład autovacuum\_analyze\_threshold.

\sphinxAtStartPar
Próg uruchamiania VACUUM ustala się wzorem:
\begin{quote}

\sphinxAtStartPar
autovacuum\_vacuum\_threshold + autovacuum\_vacuum\_scale\_factor * liczba\_wierszy
\end{quote}

\sphinxAtStartPar
Podobnie dla ANALYZE:
\begin{quote}

\sphinxAtStartPar
autovacuum\_analyze\_threshold + autovacuum\_analyze\_scale\_factor * liczba\_wierszy
\end{quote}


\subsection{Schemat bazy danych}
\label{\detokenize{rozdzial2/Kontrola_i_konserwacja/kontrola_i_konserwacja:schemat-bazy-danych}}
\sphinxAtStartPar
Autor: Bartłomiej Czyż


\subsubsection{Czym jest schemat bazy danych?}
\label{\detokenize{rozdzial2/Kontrola_i_konserwacja/kontrola_i_konserwacja:czym-jest-schemat-bazy-danych}}
\sphinxAtStartPar
Schemat bazy danych to logiczna struktura opisująca organizację danych, typy danych, relacje między tabelami, ograniczenia integralności, procedury składowane, widoki i inne obiekty. Innymi słowy, schemat jest „szkieletem” bazy danych.

\sphinxAtStartPar
Przykładowe elementy schematu:
\begin{itemize}
\item {} 
\sphinxAtStartPar
Tabele (np. users, orders),

\item {} 
\sphinxAtStartPar
Typy danych (np. INT, VARCHAR, DATE),

\item {} 
\sphinxAtStartPar
Klucze główne i obce,

\item {} 
\sphinxAtStartPar
Indeksy,

\item {} 
\sphinxAtStartPar
Widoki (VIEW),

\item {} 
\sphinxAtStartPar
Procedury i funkcje (STORED PROCEDURES),

\item {} 
\sphinxAtStartPar
Ograniczenia (CHECK, NOT NULL, UNIQUE).

\end{itemize}


\subsubsection{Rola schematu w konserwacji bazy danych}
\label{\detokenize{rozdzial2/Kontrola_i_konserwacja/kontrola_i_konserwacja:rola-schematu-w-konserwacji-bazy-danych}}
\sphinxAtStartPar
Schemat ma kluczowe znaczenie dla utrzymania spójności i integralności danych, dlatego jego kontrola i konserwacja obejmuje m.in.:
\begin{itemize}
\item {} 
\sphinxAtStartPar
Dokumentację schematu \sphinxhyphen{} niezbędna przy aktualizacjach i migracjach,

\item {} 
\sphinxAtStartPar
Weryfikację integralności relacji \sphinxhyphen{} sprawdzenie czy klucze obce i reguły są respektowane,

\item {} 
\sphinxAtStartPar
Normalizację \sphinxhyphen{} kontrola nad nadmiarem danych i poprawnością logiczną,

\item {} 
\sphinxAtStartPar
Aktualizacje schematu \sphinxhyphen{} np. dodawanie nowych kolumn, zmiana typu danych,

\item {} 
\sphinxAtStartPar
Kontrola zgodności \sphinxhyphen{} wersjonowanie schematu (np. za pomocą narzędzi typu Liquibase, Flyway),

\item {} 
\sphinxAtStartPar
Zabezpieczenia schematów \sphinxhyphen{} nadawanie uprawnień tylko zaufanym użytkownikom.

\end{itemize}

\sphinxAtStartPar
Przykład konserwacji:

\sphinxAtStartPar
W PostgreSQL można analizować i optymalizować strukturę przy pomocy pgAdmin oraz narzędzi takich jak pg\_dump \textendash{}schema\sphinxhyphen{}only.


\subsubsection{Różnice w implementacji schematu w różnych systemach}
\label{\detokenize{rozdzial2/Kontrola_i_konserwacja/kontrola_i_konserwacja:roznice-w-implementacji-schematu-w-roznych-systemach}}\begin{itemize}
\item {} 
\sphinxAtStartPar
MySQL \sphinxhyphen{} obsługuje wiele schematów w jednej bazie; ograniczone typy kolumn w starszych wersjach,

\item {} 
\sphinxAtStartPar
PostgreSQL \sphinxhyphen{} bardzo elastyczny system schematów \sphinxhyphen{} możliwość teorzenia przestrzeni nazw,

\item {} 
\sphinxAtStartPar
SQLite \sphinxhyphen{} pojedynczy schemat, uproszczony system typów,

\item {} 
\sphinxAtStartPar
SQL Server \sphinxhyphen{} schemat jako logiczna przestrzeń obiektów, np. dbo, hr, finance.

\end{itemize}


\subsection{Transakcje}
\label{\detokenize{rozdzial2/Kontrola_i_konserwacja/kontrola_i_konserwacja:transakcje}}
\sphinxAtStartPar
Autor: Bartłomiej Czyż


\subsubsection{Czym jest transakcja?}
\label{\detokenize{rozdzial2/Kontrola_i_konserwacja/kontrola_i_konserwacja:czym-jest-transakcja}}
\sphinxAtStartPar
Transakcja to zbiór operacji na bazie danych, które są traktowane jako jedna, nierozdzielna całość. Albo wykonują się wszystkie operacje, albo żadna \sphinxhyphen{} zasada atomiczności. Transakcje są podstawą do zachowania spójności danych, szczególnie w środowiskach wieloużytkownikowych.


\subsubsection{Zasady ACID}
\label{\detokenize{rozdzial2/Kontrola_i_konserwacja/kontrola_i_konserwacja:zasady-acid}}
\sphinxAtStartPar
Transakcje w bazach danych opierają się na czterech podstawowych zasadach, znanych jako ACID:
\begin{itemize}
\item {} 
\sphinxAtStartPar
A \sphinxhyphen{} Atomicity (Atomowość) \sphinxhyphen{} operacje wchodzące w skład transakcji są niepodzielne \sphinxhyphen{} wszystkie muszą się powieść, lub wszystkie są wycofywane,

\item {} 
\sphinxAtStartPar
C \sphinxhyphen{} Consistency (Spójność) \sphinxhyphen{} transakcje przekształcają dane ze stanu spójnego w stan spójny,

\item {} 
\sphinxAtStartPar
I \sphinxhyphen{} Isolation (Izolacja) \sphinxhyphen{} równoczesne transakcje nie wpływają na siebie nawzajem,

\item {} 
\sphinxAtStartPar
D \sphinxhyphen{} Durability (Trwałość) \sphinxhyphen{} po zatwierdzeniu transakcji dane są trwale zapisane, nawet w przypadku awarii.

\end{itemize}


\subsubsection{Rola transakcji w kontroli i konserwacji}
\label{\detokenize{rozdzial2/Kontrola_i_konserwacja/kontrola_i_konserwacja:rola-transakcji-w-kontroli-i-konserwacji}}
\sphinxAtStartPar
Transakcje mają ogromne znaczenie dla bezpieczeństwa danych, dlatego są nieodłącznym elementem procesów konserwacyjnych. Ich zastosowanie obejmuje:
\begin{itemize}
\item {} 
\sphinxAtStartPar
Zabezpieczenie operacji aktualizacji \sphinxhyphen{} np. przy masowych zmianach danych,

\item {} 
\sphinxAtStartPar
Replikacja i synchronizacja danych \sphinxhyphen{} transakcje zapewniają spójność między główną bazą, a replikami,

\item {} 
\sphinxAtStartPar
Zarządzanie błędami \sphinxhyphen{} w przypadku błędu można wykonać ROLLBACK i przywrócić stan bazy,

\item {} 
\sphinxAtStartPar
Tworzenie backupów spójnych z punktu w czasie \sphinxhyphen{} snapshoty danych często wymagają wsparcia transakcyjnego,

\item {} 
\sphinxAtStartPar
Ochrona przed uszkodzeniami logicznymi \sphinxhyphen{} np. przez niekompletne aktualizacje.

\end{itemize}


\subsubsection{Różnice w implementacji transakcji w różnych systemach}
\label{\detokenize{rozdzial2/Kontrola_i_konserwacja/kontrola_i_konserwacja:roznice-w-implementacji-transakcji-w-roznych-systemach}}\begin{itemize}
\item {} 
\sphinxAtStartPar
MySQL \sphinxhyphen{} w pełni wspierane w silniku InnoDB; START TRANSACTION, COMMIT, ROLLBACK,

\item {} 
\sphinxAtStartPar
PostgreSQL \sphinxhyphen{} silne wsparcie ACID, zaawansowana izolacja (REPEATABLE READ, SERIALIZABLE),

\item {} 
\sphinxAtStartPar
SQLite \sphinxhyphen{} transakcje działają w trybie plikowym; BEGIN, COMMIT i ROLLBACK są wspierane,

\item {} 
\sphinxAtStartPar
SQL Server \sphinxhyphen{} zaawansowany mechanizm transakcji z kontrolą poziomów izolacji, także eksplicytny SAVEPOINT.

\end{itemize}


\subsection{Literatura}
\label{\detokenize{rozdzial2/Kontrola_i_konserwacja/kontrola_i_konserwacja:literatura}}\begin{itemize}
\item {} 
\sphinxAtStartPar
\sphinxhref{https://www.postgresql.org/docs/current/index.html}{Oficjalna dokumentacja PostgreSQL}

\item {} 
\sphinxAtStartPar
Riggs S., Krosing H., PostgreSQL. Receptury dla administratora, Helion 2011

\item {} 
\sphinxAtStartPar
Matthew N., Stones R., Beginning Databases with PostgreSQL. From Novice to Professional, Apress 2006

\item {} 
\sphinxAtStartPar
Juba S., Vannahme A., Volkov A., Learning PostgreSQL, Packt Publishing 2015

\end{itemize}

\sphinxstepscope


\section{Partycjonowanie danych w PostgreSQL \textendash{} analiza, typy, zastosowania i dobre praktyki}
\label{\detokenize{rozdzial2/Partycjonowanie-danych/source/Partycjonowanie:partycjonowanie-danych-w-postgresql-analiza-typy-zastosowania-i-dobre-praktyki}}\label{\detokenize{rozdzial2/Partycjonowanie-danych/source/Partycjonowanie::doc}}\begin{quote}\begin{description}
\sphinxlineitem{Autor}
\sphinxAtStartPar
Bartosz Potoczny

\sphinxlineitem{Data}
\sphinxAtStartPar
2025\sphinxhyphen{}06\sphinxhyphen{}12

\end{description}\end{quote}


\subsection{Streszczenie}
\label{\detokenize{rozdzial2/Partycjonowanie-danych/source/Partycjonowanie:streszczenie}}
\sphinxAtStartPar
Celem niniejszego sprawozdania jest kompleksowa analiza zagadnienia partycjonowania danych w systemie zarządzania relacyjną bazą danych PostgreSQL. Praca omawia teoretyczne podstawy partycjonowania, szczegółowo wyjaśnia wszystkie dostępne mechanizmy oraz przedstawia metody realizacji partycjonowania w praktyce. Zaprezentowano również typowe scenariusze użycia, narzędzia monitorowania oraz najlepsze praktyki projektowe. Całość przeanalizowano pod kątem wydajności, utrzymania i bezpieczeństwa danych.


\subsection{1. Wprowadzenie}
\label{\detokenize{rozdzial2/Partycjonowanie-danych/source/Partycjonowanie:wprowadzenie}}
\sphinxAtStartPar
Współczesne systemy informatyczne generują i przetwarzają coraz większe ilości danych, co wymusza stosowanie zaawansowanych mechanizmów optymalizacji przechowywania i dostępu do informacji. Partycjonowanie danych jest jedną z kluczowych technik pozwalających na poprawę wydajności, skalowalności i zarządzalności baz danych. PostgreSQL, jako zaawansowany system zarządzania relacyjną bazą danych (RDBMS), oferuje rozbudowane wsparcie dla partycjonowania, umożliwiając dostosowanie architektury bazy do indywidualnych potrzeb.


\subsection{2. Definicja i cel partycjonowania}
\label{\detokenize{rozdzial2/Partycjonowanie-danych/source/Partycjonowanie:definicja-i-cel-partycjonowania}}
\sphinxAtStartPar
Partycjonowanie polega na logicznym podziale dużej tabeli na mniejsze, łatwiejsze w zarządzaniu fragmenty zwane partycjami. Mimo fizycznego rozdzielenia, partycje są prezentowane użytkownikowi jako jedna wspólna tabela nadrzędna (ang. partitioned table, master table). Celem partycjonowania jest:
\begin{itemize}
\item {} 
\sphinxAtStartPar
Zwiększenie wydajności operacji SELECT, INSERT, UPDATE, DELETE poprzez ograniczenie zakresu danych do przeszukania (partition pruning).

\item {} 
\sphinxAtStartPar
Ułatwienie zarządzania i archiwizacji danych (np. szybkie usuwanie lub przenoszenie całych partycji).

\item {} 
\sphinxAtStartPar
Lepsze rozłożenie obciążenia (możliwość przechowywania partycji na różnych dyskach/tablespaces).

\item {} 
\sphinxAtStartPar
Zmniejszenie ryzyka zablokowania całej tabeli podczas operacji konserwacyjnych (VACUUM, REINDEX itp.).

\end{itemize}


\subsection{3. Modele i typy partycjonowania w PostgreSQL}
\label{\detokenize{rozdzial2/Partycjonowanie-danych/source/Partycjonowanie:modele-i-typy-partycjonowania-w-postgresql}}
\sphinxAtStartPar
PostgreSQL obsługuje trzy podstawowe typy partycjonowania:

\sphinxAtStartPar
\#\#\# 3.1 Partycjonowanie zakresowe (RANGE)

\sphinxAtStartPar
Dane są przypisywane do partycji na podstawie wartości mieszczącej się w określonym zakresie (np. daty, numery, id). Każda partycja odpowiada innemu przedziałowi.

\sphinxAtStartPar
\sphinxstylestrong{Przykład:}

\begin{sphinxVerbatim}[commandchars=\\\{\}]
\PYG{k}{CREATE}\PYG{+w}{ }\PYG{k}{TABLE}\PYG{+w}{ }\PYG{n}{events}\PYG{+w}{ }\PYG{p}{(}
\PYG{+w}{    }\PYG{n}{event\PYGZus{}id}\PYG{+w}{ }\PYG{n+nb}{serial}\PYG{+w}{ }\PYG{k}{PRIMARY}\PYG{+w}{ }\PYG{k}{KEY}\PYG{p}{,}
\PYG{+w}{    }\PYG{n}{event\PYGZus{}date}\PYG{+w}{ }\PYG{n+nb}{date}\PYG{+w}{ }\PYG{k}{NOT}\PYG{+w}{ }\PYG{k}{NULL}\PYG{p}{,}
\PYG{+w}{    }\PYG{n}{description}\PYG{+w}{ }\PYG{n+nb}{text}
\PYG{p}{)}\PYG{+w}{ }\PYG{n}{PARTITION}\PYG{+w}{ }\PYG{k}{BY}\PYG{+w}{ }\PYG{n}{RANGE}\PYG{+w}{ }\PYG{p}{(}\PYG{n}{event\PYGZus{}date}\PYG{p}{)}\PYG{p}{;}

\PYG{k}{CREATE}\PYG{+w}{ }\PYG{k}{TABLE}\PYG{+w}{ }\PYG{n}{events\PYGZus{}2023}\PYG{+w}{ }\PYG{n}{PARTITION}\PYG{+w}{ }\PYG{k}{OF}\PYG{+w}{ }\PYG{n}{events}
\PYG{+w}{    }\PYG{k}{FOR}\PYG{+w}{ }\PYG{k}{VALUES}\PYG{+w}{ }\PYG{k}{FROM}\PYG{+w}{ }\PYG{p}{(}\PYG{l+s+s1}{\PYGZsq{}2023\PYGZhy{}01\PYGZhy{}01\PYGZsq{}}\PYG{p}{)}\PYG{+w}{ }\PYG{k}{TO}\PYG{+w}{ }\PYG{p}{(}\PYG{l+s+s1}{\PYGZsq{}2024\PYGZhy{}01\PYGZhy{}01\PYGZsq{}}\PYG{p}{)}\PYG{p}{;}

\PYG{k}{CREATE}\PYG{+w}{ }\PYG{k}{TABLE}\PYG{+w}{ }\PYG{n}{events\PYGZus{}2024}\PYG{+w}{ }\PYG{n}{PARTITION}\PYG{+w}{ }\PYG{k}{OF}\PYG{+w}{ }\PYG{n}{events}
\PYG{+w}{    }\PYG{k}{FOR}\PYG{+w}{ }\PYG{k}{VALUES}\PYG{+w}{ }\PYG{k}{FROM}\PYG{+w}{ }\PYG{p}{(}\PYG{l+s+s1}{\PYGZsq{}2024\PYGZhy{}01\PYGZhy{}01\PYGZsq{}}\PYG{p}{)}\PYG{+w}{ }\PYG{k}{TO}\PYG{+w}{ }\PYG{p}{(}\PYG{l+s+s1}{\PYGZsq{}2025\PYGZhy{}01\PYGZhy{}01\PYGZsq{}}\PYG{p}{)}\PYG{p}{;}
\end{sphinxVerbatim}

\sphinxAtStartPar
\sphinxstylestrong{Zastosowania:} logi systemowe, zamówienia, dane czasowe.

\sphinxAtStartPar
\#\#\# 3.2 Partycjonowanie listowe (LIST)

\sphinxAtStartPar
Dane są przypisywane do partycji na podstawie konkretnej wartości z listy (np. kraj, status, kategoria).

\sphinxAtStartPar
\sphinxstylestrong{Przykład:}

\begin{sphinxVerbatim}[commandchars=\\\{\}]
\PYG{k}{CREATE}\PYG{+w}{ }\PYG{k}{TABLE}\PYG{+w}{ }\PYG{n}{sales}\PYG{+w}{ }\PYG{p}{(}
\PYG{+w}{    }\PYG{n}{sale\PYGZus{}id}\PYG{+w}{ }\PYG{n+nb}{serial}\PYG{+w}{ }\PYG{k}{PRIMARY}\PYG{+w}{ }\PYG{k}{KEY}\PYG{p}{,}
\PYG{+w}{    }\PYG{n}{country}\PYG{+w}{ }\PYG{n+nb}{text}\PYG{p}{,}
\PYG{+w}{    }\PYG{n}{value}\PYG{+w}{ }\PYG{n+nb}{numeric}
\PYG{p}{)}\PYG{+w}{ }\PYG{n}{PARTITION}\PYG{+w}{ }\PYG{k}{BY}\PYG{+w}{ }\PYG{n}{LIST}\PYG{+w}{ }\PYG{p}{(}\PYG{n}{country}\PYG{p}{)}\PYG{p}{;}

\PYG{k}{CREATE}\PYG{+w}{ }\PYG{k}{TABLE}\PYG{+w}{ }\PYG{n}{sales\PYGZus{}pl}\PYG{+w}{ }\PYG{n}{PARTITION}\PYG{+w}{ }\PYG{k}{OF}\PYG{+w}{ }\PYG{n}{sales}\PYG{+w}{ }\PYG{k}{FOR}\PYG{+w}{ }\PYG{k}{VALUES}\PYG{+w}{ }\PYG{k}{IN}\PYG{+w}{ }\PYG{p}{(}\PYG{l+s+s1}{\PYGZsq{}Poland\PYGZsq{}}\PYG{p}{)}\PYG{p}{;}
\PYG{k}{CREATE}\PYG{+w}{ }\PYG{k}{TABLE}\PYG{+w}{ }\PYG{n}{sales\PYGZus{}de}\PYG{+w}{ }\PYG{n}{PARTITION}\PYG{+w}{ }\PYG{k}{OF}\PYG{+w}{ }\PYG{n}{sales}\PYG{+w}{ }\PYG{k}{FOR}\PYG{+w}{ }\PYG{k}{VALUES}\PYG{+w}{ }\PYG{k}{IN}\PYG{+w}{ }\PYG{p}{(}\PYG{l+s+s1}{\PYGZsq{}Germany\PYGZsq{}}\PYG{p}{)}\PYG{p}{;}
\PYG{k}{CREATE}\PYG{+w}{ }\PYG{k}{TABLE}\PYG{+w}{ }\PYG{n}{sales\PYGZus{}other}\PYG{+w}{ }\PYG{n}{PARTITION}\PYG{+w}{ }\PYG{k}{OF}\PYG{+w}{ }\PYG{n}{sales}\PYG{+w}{ }\PYG{k}{DEFAULT}\PYG{p}{;}
\end{sphinxVerbatim}

\sphinxAtStartPar
\sphinxstylestrong{Zastosowania:} dane geograficzne, statusowe, podział według typu klienta.

\sphinxAtStartPar
\#\#\# 3.3 Partycjonowanie haszowe (HASH)

\sphinxAtStartPar
Dane są rozdzielane pomiędzy partycje na podstawie funkcji haszującej zastosowanej do wybranej kolumny. Pozwala to równomiernie rozłożyć dane, gdy nie ma logicznego podziału zakresowego ani listowego.

\sphinxAtStartPar
\sphinxstylestrong{Przykład:}

\begin{sphinxVerbatim}[commandchars=\\\{\}]
\PYG{k}{CREATE}\PYG{+w}{ }\PYG{k}{TABLE}\PYG{+w}{ }\PYG{n}{logs}\PYG{+w}{ }\PYG{p}{(}
\PYG{+w}{    }\PYG{n}{log\PYGZus{}id}\PYG{+w}{ }\PYG{n+nb}{serial}\PYG{+w}{ }\PYG{k}{PRIMARY}\PYG{+w}{ }\PYG{k}{KEY}\PYG{p}{,}
\PYG{+w}{    }\PYG{n}{user\PYGZus{}id}\PYG{+w}{ }\PYG{n+nb}{int}\PYG{p}{,}
\PYG{+w}{    }\PYG{n}{log\PYGZus{}time}\PYG{+w}{ }\PYG{k}{timestamp}
\PYG{p}{)}\PYG{+w}{ }\PYG{n}{PARTITION}\PYG{+w}{ }\PYG{k}{BY}\PYG{+w}{ }\PYG{n}{HASH}\PYG{+w}{ }\PYG{p}{(}\PYG{n}{user\PYGZus{}id}\PYG{p}{)}\PYG{p}{;}

\PYG{k}{CREATE}\PYG{+w}{ }\PYG{k}{TABLE}\PYG{+w}{ }\PYG{n}{logs\PYGZus{}p0}\PYG{+w}{ }\PYG{n}{PARTITION}\PYG{+w}{ }\PYG{k}{OF}\PYG{+w}{ }\PYG{n}{logs}\PYG{+w}{ }\PYG{k}{FOR}\PYG{+w}{ }\PYG{k}{VALUES}\PYG{+w}{ }\PYG{k}{WITH}\PYG{+w}{ }\PYG{p}{(}\PYG{n}{MODULUS}\PYG{+w}{ }\PYG{l+m+mi}{4}\PYG{p}{,}\PYG{+w}{ }\PYG{n}{REMAINDER}\PYG{+w}{ }\PYG{l+m+mi}{0}\PYG{p}{)}\PYG{p}{;}
\PYG{k}{CREATE}\PYG{+w}{ }\PYG{k}{TABLE}\PYG{+w}{ }\PYG{n}{logs\PYGZus{}p1}\PYG{+w}{ }\PYG{n}{PARTITION}\PYG{+w}{ }\PYG{k}{OF}\PYG{+w}{ }\PYG{n}{logs}\PYG{+w}{ }\PYG{k}{FOR}\PYG{+w}{ }\PYG{k}{VALUES}\PYG{+w}{ }\PYG{k}{WITH}\PYG{+w}{ }\PYG{p}{(}\PYG{n}{MODULUS}\PYG{+w}{ }\PYG{l+m+mi}{4}\PYG{p}{,}\PYG{+w}{ }\PYG{n}{REMAINDER}\PYG{+w}{ }\PYG{l+m+mi}{1}\PYG{p}{)}\PYG{p}{;}
\PYG{k}{CREATE}\PYG{+w}{ }\PYG{k}{TABLE}\PYG{+w}{ }\PYG{n}{logs\PYGZus{}p2}\PYG{+w}{ }\PYG{n}{PARTITION}\PYG{+w}{ }\PYG{k}{OF}\PYG{+w}{ }\PYG{n}{logs}\PYG{+w}{ }\PYG{k}{FOR}\PYG{+w}{ }\PYG{k}{VALUES}\PYG{+w}{ }\PYG{k}{WITH}\PYG{+w}{ }\PYG{p}{(}\PYG{n}{MODULUS}\PYG{+w}{ }\PYG{l+m+mi}{4}\PYG{p}{,}\PYG{+w}{ }\PYG{n}{REMAINDER}\PYG{+w}{ }\PYG{l+m+mi}{2}\PYG{p}{)}\PYG{p}{;}
\PYG{k}{CREATE}\PYG{+w}{ }\PYG{k}{TABLE}\PYG{+w}{ }\PYG{n}{logs\PYGZus{}p3}\PYG{+w}{ }\PYG{n}{PARTITION}\PYG{+w}{ }\PYG{k}{OF}\PYG{+w}{ }\PYG{n}{logs}\PYG{+w}{ }\PYG{k}{FOR}\PYG{+w}{ }\PYG{k}{VALUES}\PYG{+w}{ }\PYG{k}{WITH}\PYG{+w}{ }\PYG{p}{(}\PYG{n}{MODULUS}\PYG{+w}{ }\PYG{l+m+mi}{4}\PYG{p}{,}\PYG{+w}{ }\PYG{n}{REMAINDER}\PYG{+w}{ }\PYG{l+m+mi}{3}\PYG{p}{)}\PYG{p}{;}
\end{sphinxVerbatim}

\sphinxAtStartPar
\sphinxstylestrong{Zastosowania:} przypadki wymagające równomiernego rozłożenia danych, np. duże systemy telemetryczne.

\sphinxAtStartPar
\#\#\# 3.4 Partycjonowanie wielopoziomowe (Composite/Hierarchical Partitioning)

\sphinxAtStartPar
PostgreSQL umożliwia tworzenie partycji podrzędnych, czyli partycjonowanie już partycjonowanych tabel (tzw. subpartitioning).

\sphinxAtStartPar
\sphinxstylestrong{Przykład:}

\begin{sphinxVerbatim}[commandchars=\\\{\}]
\PYG{k}{CREATE}\PYG{+w}{ }\PYG{k}{TABLE}\PYG{+w}{ }\PYG{n}{measurements}\PYG{+w}{ }\PYG{p}{(}
\PYG{+w}{    }\PYG{n}{id}\PYG{+w}{ }\PYG{n+nb}{serial}\PYG{+w}{ }\PYG{k}{PRIMARY}\PYG{+w}{ }\PYG{k}{KEY}\PYG{p}{,}
\PYG{+w}{    }\PYG{n}{region}\PYG{+w}{ }\PYG{n+nb}{text}\PYG{p}{,}
\PYG{+w}{    }\PYG{n}{measurement\PYGZus{}date}\PYG{+w}{ }\PYG{n+nb}{date}\PYG{p}{,}
\PYG{+w}{    }\PYG{n}{value}\PYG{+w}{ }\PYG{n+nb}{numeric}
\PYG{p}{)}\PYG{+w}{ }\PYG{n}{PARTITION}\PYG{+w}{ }\PYG{k}{BY}\PYG{+w}{ }\PYG{n}{LIST}\PYG{+w}{ }\PYG{p}{(}\PYG{n}{region}\PYG{p}{)}\PYG{p}{;}

\PYG{k}{CREATE}\PYG{+w}{ }\PYG{k}{TABLE}\PYG{+w}{ }\PYG{n}{measurements\PYGZus{}europe}\PYG{+w}{ }\PYG{n}{PARTITION}\PYG{+w}{ }\PYG{k}{OF}\PYG{+w}{ }\PYG{n}{measurements}
\PYG{+w}{    }\PYG{k}{FOR}\PYG{+w}{ }\PYG{k}{VALUES}\PYG{+w}{ }\PYG{k}{IN}\PYG{+w}{ }\PYG{p}{(}\PYG{l+s+s1}{\PYGZsq{}Europe\PYGZsq{}}\PYG{p}{)}\PYG{+w}{ }\PYG{n}{PARTITION}\PYG{+w}{ }\PYG{k}{BY}\PYG{+w}{ }\PYG{n}{RANGE}\PYG{+w}{ }\PYG{p}{(}\PYG{n}{measurement\PYGZus{}date}\PYG{p}{)}\PYG{p}{;}

\PYG{k}{CREATE}\PYG{+w}{ }\PYG{k}{TABLE}\PYG{+w}{ }\PYG{n}{measurements\PYGZus{}europe\PYGZus{}2024}\PYG{+w}{ }\PYG{n}{PARTITION}\PYG{+w}{ }\PYG{k}{OF}\PYG{+w}{ }\PYG{n}{measurements\PYGZus{}europe}
\PYG{+w}{    }\PYG{k}{FOR}\PYG{+w}{ }\PYG{k}{VALUES}\PYG{+w}{ }\PYG{k}{FROM}\PYG{+w}{ }\PYG{p}{(}\PYG{l+s+s1}{\PYGZsq{}2024\PYGZhy{}01\PYGZhy{}01\PYGZsq{}}\PYG{p}{)}\PYG{+w}{ }\PYG{k}{TO}\PYG{+w}{ }\PYG{p}{(}\PYG{l+s+s1}{\PYGZsq{}2025\PYGZhy{}01\PYGZhy{}01\PYGZsq{}}\PYG{p}{)}\PYG{p}{;}
\end{sphinxVerbatim}

\sphinxAtStartPar
\sphinxstylestrong{Zastosowania:} bardzo duże tabele, złożona struktura danych (np. po regionie i dacie).


\subsection{4. Implementacja partycjonowania w praktyce}
\label{\detokenize{rozdzial2/Partycjonowanie-danych/source/Partycjonowanie:implementacja-partycjonowania-w-praktyce}}
\sphinxAtStartPar
\#\#\# 4.1 Tworzenie i zarządzanie partycjami
\begin{itemize}
\item {} 
\sphinxAtStartPar
\sphinxstylestrong{Tworzenie partycji:} Partycje tworzone są jako osobne tabele, ale zarządzane przez tabelę nadrzędną.

\item {} 
\sphinxAtStartPar
\sphinxstylestrong{Dodawanie partycji:} Możliwe w dowolnym momencie przy użyciu CREATE TABLE … PARTITION OF.

\item {} 
\sphinxAtStartPar
\sphinxstylestrong{Usuwanie partycji:} ALTER TABLE … DETACH PARTITION + DROP TABLE (po odłączeniu partycji).

\item {} 
\sphinxAtStartPar
\sphinxstylestrong{Domyślna partycja:} Można zdefiniować partycję przechowującą dane niepasujące do żadnej innej (DEFAULT).

\end{itemize}

\sphinxAtStartPar
\#\#\# 4.2 Wstawianie i odczyt danych
\begin{itemize}
\item {} 
\sphinxAtStartPar
Dane są automatycznie kierowane do właściwej partycji na podstawie klucza partycjonowania.

\item {} 
\sphinxAtStartPar
W przypadku braku pasującej partycji (i braku DEFAULT) \textendash{} błąd constraint violation.

\item {} 
\sphinxAtStartPar
Zapytania ograniczone do klucza partycjonowania korzystają z partition pruning \textendash{} przeszukują tylko wybrane partycje.

\end{itemize}

\sphinxAtStartPar
\#\#\# 4.3 Indeksowanie partycji
\begin{itemize}
\item {} 
\sphinxAtStartPar
Możliwe jest tworzenie indeksów na każdej partycji osobno lub dziedziczenie indeksów z tabeli nadrzędnej (od PostgreSQL 11 wzwyż).

\item {} 
\sphinxAtStartPar
Indeksy globalne (na całą tabelę partycjonowaną) nie są jeszcze dostępne (stan na 2025).

\end{itemize}

\sphinxAtStartPar
\#\#\# 4.4 Ograniczenia partycjonowania
\begin{itemize}
\item {} 
\sphinxAtStartPar
Klucz partycjonowania musi być częścią klucza głównego (PRIMARY KEY).

\item {} 
\sphinxAtStartPar
Niektóre operacje mogą wymagać wykonywania osobno na każdej partycji (np. VACUUM, REINDEX).

\item {} 
\sphinxAtStartPar
Wersje PostgreSQL \textless{}10 obsługują partycjonowanie tylko przez dziedziczenie \textendash{} obecnie uznawane za przestarzałe.

\end{itemize}


\subsection{5. Monitorowanie i administracja}
\label{\detokenize{rozdzial2/Partycjonowanie-danych/source/Partycjonowanie:monitorowanie-i-administracja}}
\sphinxAtStartPar
\#\#\# 5.1 Sprawdzanie rozmieszczenia danych

\begin{sphinxVerbatim}[commandchars=\\\{\}]
\PYG{k}{SELECT}\PYG{+w}{ }\PYG{n}{tableoid}\PYG{p}{:}\PYG{p}{:}\PYG{n}{regclass}\PYG{+w}{ }\PYG{k}{AS}\PYG{+w}{ }\PYG{n}{partition}\PYG{p}{,}\PYG{+w}{ }\PYG{o}{*}\PYG{+w}{ }\PYG{k}{FROM}\PYG{+w}{ }\PYG{n}{measurements}\PYG{p}{;}
\end{sphinxVerbatim}

\sphinxAtStartPar
\#\#\# 5.2 Lista partycji

\begin{sphinxVerbatim}[commandchars=\\\{\}]
\PYG{k}{SELECT}\PYG{+w}{ }\PYG{n}{inhrelid}\PYG{p}{:}\PYG{p}{:}\PYG{n}{regclass}\PYG{+w}{ }\PYG{k}{AS}\PYG{+w}{ }\PYG{n}{partition}
\PYG{k}{FROM}\PYG{+w}{ }\PYG{n}{pg\PYGZus{}inherits}
\PYG{k}{WHERE}\PYG{+w}{ }\PYG{n}{inhparent}\PYG{+w}{ }\PYG{o}{=}\PYG{+w}{ }\PYG{l+s+s1}{\PYGZsq{}measurements\PYGZsq{}}\PYG{p}{:}\PYG{p}{:}\PYG{n}{regclass}\PYG{p}{;}
\end{sphinxVerbatim}

\sphinxAtStartPar
\#\#\# 5.3 Rozmiar partycji

\begin{sphinxVerbatim}[commandchars=\\\{\}]
\PYG{k}{SELECT}\PYG{+w}{ }\PYG{n}{relname}\PYG{+w}{ }\PYG{k}{AS}\PYG{+w}{ }\PYG{l+s+ss}{\PYGZdq{}Partition\PYGZdq{}}\PYG{p}{,}\PYG{+w}{ }\PYG{n}{pg\PYGZus{}size\PYGZus{}pretty}\PYG{p}{(}\PYG{n}{pg\PYGZus{}total\PYGZus{}relation\PYGZus{}size}\PYG{p}{(}\PYG{n}{relid}\PYG{p}{)}\PYG{p}{)}\PYG{+w}{ }\PYG{k}{AS}\PYG{+w}{ }\PYG{l+s+ss}{\PYGZdq{}Size\PYGZdq{}}
\PYG{k}{FROM}\PYG{+w}{ }\PYG{n}{pg\PYGZus{}catalog}\PYG{p}{.}\PYG{n}{pg\PYGZus{}statio\PYGZus{}user\PYGZus{}tables}
\PYG{k}{WHERE}\PYG{+w}{ }\PYG{n}{relname}\PYG{+w}{ }\PYG{k}{LIKE}\PYG{+w}{ }\PYG{l+s+s1}{\PYGZsq{}measurements\PYGZpc{}\PYGZsq{}}
\PYG{k}{ORDER}\PYG{+w}{ }\PYG{k}{BY}\PYG{+w}{ }\PYG{n}{pg\PYGZus{}total\PYGZus{}relation\PYGZus{}size}\PYG{p}{(}\PYG{n}{relid}\PYG{p}{)}\PYG{+w}{ }\PYG{k}{DESC}\PYG{p}{;}
\end{sphinxVerbatim}

\sphinxAtStartPar
\#\#\# 5.4 Analiza planu zapytania (partition pruning)

\begin{sphinxVerbatim}[commandchars=\\\{\}]
\PYG{k}{EXPLAIN}\PYG{+w}{ }\PYG{k}{ANALYZE}
\PYG{k}{SELECT}\PYG{+w}{ }\PYG{o}{*}\PYG{+w}{ }\PYG{k}{FROM}\PYG{+w}{ }\PYG{n}{measurements}\PYG{+w}{ }\PYG{k}{WHERE}\PYG{+w}{ }\PYG{n}{region}\PYG{+w}{ }\PYG{o}{=}\PYG{+w}{ }\PYG{l+s+s1}{\PYGZsq{}Europe\PYGZsq{}}\PYG{+w}{ }\PYG{k}{AND}\PYG{+w}{ }\PYG{n}{measurement\PYGZus{}date}\PYG{+w}{ }\PYG{o}{\PYGZgt{}}\PYG{o}{=}\PYG{+w}{ }\PYG{l+s+s1}{\PYGZsq{}2024\PYGZhy{}01\PYGZhy{}01\PYGZsq{}}\PYG{p}{;}

\PYG{c+c1}{\PYGZhy{}\PYGZhy{} W planie widać użycie tylko właściwych partycji.}
\end{sphinxVerbatim}


\subsection{6. Typowe scenariusze zastosowań}
\label{\detokenize{rozdzial2/Partycjonowanie-danych/source/Partycjonowanie:typowe-scenariusze-zastosowan}}\begin{itemize}
\item {} 
\sphinxAtStartPar
\sphinxstylestrong{Przetwarzanie danych czasowych:} partycjonowanie zakresowe po dacie (logi, zamówienia, pomiary).

\item {} 
\sphinxAtStartPar
\sphinxstylestrong{Dane geograficzne lub kategoryczne:} partycjonowanie listowe (kraj, region, kategoria produktu).

\item {} 
\sphinxAtStartPar
\sphinxstylestrong{Systemy telemetryczne i IoT:} partycjonowanie haszowe lub wielopoziomowe (np. urządzenie + czas).

\item {} 
\sphinxAtStartPar
\sphinxstylestrong{Duże systemy ERP/CRM:} partycjonowanie po kliencie, regionie, a następnie po dacie.

\end{itemize}


\subsection{7. Dobre praktyki projektowania partycji}
\label{\detokenize{rozdzial2/Partycjonowanie-danych/source/Partycjonowanie:dobre-praktyki-projektowania-partycji}}\begin{itemize}
\item {} 
\sphinxAtStartPar
\sphinxstylestrong{Dobór klucza partycjonowania:} Powinien odpowiadać najczęściej używanym warunkom w zapytaniach WHERE.

\item {} 
\sphinxAtStartPar
\sphinxstylestrong{Optymalna liczba partycji:} Zbyt mała liczba partycji nie daje efektu, zbyt duża zwiększa narzut administracyjny.

\item {} 
\sphinxAtStartPar
\sphinxstylestrong{Automatyzacja tworzenia partycji:} Skrypty lub narzędzia generujące nowe partycje np. na kolejne miesiące/lata.

\item {} 
\sphinxAtStartPar
\sphinxstylestrong{Monitorowanie wydajności:} Regularne sprawdzanie rozmiarów partycji, statystyk oraz planów wykonania zapytań.

\item {} 
\sphinxAtStartPar
\sphinxstylestrong{Bezpieczeństwo danych:} Możliwość szybkiego backupu lub usunięcia starych partycji.

\end{itemize}


\subsection{8. Ograniczenia i potencjalne problemy}
\label{\detokenize{rozdzial2/Partycjonowanie-danych/source/Partycjonowanie:ograniczenia-i-potencjalne-problemy}}\begin{itemize}
\item {} 
\sphinxAtStartPar
Brak natywnych indeksów globalnych (stan na 2025) utrudnia niektóre zapytania przekrojowe.

\item {} 
\sphinxAtStartPar
Operacje DDL na tabeli nadrzędnej mogą być kosztowne przy dużej liczbie partycji.

\item {} 
\sphinxAtStartPar
Niektóre narzędzia zewnętrzne mogą nie obsługiwać partycji w pełni transparentnie.

\item {} 
\sphinxAtStartPar
Przenoszenie danych między partycjami wymaga operacji INSERT + DELETE lub narzędzi specjalistycznych.

\end{itemize}


\subsection{9. Podsumowanie i wnioski}
\label{\detokenize{rozdzial2/Partycjonowanie-danych/source/Partycjonowanie:podsumowanie-i-wnioski}}
\sphinxAtStartPar
Partycjonowanie danych w PostgreSQL jest zaawansowanym i elastycznym narzędziem, pozwalającym na istotną poprawę wydajności oraz ułatwiającym zarządzanie dużymi zbiorami danych. Właściwy dobór typu partycjonowania, klucza oraz liczby i organizacji partycji wymaga analizy charakterystyki danych i typowych zapytań. Zaleca się regularne monitorowanie i dostosowywanie architektury partycjonowania, zwłaszcza w przypadku dynamicznie rosnących zbiorów danych.


\subsection{10. Krótkie porównanie partycjonowania w PostgreSQL i innych systemach bazodanowych}
\label{\detokenize{rozdzial2/Partycjonowanie-danych/source/Partycjonowanie:krotkie-porownanie-partycjonowania-w-postgresql-i-innych-systemach-bazodanowych}}
\sphinxAtStartPar
Partycjonowanie danych jest wspierane przez większość nowoczesnych systemów baz danych, jednak szczegóły implementacji i dostępne możliwości mogą się różnić:
\begin{itemize}
\item {} 
\sphinxAtStartPar
\sphinxstylestrong{PostgreSQL:}
Umożliwia partycjonowanie zakresowe, listowe, haszowe oraz wielopoziomowe (od wersji 10). Partycje są w pełni zintegrowane z silnikiem (od wersji 10), a operacje na partycjonowanych tabelach są transparentne dla użytkownika. Nie obsługuje jeszcze natywnych indeksów globalnych (stan na 2025).

\item {} 
\sphinxAtStartPar
\sphinxstylestrong{Oracle Database:}
Bardzo rozbudowane opcje partycjonowania (RANGE, LIST, HASH, COMPOSITE), obsługuje indeksy lokalne i globalne, automatyczne zarządzanie partycjami, także partycjonowanie na poziomie fizycznym (np. partycjonowanie indeksów, tabel LOB). Mechanizmy zaawansowane, ale często dostępne tylko w płatnych edycjach.

\item {} 
\sphinxAtStartPar
\sphinxstylestrong{MySQL (InnoDB):}
Wspiera partycjonowanie RANGE, LIST, HASH, KEY. Możliwości są jednak bardziej ograniczone niż w PostgreSQL czy Oracle. Nie wszystkie operacje i typy indeksów są wspierane na partycjonowanych tabelach.

\item {} 
\sphinxAtStartPar
\sphinxstylestrong{Microsoft SQL Server:}
Umożliwia partycjonowanie tabel i indeksów przy użyciu tzw. partition schemes i partition functions. Pozwala na łatwe przenoszenie partycji oraz obsługuje indeksy globalne, co ułatwia optymalizację zapytań przekrojowych.

\end{itemize}

\sphinxAtStartPar
\sphinxstylestrong{Podsumowanie:}
PostgreSQL oferuje bardzo elastyczne i wydajne partycjonowanie, jednak niektóre zaawansowane funkcje (np. partycjonowanie indeksów globalnych) są jeszcze w fazie rozwoju, podczas gdy w Oracle czy SQL Server są już dojrzałymi rozwiązaniami.


\subsection{11. Przykład migracji niepartyconowanej tabeli na partycjonowaną}
\label{\detokenize{rozdzial2/Partycjonowanie-danych/source/Partycjonowanie:przyklad-migracji-niepartyconowanej-tabeli-na-partycjonowana}}
\sphinxAtStartPar
Migracja istniejącej tabeli na partycjonowaną w PostgreSQL wymaga kilku kroków. Oto przykładowy proces dla tabeli \sphinxcode{\sphinxupquote{orders}}:

\sphinxAtStartPar
\sphinxstylestrong{Załóżmy, że mamy tabelę:}

\begin{sphinxVerbatim}[commandchars=\\\{\}]
\PYG{k}{CREATE}\PYG{+w}{ }\PYG{k}{TABLE}\PYG{+w}{ }\PYG{n}{orders}\PYG{+w}{ }\PYG{p}{(}
\PYG{+w}{    }\PYG{n}{id}\PYG{+w}{ }\PYG{n+nb}{serial}\PYG{+w}{ }\PYG{k}{PRIMARY}\PYG{+w}{ }\PYG{k}{KEY}\PYG{p}{,}
\PYG{+w}{    }\PYG{n}{order\PYGZus{}date}\PYG{+w}{ }\PYG{n+nb}{date}\PYG{+w}{ }\PYG{k}{NOT}\PYG{+w}{ }\PYG{k}{NULL}\PYG{p}{,}
\PYG{+w}{    }\PYG{n}{customer\PYGZus{}id}\PYG{+w}{ }\PYG{n+nb}{int}\PYG{p}{,}
\PYG{+w}{    }\PYG{n}{amount}\PYG{+w}{ }\PYG{n+nb}{numeric}
\PYG{p}{)}\PYG{p}{;}
\end{sphinxVerbatim}

\sphinxAtStartPar
\sphinxstylestrong{Chcemy ją partycjonować po kolumnie \textasciigrave{}\textasciigrave{}order\_date\textasciigrave{}\textasciigrave{} (zakresy roczne):}
\begin{enumerate}
\sphinxsetlistlabels{\arabic}{enumi}{enumii}{}{.}%
\item {} 
\sphinxAtStartPar
Zmień nazwę oryginalnej tabeli:

\begin{sphinxVerbatim}[commandchars=\\\{\}]
\PYG{k}{ALTER}\PYG{+w}{ }\PYG{k}{TABLE}\PYG{+w}{ }\PYG{n}{orders}\PYG{+w}{ }\PYG{k}{RENAME}\PYG{+w}{ }\PYG{k}{TO}\PYG{+w}{ }\PYG{n}{orders\PYGZus{}old}\PYG{p}{;}
\end{sphinxVerbatim}

\item {} 
\sphinxAtStartPar
Utwórz nową tabelę partycjonowaną:

\begin{sphinxVerbatim}[commandchars=\\\{\}]
\PYG{k}{CREATE}\PYG{+w}{ }\PYG{k}{TABLE}\PYG{+w}{ }\PYG{n}{orders}\PYG{+w}{ }\PYG{p}{(}
\PYG{+w}{    }\PYG{n}{id}\PYG{+w}{ }\PYG{n+nb}{serial}\PYG{+w}{ }\PYG{k}{PRIMARY}\PYG{+w}{ }\PYG{k}{KEY}\PYG{p}{,}
\PYG{+w}{    }\PYG{n}{order\PYGZus{}date}\PYG{+w}{ }\PYG{n+nb}{date}\PYG{+w}{ }\PYG{k}{NOT}\PYG{+w}{ }\PYG{k}{NULL}\PYG{p}{,}
\PYG{+w}{    }\PYG{n}{customer\PYGZus{}id}\PYG{+w}{ }\PYG{n+nb}{int}\PYG{p}{,}
\PYG{+w}{    }\PYG{n}{amount}\PYG{+w}{ }\PYG{n+nb}{numeric}
\PYG{p}{)}\PYG{+w}{ }\PYG{n}{PARTITION}\PYG{+w}{ }\PYG{k}{BY}\PYG{+w}{ }\PYG{n}{RANGE}\PYG{+w}{ }\PYG{p}{(}\PYG{n}{order\PYGZus{}date}\PYG{p}{)}\PYG{p}{;}

\PYG{k}{CREATE}\PYG{+w}{ }\PYG{k}{TABLE}\PYG{+w}{ }\PYG{n}{orders\PYGZus{}2023}\PYG{+w}{ }\PYG{n}{PARTITION}\PYG{+w}{ }\PYG{k}{OF}\PYG{+w}{ }\PYG{n}{orders}
\PYG{+w}{    }\PYG{k}{FOR}\PYG{+w}{ }\PYG{k}{VALUES}\PYG{+w}{ }\PYG{k}{FROM}\PYG{+w}{ }\PYG{p}{(}\PYG{l+s+s1}{\PYGZsq{}2023\PYGZhy{}01\PYGZhy{}01\PYGZsq{}}\PYG{p}{)}\PYG{+w}{ }\PYG{k}{TO}\PYG{+w}{ }\PYG{p}{(}\PYG{l+s+s1}{\PYGZsq{}2024\PYGZhy{}01\PYGZhy{}01\PYGZsq{}}\PYG{p}{)}\PYG{p}{;}

\PYG{k}{CREATE}\PYG{+w}{ }\PYG{k}{TABLE}\PYG{+w}{ }\PYG{n}{orders\PYGZus{}2024}\PYG{+w}{ }\PYG{n}{PARTITION}\PYG{+w}{ }\PYG{k}{OF}\PYG{+w}{ }\PYG{n}{orders}
\PYG{+w}{    }\PYG{k}{FOR}\PYG{+w}{ }\PYG{k}{VALUES}\PYG{+w}{ }\PYG{k}{FROM}\PYG{+w}{ }\PYG{p}{(}\PYG{l+s+s1}{\PYGZsq{}2024\PYGZhy{}01\PYGZhy{}01\PYGZsq{}}\PYG{p}{)}\PYG{+w}{ }\PYG{k}{TO}\PYG{+w}{ }\PYG{p}{(}\PYG{l+s+s1}{\PYGZsq{}2025\PYGZhy{}01\PYGZhy{}01\PYGZsq{}}\PYG{p}{)}\PYG{p}{;}
\end{sphinxVerbatim}

\item {} 
\sphinxAtStartPar
Skopiuj dane do partycji:

\begin{sphinxVerbatim}[commandchars=\\\{\}]
\PYG{k}{INSERT}\PYG{+w}{ }\PYG{k}{INTO}\PYG{+w}{ }\PYG{n}{orders}\PYG{+w}{ }\PYG{p}{(}\PYG{n}{id}\PYG{p}{,}\PYG{+w}{ }\PYG{n}{order\PYGZus{}date}\PYG{p}{,}\PYG{+w}{ }\PYG{n}{customer\PYGZus{}id}\PYG{p}{,}\PYG{+w}{ }\PYG{n}{amount}\PYG{p}{)}
\PYG{k}{SELECT}\PYG{+w}{ }\PYG{n}{id}\PYG{p}{,}\PYG{+w}{ }\PYG{n}{order\PYGZus{}date}\PYG{p}{,}\PYG{+w}{ }\PYG{n}{customer\PYGZus{}id}\PYG{p}{,}\PYG{+w}{ }\PYG{n}{amount}\PYG{+w}{ }\PYG{k}{FROM}\PYG{+w}{ }\PYG{n}{orders\PYGZus{}old}\PYG{p}{;}
\end{sphinxVerbatim}

\item {} 
\sphinxAtStartPar
Sprawdź, czy dane zostały poprawnie rozdzielone:

\begin{sphinxVerbatim}[commandchars=\\\{\}]
\PYG{k}{SELECT}\PYG{+w}{ }\PYG{n}{tableoid}\PYG{p}{:}\PYG{p}{:}\PYG{n}{regclass}\PYG{p}{,}\PYG{+w}{ }\PYG{k}{COUNT}\PYG{p}{(}\PYG{o}{*}\PYG{p}{)}\PYG{+w}{ }\PYG{k}{FROM}\PYG{+w}{ }\PYG{n}{orders}\PYG{+w}{ }\PYG{k}{GROUP}\PYG{+w}{ }\PYG{k}{BY}\PYG{+w}{ }\PYG{n}{tableoid}\PYG{p}{;}
\end{sphinxVerbatim}

\item {} 
\sphinxAtStartPar
Usuń starą tabelę po upewnieniu się, że wszystko działa:

\begin{sphinxVerbatim}[commandchars=\\\{\}]
\PYG{k}{DROP}\PYG{+w}{ }\PYG{k}{TABLE}\PYG{+w}{ }\PYG{n}{orders\PYGZus{}old}\PYG{p}{;}
\end{sphinxVerbatim}

\end{enumerate}

\sphinxAtStartPar
Można też użyć narzędzi automatyzujących migracje (np. pg\_partman), jeśli tabel jest bardzo dużo lub są bardzo duże.


\subsection{12. Bibliografia}
\label{\detokenize{rozdzial2/Partycjonowanie-danych/source/Partycjonowanie:bibliografia}}\begin{enumerate}
\sphinxsetlistlabels{\arabic}{enumi}{enumii}{}{.}%
\item {} 
\sphinxAtStartPar
Dokumentacja PostgreSQL: \sphinxurl{https://www.postgresql.org/docs/current/ddl-partitioning.html}

\item {} 
\sphinxAtStartPar
„PostgreSQL. Zaawansowane techniki programistyczne”, Grzegorz Wójtowicz, Helion 2021

\item {} 
\sphinxAtStartPar
\sphinxurl{https://wiki.postgresql.org/wiki/Partitioning}

\item {} 
\sphinxAtStartPar
Oficjalny blog PostgreSQL: \sphinxurl{https://www.postgresql.org/about/news/}

\end{enumerate}

\sphinxstepscope


\chapter{3. Projekt bazy danych „Karnety na siłowni”}
\label{\detokenize{rozdzial3/rozdzial3:projekt-bazy-danych-karnety-na-silowni}}\label{\detokenize{rozdzial3/rozdzial3::doc}}
\sphinxAtStartPar
W tym rozdziale szczegółowo przedstawiono proces projektowania i implementacji bazy danych, od modelu koncepcyjnego po fizyczną realizację.


\section{Opis procesów biznesowych}
\label{\detokenize{rozdzial3/rozdzial3:opis-procesow-biznesowych}}
\sphinxAtStartPar
System bazodanowy został zaprojektowany w celu wsparcia trzech fundamentalnych procesów biznesowych siłowni:
\begin{enumerate}
\sphinxsetlistlabels{\arabic}{enumi}{enumii}{}{.}%
\item {} 
\sphinxAtStartPar
\sphinxstylestrong{Rejestracja nowego klienta:} Proces polega na zebraniu podstawowych danych klienta (imię, nazwisko, dane kontaktowe) i zapisaniu ich w systemie. Każdy klient otrzymuje unikalny identyfikator. Proces ten jest realizowany poprzez operację \sphinxtitleref{INSERT} na tabeli \sphinxtitleref{Klienci}.

\item {} 
\sphinxAtStartPar
\sphinxstylestrong{Sprzedaż i aktywacja karnetu:} Klient może zakupić jeden z dostępnych karnetów. System rejestruje typ karnetu, datę zakupu, oblicza datę ważności i zapisuje cenę transakcji. Karnet jest jednoznacznie powiązany z klientem, który go zakupił. Proces ten obsługuje operacja \sphinxtitleref{INSERT} na tabeli \sphinxtitleref{Karnety}, z kluczem obcym wskazującym na klienta.

\item {} 
\sphinxAtStartPar
\sphinxstylestrong{Rejestracja wejścia na siłownię:} Przy każdej wizycie klienta, system weryfikuje, czy posiada on aktywny (ważny) karnet. Weryfikacja polega na wyszukaniu w tabeli \sphinxtitleref{Karnety} rekordu powiązanego z danym klientem, którego \sphinxtitleref{data\_waznosci} jest późniejsza lub równa bieżącej dacie. Jeśli weryfikacja przebiegnie pomyślnie, system rejestruje wejście, zapisując identyfikator klienta i dokładny czas w tabeli \sphinxtitleref{Wejscia}.

\end{enumerate}


\section{Model Koncepcyjny (ERD)}
\label{\detokenize{rozdzial3/rozdzial3:model-koncepcyjny-erd}}
\noindent\sphinxincludegraphics{{erd_diagram}.png}


\section{Model Logiczny}
\label{\detokenize{rozdzial3/rozdzial3:model-logiczny}}
\sphinxAtStartPar
Model logiczny przekłada koncepcje na konkretną strukturę tabel, kolumn, typów danych i więzów integralności.
\begin{itemize}
\item {} \begin{description}
\sphinxlineitem{\sphinxstylestrong{Tabela: Klienci}}\begin{itemize}
\item {} 
\sphinxAtStartPar
\sphinxtitleref{klient\_id} (SERIAL, PK): Unikalny, automatycznie inkrementowany identyfikator klienta. Klucz główny.

\item {} 
\sphinxAtStartPar
\sphinxtitleref{imie} (VARCHAR(50), NOT NULL): Imię klienta.

\item {} 
\sphinxAtStartPar
\sphinxtitleref{nazwisko} (VARCHAR(50), NOT NULL): Nazwisko klienta.

\item {} 
\sphinxAtStartPar
\sphinxtitleref{email} (VARCHAR(100), UNIQUE, NOT NULL): Adres e\sphinxhyphen{}mail, musi być unikalny, służy jako login lub do komunikacji.

\item {} 
\sphinxAtStartPar
\sphinxtitleref{nr\_telefonu} (VARCHAR(15)): Numer telefonu, opcjonalny.

\end{itemize}

\end{description}

\item {} \begin{description}
\sphinxlineitem{\sphinxstylestrong{Tabela: Karnety}}\begin{itemize}
\item {} 
\sphinxAtStartPar
\sphinxtitleref{karnet\_id} (SERIAL, PK): Unikalny identyfikator transakcji zakupu karnetu.

\item {} 
\sphinxAtStartPar
\sphinxtitleref{klient\_id} (INTEGER, FK, NOT NULL): Klucz obcy wskazujący na klienta, który zakupił karnet.

\item {} 
\sphinxAtStartPar
\sphinxtitleref{typ\_karnetu} (VARCHAR(20), NOT NULL): Typ karnetu (np. «miesieczny»). Ograniczony więzem CHECK.

\item {} 
\sphinxAtStartPar
\sphinxtitleref{data\_zakupu} (DATE, NOT NULL): Data, w której karnet został sprzedany.

\item {} 
\sphinxAtStartPar
\sphinxtitleref{data\_waznosci} (DATE, NOT NULL): Data, do której karnet jest ważny.

\item {} 
\sphinxAtStartPar
\sphinxtitleref{cena} (NUMERIC(10, 2), NOT NULL): Cena zapłacona za karnet. Użycie typu \sphinxtitleref{NUMERIC} zapobiega błędom zaokrągleń typowym dla typów zmiennoprzecinkowych.

\end{itemize}

\end{description}

\item {} \begin{description}
\sphinxlineitem{\sphinxstylestrong{Tabela: Wejscia}}\begin{itemize}
\item {} 
\sphinxAtStartPar
\sphinxtitleref{wejscie\_id} (SERIAL, PK): Unikalny identyfikator zdarzenia wejścia.

\item {} 
\sphinxAtStartPar
\sphinxtitleref{klient\_id} (INTEGER, FK, NOT NULL): Klucz obcy wskazujący na wchodzącego klienta.

\item {} 
\sphinxAtStartPar
\sphinxtitleref{data\_wejscia} (TIMESTAMP, NOT NULL): Dokładna data i godzina wejścia. Wartość domyślna to \sphinxtitleref{CURRENT\_TIMESTAMP}.

\end{itemize}

\end{description}

\end{itemize}


\section{Model Fizyczny (Kod SQL)}
\label{\detokenize{rozdzial3/rozdzial3:model-fizyczny-kod-sql}}
\sphinxAtStartPar
Poniższy kod DDL (Data Definition Language) dla PostgreSQL tworzy opisaną strukturę bazy danych.
\sphinxSetupCaptionForVerbatim{Skrypt tworzący strukturę bazy danych}
\def\sphinxLiteralBlockLabel{\label{\detokenize{rozdzial3/rozdzial3:create-tables}}}
\begin{sphinxVerbatim}[commandchars=\\\{\}]
\PYG{c+c1}{\PYGZhy{}\PYGZhy{} Tabela przechowująca dane klientów siłowni.}
\PYG{c+c1}{\PYGZhy{}\PYGZhy{} Każdy klient jest unikalnie identyfikowany przez email.}
\PYG{k}{CREATE}\PYG{+w}{ }\PYG{k}{TABLE}\PYG{+w}{ }\PYG{n}{Klienci}\PYG{+w}{ }\PYG{p}{(}
\PYG{+w}{    }\PYG{n}{klient\PYGZus{}id}\PYG{+w}{ }\PYG{n+nb}{SERIAL}\PYG{+w}{ }\PYG{k}{PRIMARY}\PYG{+w}{ }\PYG{k}{KEY}\PYG{p}{,}
\PYG{+w}{    }\PYG{n}{imie}\PYG{+w}{ }\PYG{n+nb}{VARCHAR}\PYG{p}{(}\PYG{l+m+mi}{50}\PYG{p}{)}\PYG{+w}{ }\PYG{k}{NOT}\PYG{+w}{ }\PYG{k}{NULL}\PYG{p}{,}
\PYG{+w}{    }\PYG{n}{nazwisko}\PYG{+w}{ }\PYG{n+nb}{VARCHAR}\PYG{p}{(}\PYG{l+m+mi}{50}\PYG{p}{)}\PYG{+w}{ }\PYG{k}{NOT}\PYG{+w}{ }\PYG{k}{NULL}\PYG{p}{,}
\PYG{+w}{    }\PYG{n}{email}\PYG{+w}{ }\PYG{n+nb}{VARCHAR}\PYG{p}{(}\PYG{l+m+mi}{100}\PYG{p}{)}\PYG{+w}{ }\PYG{k}{UNIQUE}\PYG{+w}{ }\PYG{k}{NOT}\PYG{+w}{ }\PYG{k}{NULL}\PYG{p}{,}
\PYG{+w}{    }\PYG{n}{nr\PYGZus{}telefonu}\PYG{+w}{ }\PYG{n+nb}{VARCHAR}\PYG{p}{(}\PYG{l+m+mi}{15}\PYG{p}{)}
\PYG{p}{)}\PYG{p}{;}

\PYG{c+c1}{\PYGZhy{}\PYGZhy{} Tabela przechowująca informacje o zakupionych karnetach.}
\PYG{c+c1}{\PYGZhy{}\PYGZhy{} Więzy integralności (FOREIGN KEY z ON DELETE CASCADE) zapewniają,}
\PYG{c+c1}{\PYGZhy{}\PYGZhy{} że usunięcie klienta spowoduje usunięcie jego karnetów.}
\PYG{k}{CREATE}\PYG{+w}{ }\PYG{k}{TABLE}\PYG{+w}{ }\PYG{n}{Karnety}\PYG{+w}{ }\PYG{p}{(}
\PYG{+w}{    }\PYG{n}{karnet\PYGZus{}id}\PYG{+w}{ }\PYG{n+nb}{SERIAL}\PYG{+w}{ }\PYG{k}{PRIMARY}\PYG{+w}{ }\PYG{k}{KEY}\PYG{p}{,}
\PYG{+w}{    }\PYG{n}{klient\PYGZus{}id}\PYG{+w}{ }\PYG{n+nb}{INTEGER}\PYG{+w}{ }\PYG{k}{NOT}\PYG{+w}{ }\PYG{k}{NULL}\PYG{p}{,}
\PYG{+w}{    }\PYG{n}{typ\PYGZus{}karnetu}\PYG{+w}{ }\PYG{n+nb}{VARCHAR}\PYG{p}{(}\PYG{l+m+mi}{20}\PYG{p}{)}\PYG{+w}{ }\PYG{k}{NOT}\PYG{+w}{ }\PYG{k}{NULL}\PYG{p}{,}
\PYG{+w}{    }\PYG{n}{data\PYGZus{}zakupu}\PYG{+w}{ }\PYG{n+nb}{DATE}\PYG{+w}{ }\PYG{k}{NOT}\PYG{+w}{ }\PYG{k}{NULL}\PYG{p}{,}
\PYG{+w}{    }\PYG{n}{data\PYGZus{}waznosci}\PYG{+w}{ }\PYG{n+nb}{DATE}\PYG{+w}{ }\PYG{k}{NOT}\PYG{+w}{ }\PYG{k}{NULL}\PYG{p}{,}
\PYG{+w}{    }\PYG{n}{cena}\PYG{+w}{ }\PYG{n+nb}{NUMERIC}\PYG{p}{(}\PYG{l+m+mi}{10}\PYG{p}{,}\PYG{+w}{ }\PYG{l+m+mi}{2}\PYG{p}{)}\PYG{+w}{ }\PYG{k}{NOT}\PYG{+w}{ }\PYG{k}{NULL}\PYG{p}{,}

\PYG{+w}{    }\PYG{k}{CONSTRAINT}\PYG{+w}{ }\PYG{n}{fk\PYGZus{}klient}
\PYG{+w}{        }\PYG{k}{FOREIGN}\PYG{+w}{ }\PYG{k}{KEY}\PYG{p}{(}\PYG{n}{klient\PYGZus{}id}\PYG{p}{)}
\PYG{+w}{        }\PYG{k}{REFERENCES}\PYG{+w}{ }\PYG{n}{Klienci}\PYG{p}{(}\PYG{n}{klient\PYGZus{}id}\PYG{p}{)}
\PYG{+w}{        }\PYG{k}{ON}\PYG{+w}{ }\PYG{k}{DELETE}\PYG{+w}{ }\PYG{k}{CASCADE}\PYG{p}{,}

\PYG{+w}{    }\PYG{k}{CONSTRAINT}\PYG{+w}{ }\PYG{n}{chk\PYGZus{}typ\PYGZus{}karnetu}
\PYG{+w}{        }\PYG{k}{CHECK}\PYG{+w}{ }\PYG{p}{(}\PYG{n}{typ\PYGZus{}karnetu}\PYG{+w}{ }\PYG{k}{IN}\PYG{+w}{ }\PYG{p}{(}\PYG{l+s+s1}{\PYGZsq{}miesieczny\PYGZsq{}}\PYG{p}{,}\PYG{+w}{ }\PYG{l+s+s1}{\PYGZsq{}trzymiesieczny\PYGZsq{}}\PYG{p}{,}\PYG{+w}{ }\PYG{l+s+s1}{\PYGZsq{}polroczny\PYGZsq{}}\PYG{p}{)}\PYG{p}{)}
\PYG{p}{)}\PYG{p}{;}

\PYG{c+c1}{\PYGZhy{}\PYGZhy{} Tabela rejestrująca wejścia klientów.}
\PYG{c+c1}{\PYGZhy{}\PYGZhy{} Każde wejście jest powiązane z istniejącym klientem.}
\PYG{k}{CREATE}\PYG{+w}{ }\PYG{k}{TABLE}\PYG{+w}{ }\PYG{n}{Wejscia}\PYG{+w}{ }\PYG{p}{(}
\PYG{+w}{    }\PYG{n}{wejscie\PYGZus{}id}\PYG{+w}{ }\PYG{n+nb}{SERIAL}\PYG{+w}{ }\PYG{k}{PRIMARY}\PYG{+w}{ }\PYG{k}{KEY}\PYG{p}{,}
\PYG{+w}{    }\PYG{n}{klient\PYGZus{}id}\PYG{+w}{ }\PYG{n+nb}{INTEGER}\PYG{+w}{ }\PYG{k}{NOT}\PYG{+w}{ }\PYG{k}{NULL}\PYG{p}{,}
\PYG{+w}{    }\PYG{n}{data\PYGZus{}wejscia}\PYG{+w}{ }\PYG{k}{TIMESTAMP}\PYG{+w}{ }\PYG{k}{NOT}\PYG{+w}{ }\PYG{k}{NULL}\PYG{+w}{ }\PYG{k}{DEFAULT}\PYG{+w}{ }\PYG{k}{CURRENT\PYGZus{}TIMESTAMP}\PYG{p}{,}

\PYG{+w}{    }\PYG{k}{CONSTRAINT}\PYG{+w}{ }\PYG{n}{fk\PYGZus{}klient}
\PYG{+w}{        }\PYG{k}{FOREIGN}\PYG{+w}{ }\PYG{k}{KEY}\PYG{p}{(}\PYG{n}{klient\PYGZus{}id}\PYG{p}{)}
\PYG{+w}{        }\PYG{k}{REFERENCES}\PYG{+w}{ }\PYG{n}{Klienci}\PYG{p}{(}\PYG{n}{klient\PYGZus{}id}\PYG{p}{)}
\PYG{+w}{        }\PYG{k}{ON}\PYG{+w}{ }\PYG{k}{DELETE}\PYG{+w}{ }\PYG{k}{CASCADE}
\PYG{p}{)}\PYG{p}{;}
\end{sphinxVerbatim}

\sphinxstepscope


\chapter{4. Normalizacja, Wydajność i Bezpieczeństwo}
\label{\detokenize{rozdzial4/rozdzial4:normalizacja-wydajnosc-i-bezpieczenstwo}}\label{\detokenize{rozdzial4/rozdzial4::doc}}
\sphinxAtStartPar
W tym rozdziale przeprowadzono analizę kluczowych niefunkcjonalnych aspektów zaprojektowanej bazy danych.


\section{Analiza normalizacji}
\label{\detokenize{rozdzial4/rozdzial4:analiza-normalizacji}}
\sphinxAtStartPar
Normalizacja jest procesem projektowania schematu bazy danych w celu zminimalizowania redundancji danych i wyeliminowania niepożądanych charakterystyk, takich jak anomalie wstawiania, aktualizacji i usuwania. Zaproponowany schemat jest zgodny z \sphinxstylestrong{trzecią postacią normalną (3NF)}.

\sphinxAtStartPar
\sphinxstylestrong{Zaczynamy od: jednej dużej tabeli (czyli dane nie są jeszcze uporządkowane)}

\sphinxAtStartPar
Wyobraźmy sobie, że wszystko zapisujemy w jednej tabeli \sphinxtitleref{Rejestr\_Silowni}.


\begin{savenotes}\sphinxattablestart
\sphinxthistablewithglobalstyle
\centering
\sphinxcapstartof{table}
\sphinxthecaptionisattop
\sphinxcaption{Przykład nieuporządkowanej tabeli (\sphinxtitleref{Rejestr\_Silowni})}\label{\detokenize{rozdzial4/rozdzial4:id1}}
\sphinxaftertopcaption
\begin{tabular}[t]{\X{15}{105}\X{15}{105}\X{20}{105}\X{15}{105}\X{15}{105}\X{25}{105}}
\sphinxtoprule
\sphinxstyletheadfamily 
\sphinxAtStartPar
Imie\_Klienta
&\sphinxstyletheadfamily 
\sphinxAtStartPar
Nazwisko\_Klienta
&\sphinxstyletheadfamily 
\sphinxAtStartPar
Email\_Klienta
&\sphinxstyletheadfamily 
\sphinxAtStartPar
Typ\_Karnetu
&\sphinxstyletheadfamily 
\sphinxAtStartPar
Cena\_Karnetu
&\sphinxstyletheadfamily 
\sphinxAtStartPar
Daty\_Wejsc
\\
\sphinxmidrule
\sphinxtableatstartofbodyhook
\sphinxAtStartPar
Jan
&
\sphinxAtStartPar
Kowalski
&
\sphinxAtStartPar
\sphinxhref{mailto:jan.kowalski@ex.com}{jan.kowalski@ex.com}
&
\sphinxAtStartPar
miesieczny
&
\sphinxAtStartPar
120.00
&
\sphinxAtStartPar
«2025\sphinxhyphen{}07\sphinxhyphen{}02, 2025\sphinxhyphen{}07\sphinxhyphen{}05»
\\
\sphinxhline
\sphinxAtStartPar
Anna
&
\sphinxAtStartPar
Nowak
&
\sphinxAtStartPar
\sphinxhref{mailto:anna.nowak@ex.com}{anna.nowak@ex.com}
&
\sphinxAtStartPar
polroczny
&
\sphinxAtStartPar
500.00
&
\sphinxAtStartPar
«2025\sphinxhyphen{}07\sphinxhyphen{}03»
\\
\sphinxhline
\sphinxAtStartPar
Jan
&
\sphinxAtStartPar
Kowalski
&
\sphinxAtStartPar
\sphinxhref{mailto:jan.kowalski@ex.com}{jan.kowalski@ex.com}
&
\sphinxAtStartPar
trzymiesieczny
&
\sphinxAtStartPar
300.00
&
\sphinxAtStartPar
«2025\sphinxhyphen{}08\sphinxhyphen{}01, 2025\sphinxhyphen{}08\sphinxhyphen{}04»
\\
\sphinxbottomrule
\end{tabular}
\sphinxtableafterendhook\par
\sphinxattableend\end{savenotes}

\sphinxAtStartPar
Problemy z taką tabelą:
\begin{itemize}
\item {} 
\sphinxAtStartPar
\sphinxstylestrong{Nie można dodać klienta}, jeśli jeszcze nic nie kupił.

\item {} 
\sphinxAtStartPar
\sphinxstylestrong{Usunięcie jednego wejścia} może przypadkiem usunąć dane o kliencie.

\item {} 
\sphinxAtStartPar
\sphinxstylestrong{Zmiana adresu e\sphinxhyphen{}mail} klienta wymaga edytowania kilku wierszy.

\item {} 
\sphinxAtStartPar
\sphinxstylestrong{Dużo powtarzających się danych} — np. Jan Kowalski występuje kilka razy.

\end{itemize}

\sphinxAtStartPar
\sphinxstylestrong{Krok 1: Pierwsza postać normalna (1NF)}

\sphinxAtStartPar
Tabela jest w 1NF, jeśli nie ma list w kolumnach — każda komórka ma jedną wartość.

\sphinxAtStartPar
U nas kolumna \sphinxtitleref{Daty\_Wejsc} zawiera listy. Rozbijmy to:
\begin{enumerate}
\sphinxsetlistlabels{\arabic}{enumi}{enumii}{}{.}%
\item {} 
\sphinxAtStartPar
Tworzymy tabelę \sphinxtitleref{Klienci\_Karnety} — każdy karnet to osobny wiersz.

\item {} 
\sphinxAtStartPar
Tworzymy tabelę \sphinxtitleref{Wejscia}, gdzie każde wejście to jeden rekord.

\end{enumerate}


\begin{savenotes}\sphinxattablestart
\sphinxthistablewithglobalstyle
\centering
\sphinxcapstartof{table}
\sphinxthecaptionisattop
\sphinxcaption{Tabela \sphinxtitleref{Klienci\_Karnety} (po 1NF)}\label{\detokenize{rozdzial4/rozdzial4:id2}}
\sphinxaftertopcaption
\begin{tabular}[t]{\X{15}{90}\X{20}{90}\X{20}{90}\X{20}{90}\X{15}{90}}
\sphinxtoprule
\sphinxstyletheadfamily 
\sphinxAtStartPar
KarnetID
&\sphinxstyletheadfamily 
\sphinxAtStartPar
Imie\_Klienta
&\sphinxstyletheadfamily 
\sphinxAtStartPar
Nazwisko\_Klienta
&\sphinxstyletheadfamily 
\sphinxAtStartPar
Email\_Klienta
&\sphinxstyletheadfamily 
\sphinxAtStartPar
Typ\_Karnetu
\\
\sphinxmidrule
\sphinxtableatstartofbodyhook
\sphinxAtStartPar
1
&
\sphinxAtStartPar
Jan
&
\sphinxAtStartPar
Kowalski
&
\sphinxAtStartPar
\sphinxhref{mailto:jan.kowalski@ex.com}{jan.kowalski@ex.com}
&
\sphinxAtStartPar
miesieczny
\\
\sphinxhline
\sphinxAtStartPar
2
&
\sphinxAtStartPar
Anna
&
\sphinxAtStartPar
Nowak
&
\sphinxAtStartPar
\sphinxhref{mailto:anna.nowak@ex.com}{anna.nowak@ex.com}
&
\sphinxAtStartPar
polroczny
\\
\sphinxhline
\sphinxAtStartPar
3
&
\sphinxAtStartPar
Jan
&
\sphinxAtStartPar
Kowalski
&
\sphinxAtStartPar
\sphinxhref{mailto:jan.kowalski@ex.com}{jan.kowalski@ex.com}
&
\sphinxAtStartPar
trzymiesieczny
\\
\sphinxbottomrule
\end{tabular}
\sphinxtableafterendhook\par
\sphinxattableend\end{savenotes}


\begin{savenotes}\sphinxattablestart
\sphinxthistablewithglobalstyle
\centering
\sphinxcapstartof{table}
\sphinxthecaptionisattop
\sphinxcaption{Tabela \sphinxtitleref{Wejscia} (po 1NF)}\label{\detokenize{rozdzial4/rozdzial4:id3}}
\sphinxaftertopcaption
\begin{tabular}[t]{\X{25}{50}\X{25}{50}}
\sphinxtoprule
\sphinxstyletheadfamily 
\sphinxAtStartPar
KarnetID\_FK
&\sphinxstyletheadfamily 
\sphinxAtStartPar
Data\_Wejscia
\\
\sphinxmidrule
\sphinxtableatstartofbodyhook
\sphinxAtStartPar
1
&
\sphinxAtStartPar
2025\sphinxhyphen{}07\sphinxhyphen{}02
\\
\sphinxhline
\sphinxAtStartPar
1
&
\sphinxAtStartPar
2025\sphinxhyphen{}07\sphinxhyphen{}05
\\
\sphinxhline
\sphinxAtStartPar
2
&
\sphinxAtStartPar
2025\sphinxhyphen{}07\sphinxhyphen{}03
\\
\sphinxhline
\sphinxAtStartPar
3
&
\sphinxAtStartPar
2025\sphinxhyphen{}08\sphinxhyphen{}01
\\
\sphinxhline
\sphinxAtStartPar
3
&
\sphinxAtStartPar
2025\sphinxhyphen{}08\sphinxhyphen{}04
\\
\sphinxbottomrule
\end{tabular}
\sphinxtableafterendhook\par
\sphinxattableend\end{savenotes}

\sphinxAtStartPar
\sphinxstylestrong{Problem:} Nadal mamy powtarzające się dane o klientach w \sphinxtitleref{Klienci\_Karnety}.

\sphinxAtStartPar
\sphinxstylestrong{Krok 2: Druga postać normalna (2NF)}

\sphinxAtStartPar
W 2NF dane powinny zależeć od całego klucza głównego, a nie tylko części.

\sphinxAtStartPar
W naszej tabeli \sphinxtitleref{Klienci\_Karnety} dane o kliencie nie zależą od \sphinxtitleref{KarnetID}, tylko od klienta. Trzeba to rozdzielić:
\begin{enumerate}
\sphinxsetlistlabels{\arabic}{enumi}{enumii}{}{.}%
\item {} 
\sphinxAtStartPar
Tworzymy tabelę \sphinxtitleref{Klienci} z unikalnym ID.

\item {} 
\sphinxAtStartPar
W tabeli \sphinxtitleref{Karnety} trzymamy tylko typ karnetu i odwołanie do klienta.

\end{enumerate}


\begin{savenotes}\sphinxattablestart
\sphinxthistablewithglobalstyle
\centering
\sphinxcapstartof{table}
\sphinxthecaptionisattop
\sphinxcaption{Tabela \sphinxtitleref{Klienci} (po 2NF)}\label{\detokenize{rozdzial4/rozdzial4:id4}}
\sphinxaftertopcaption
\begin{tabular}[t]{\X{20}{100}\X{25}{100}\X{25}{100}\X{30}{100}}
\sphinxtoprule
\sphinxstyletheadfamily 
\sphinxAtStartPar
KlientID
&\sphinxstyletheadfamily 
\sphinxAtStartPar
Imie
&\sphinxstyletheadfamily 
\sphinxAtStartPar
Nazwisko
&\sphinxstyletheadfamily 
\sphinxAtStartPar
Email
\\
\sphinxmidrule
\sphinxtableatstartofbodyhook
\sphinxAtStartPar
101
&
\sphinxAtStartPar
Jan
&
\sphinxAtStartPar
Kowalski
&
\sphinxAtStartPar
\sphinxhref{mailto:jan.kowalski@ex.com}{jan.kowalski@ex.com}
\\
\sphinxhline
\sphinxAtStartPar
102
&
\sphinxAtStartPar
Anna
&
\sphinxAtStartPar
Nowak
&
\sphinxAtStartPar
\sphinxhref{mailto:anna.nowak@ex.com}{anna.nowak@ex.com}
\\
\sphinxbottomrule
\end{tabular}
\sphinxtableafterendhook\par
\sphinxattableend\end{savenotes}


\begin{savenotes}\sphinxattablestart
\sphinxthistablewithglobalstyle
\centering
\sphinxcapstartof{table}
\sphinxthecaptionisattop
\sphinxcaption{Tabela \sphinxtitleref{Karnety} (po 2NF)}\label{\detokenize{rozdzial4/rozdzial4:id5}}
\sphinxaftertopcaption
\begin{tabular}[t]{\X{25}{75}\X{25}{75}\X{25}{75}}
\sphinxtoprule
\sphinxstyletheadfamily 
\sphinxAtStartPar
KarnetID
&\sphinxstyletheadfamily 
\sphinxAtStartPar
KlientID\_FK
&\sphinxstyletheadfamily 
\sphinxAtStartPar
Typ\_Karnetu
\\
\sphinxmidrule
\sphinxtableatstartofbodyhook
\sphinxAtStartPar
1
&
\sphinxAtStartPar
101
&
\sphinxAtStartPar
miesieczny
\\
\sphinxhline
\sphinxAtStartPar
2
&
\sphinxAtStartPar
102
&
\sphinxAtStartPar
polroczny
\\
\sphinxhline
\sphinxAtStartPar
3
&
\sphinxAtStartPar
101
&
\sphinxAtStartPar
trzymiesieczny
\\
\sphinxbottomrule
\end{tabular}
\sphinxtableafterendhook\par
\sphinxattableend\end{savenotes}

\sphinxAtStartPar
Tabela \sphinxtitleref{Wejscia} zostaje bez zmian, ale możemy też rozważyć, czy nie lepiej byłoby wiązać ją bezpośrednio z klientem.

\sphinxAtStartPar
\sphinxstylestrong{Krok 3: Trzecia postać normalna (3NF)}

\sphinxAtStartPar
Tutaj chodzi o to, żeby dane nie zależały od innych danych niebędących kluczem.

\sphinxAtStartPar
Jeśli w \sphinxtitleref{Karnety} dodamy np. \sphinxtitleref{Cena}, która zależy od \sphinxtitleref{Typ\_Karnetu}, to mamy tzw. zależność przechodnią.

\sphinxAtStartPar
Zamiast tego możemy utworzyć tabelę \sphinxtitleref{Cennik} z kolumnami \sphinxtitleref{Typ\_Karnetu} i \sphinxtitleref{Cena}.

\sphinxAtStartPar
W naszym projekcie jednak \sphinxstylestrong{trzymamy cenę w tabeli \textasciigrave{}Karnety\textasciigrave{}}, ponieważ może się ona zmieniać w czasie — to celowe odstępstwo (denormalizacja), żeby zachować historię.

\sphinxAtStartPar
\sphinxstylestrong{Podsumowanie}

\sphinxAtStartPar
Zaczęliśmy od nieuporządkowanej tabeli, a skończyliśmy na trzech powiązanych:
\begin{itemize}
\item {} 
\sphinxAtStartPar
\sphinxtitleref{Klienci}

\item {} 
\sphinxAtStartPar
\sphinxtitleref{Karnety}

\item {} 
\sphinxAtStartPar
\sphinxtitleref{Wejscia}

\end{itemize}

\sphinxAtStartPar
Dzięki temu dane nie powtarzają się, łatwo je edytować i są bezpieczne przed przypadkowymi błędami.


\section{Analiza wydajności i indeksowanie}
\label{\detokenize{rozdzial4/rozdzial4:analiza-wydajnosci-i-indeksowanie}}
\sphinxAtStartPar
Wydajność zapytań jest kluczowa dla responsywności systemu. Podstawową techniką optymalizacji jest strategiczne stosowanie indeksów.

\sphinxAtStartPar
\sphinxstylestrong{Identyfikacja kandydatów do indeksowania:}
* \sphinxstylestrong{Klucze obce:} Kolumny używane jako klucze obce (\sphinxtitleref{Karnety.klient\_id}, \sphinxtitleref{Wejscia.klient\_id}) są głównymi kandydatami do indeksowania. Indeksy te drastycznie przyspieszają operacje \sphinxtitleref{JOIN} oraz wyszukiwanie rekordów powiązanych z danym klientem. PostgreSQL automatycznie nie tworzy indeksów na kluczach obcych, więc należy je dodać ręcznie.
* \sphinxstylestrong{Często filtrowane kolumny:} Kolumna \sphinxtitleref{Karnety.data\_waznosci} będzie często używana w klauzuli \sphinxtitleref{WHERE} do sprawdzania aktywnych karnetów. Dodanie na niej indeksu przyspieszy ten krytyczny proces biznesowy.

\sphinxAtStartPar
\sphinxstylestrong{Przykładowa implementacja indeksów:}

\begin{sphinxVerbatim}[commandchars=\\\{\}]
\PYG{c+c1}{\PYGZhy{}\PYGZhy{} Indeks na kluczu obcym w tabeli Karnety}
\PYG{k}{CREATE}\PYG{+w}{ }\PYG{k}{INDEX}\PYG{+w}{ }\PYG{n}{idx\PYGZus{}karnety\PYGZus{}klient\PYGZus{}id}\PYG{+w}{ }\PYG{k}{ON}\PYG{+w}{ }\PYG{n}{Karnety}\PYG{p}{(}\PYG{n}{klient\PYGZus{}id}\PYG{p}{)}\PYG{p}{;}

\PYG{c+c1}{\PYGZhy{}\PYGZhy{} Indeks na kluczu obcym w tabeli Wejscia}
\PYG{k}{CREATE}\PYG{+w}{ }\PYG{k}{INDEX}\PYG{+w}{ }\PYG{n}{idx\PYGZus{}wejscia\PYGZus{}klient\PYGZus{}id}\PYG{+w}{ }\PYG{k}{ON}\PYG{+w}{ }\PYG{n}{Wejscia}\PYG{p}{(}\PYG{n}{klient\PYGZus{}id}\PYG{p}{)}\PYG{p}{;}

\PYG{c+c1}{\PYGZhy{}\PYGZhy{} Indeks wspomagający wyszukiwanie aktywnych karnetów}
\PYG{k}{CREATE}\PYG{+w}{ }\PYG{k}{INDEX}\PYG{+w}{ }\PYG{n}{idx\PYGZus{}karnety\PYGZus{}data\PYGZus{}waznosci}\PYG{+w}{ }\PYG{k}{ON}\PYG{+w}{ }\PYG{n}{Karnety}\PYG{p}{(}\PYG{n}{data\PYGZus{}waznosci}\PYG{p}{)}\PYG{p}{;}
\end{sphinxVerbatim}

\sphinxAtStartPar
\sphinxstylestrong{Analiza planu zapytania (\textasciigrave{}EXPLAIN ANALYZE\textasciigrave{}):}
Przed dodaniem indeksu \sphinxtitleref{idx\_karnety\_klient\_id}, zapytanie o wszystkie karnety danego klienta skutkowałoby pełnym skanowaniem tabeli (\sphinxtitleref{Seq Scan}). Po jego dodaniu, planer zapytań PostgreSQL wykorzysta znacznie szybszy \sphinxtitleref{Index Scan}, co przy dużej liczbie rekordów może skrócić czas wykonania zapytania z sekund do milisekund.


\section{Zarządzanie bezpieczeństwem}
\label{\detokenize{rozdzial4/rozdzial4:zarzadzanie-bezpieczenstwem}}
\sphinxAtStartPar
Bezpieczeństwo danych osobowych i operacyjnych jest priorytetem. Zastosowano model bezpieczeństwa oparty na rolach (Role\sphinxhyphen{}Based Access Control).

\sphinxAtStartPar
\sphinxstylestrong{Definicja ról:}
* \sphinxstylestrong{\textasciigrave{}rola\_admin\textasciigrave{}}: Superużytkownik z pełnymi uprawnieniami do wszystkich tabel (CRUD \sphinxhyphen{} Create, Read, Update, Delete). Przeznaczona dla administratorów bazy danych.
* \sphinxstylestrong{\textasciigrave{}rola\_recepcja\textasciigrave{}}: Rola dla pracowników recepcji. Powinna mieć uprawnienia do:
\begin{itemize}
\item {} 
\sphinxAtStartPar
\sphinxtitleref{SELECT} na \sphinxtitleref{Klienci}.

\item {} 
\sphinxAtStartPar
\sphinxtitleref{INSERT} do \sphinxtitleref{Klienci}.

\item {} 
\sphinxAtStartPar
\sphinxtitleref{SELECT}, \sphinxtitleref{INSERT} na \sphinxtitleref{Karnety}.

\item {} 
\sphinxAtStartPar
\sphinxtitleref{SELECT}, \sphinxtitleref{INSERT} na \sphinxtitleref{Wejscia}.

\item {} 
\sphinxAtStartPar
Brak uprawnień \sphinxtitleref{DELETE} i \sphinxtitleref{UPDATE} na większości danych w celu ochrony przed przypadkowym usunięciem.

\end{itemize}
\begin{itemize}
\item {} 
\sphinxAtStartPar
\sphinxstylestrong{\textasciigrave{}rola\_analityk\textasciigrave{}}: Rola tylko do odczytu (\sphinxtitleref{SELECT}) na wszystkich tabelach. Przeznaczona dla analityków biznesowych generujących raporty.

\end{itemize}

\sphinxAtStartPar
\sphinxstylestrong{Przykładowa implementacja ról i uprawnień:}

\begin{sphinxVerbatim}[commandchars=\\\{\}]
\PYG{c+c1}{\PYGZhy{}\PYGZhy{} Tworzenie ról}
\PYG{k}{CREATE}\PYG{+w}{ }\PYG{k}{ROLE}\PYG{+w}{ }\PYG{n}{rola\PYGZus{}recepcja}\PYG{p}{;}
\PYG{k}{CREATE}\PYG{+w}{ }\PYG{k}{ROLE}\PYG{+w}{ }\PYG{n}{rola\PYGZus{}analityk}\PYG{p}{;}

\PYG{c+c1}{\PYGZhy{}\PYGZhy{} Nadawanie uprawnień dla recepcji}
\PYG{k}{GRANT}\PYG{+w}{ }\PYG{k}{SELECT}\PYG{p}{,}\PYG{+w}{ }\PYG{k}{INSERT}\PYG{+w}{ }\PYG{k}{ON}\PYG{+w}{ }\PYG{n}{Klienci}\PYG{p}{,}\PYG{+w}{ }\PYG{n}{Karnety}\PYG{p}{,}\PYG{+w}{ }\PYG{n}{Wejscia}\PYG{+w}{ }\PYG{k}{TO}\PYG{+w}{ }\PYG{n}{rola\PYGZus{}recepcja}\PYG{p}{;}
\PYG{k}{GRANT}\PYG{+w}{ }\PYG{k}{USAGE}\PYG{p}{,}\PYG{+w}{ }\PYG{k}{SELECT}\PYG{+w}{ }\PYG{k}{ON}\PYG{+w}{ }\PYG{n}{SEQUENCE}\PYG{+w}{ }\PYG{n}{klienci\PYGZus{}klient\PYGZus{}id\PYGZus{}seq}\PYG{p}{,}\PYG{+w}{ }\PYG{n}{karnety\PYGZus{}karnet\PYGZus{}id\PYGZus{}seq}\PYG{p}{,}\PYG{+w}{ }\PYG{n}{wejscia\PYGZus{}wejscie\PYGZus{}id\PYGZus{}seq}\PYG{+w}{ }\PYG{k}{TO}\PYG{+w}{ }\PYG{n}{rola\PYGZus{}recepcja}\PYG{p}{;}


\PYG{c+c1}{\PYGZhy{}\PYGZhy{} Nadawanie uprawnień dla analityka}
\PYG{k}{GRANT}\PYG{+w}{ }\PYG{k}{SELECT}\PYG{+w}{ }\PYG{k}{ON}\PYG{+w}{ }\PYG{k}{ALL}\PYG{+w}{ }\PYG{n}{TABLES}\PYG{+w}{ }\PYG{k}{IN}\PYG{+w}{ }\PYG{k}{SCHEMA}\PYG{+w}{ }\PYG{k}{public}\PYG{+w}{ }\PYG{k}{TO}\PYG{+w}{ }\PYG{n}{rola\PYGZus{}analityk}\PYG{p}{;}

\PYG{c+c1}{\PYGZhy{}\PYGZhy{} Tworzenie użytkowników i przypisywanie im ról}
\PYG{k}{CREATE}\PYG{+w}{ }\PYG{k}{USER}\PYG{+w}{ }\PYG{n}{pracownik\PYGZus{}recepcji}\PYG{+w}{ }\PYG{k}{WITH}\PYG{+w}{ }\PYG{n}{PASSWORD}\PYG{+w}{ }\PYG{l+s+s1}{\PYGZsq{}bezpieczne\PYGZus{}haslo\PYGZsq{}}\PYG{p}{;}
\PYG{k}{GRANT}\PYG{+w}{ }\PYG{n}{rola\PYGZus{}recepcja}\PYG{+w}{ }\PYG{k}{TO}\PYG{+w}{ }\PYG{n}{pracownik\PYGZus{}recepcji}\PYG{p}{;}
\end{sphinxVerbatim}


\section{Skrypty wspomagające}
\label{\detokenize{rozdzial4/rozdzial4:skrypty-wspomagajace}}\sphinxSetupCaptionForVerbatim{Skrypty w PostgreSQL}
\def\sphinxLiteralBlockLabel{\label{\detokenize{rozdzial4/rozdzial4:id6}}}
\begin{sphinxVerbatim}[commandchars=\\\{\}]
\PYG{k+kn}{import}\PYG{+w}{ }\PYG{n+nn}{psycopg2}
\PYG{k+kn}{from}\PYG{+w}{ }\PYG{n+nn}{datetime}\PYG{+w}{ }\PYG{k+kn}{import} \PYG{n}{date}\PYG{p}{,} \PYG{n}{timedelta}

\PYG{c+c1}{\PYGZsh{} ... (konfiguracja połączenia DB\PYGZus{}CONFIG) ...}

\PYG{k}{def}\PYG{+w}{ }\PYG{n+nf}{generuj\PYGZus{}raport\PYGZus{}wygasajacych\PYGZus{}karnetow}\PYG{p}{(}\PYG{n}{dni\PYGZus{}do\PYGZus{}konca}\PYG{o}{=}\PYG{l+m+mi}{7}\PYG{p}{)}\PYG{p}{:}
\PYG{+w}{    }\PYG{l+s+sd}{\PYGZdq{}\PYGZdq{}\PYGZdq{}}
\PYG{l+s+sd}{    Znajduje klientów, których karnety wygasają}
\PYG{l+s+sd}{    w ciągu najbliższych \PYGZsq{}dni\PYGZus{}do\PYGZus{}konca\PYGZsq{} dni.}
\PYG{l+s+sd}{    \PYGZdq{}\PYGZdq{}\PYGZdq{}}
    \PYG{c+c1}{\PYGZsh{} ... (logika połączenia z bazą) ...}
    \PYG{n}{query} \PYG{o}{=} \PYG{l+s+s2}{\PYGZdq{}\PYGZdq{}\PYGZdq{}}
\PYG{l+s+s2}{    SELECT k.imie, k.nazwisko, k.email, kr.data\PYGZus{}waznosci}
\PYG{l+s+s2}{    FROM Klienci k}
\PYG{l+s+s2}{    JOIN Karnety kr ON k.klient\PYGZus{}id = kr.klient\PYGZus{}id}
\PYG{l+s+s2}{    WHERE kr.data\PYGZus{}waznosci BETWEEN }\PYG{l+s+si}{\PYGZpc{}s}\PYG{l+s+s2}{ AND }\PYG{l+s+si}{\PYGZpc{}s}
\PYG{l+s+s2}{    ORDER BY kr.data\PYGZus{}waznosci ASC;}
\PYG{l+s+s2}{    }\PYG{l+s+s2}{\PYGZdq{}\PYGZdq{}\PYGZdq{}}
    \PYG{n}{dzis} \PYG{o}{=} \PYG{n}{date}\PYG{o}{.}\PYG{n}{today}\PYG{p}{(}\PYG{p}{)}
    \PYG{n}{data\PYGZus{}koncowa} \PYG{o}{=} \PYG{n}{dzis} \PYG{o}{+} \PYG{n}{timedelta}\PYG{p}{(}\PYG{n}{days}\PYG{o}{=}\PYG{n}{dni\PYGZus{}do\PYGZus{}konca}\PYG{p}{)}
    \PYG{n}{cur}\PYG{o}{.}\PYG{n}{execute}\PYG{p}{(}\PYG{n}{query}\PYG{p}{,} \PYG{p}{(}\PYG{n}{dzis}\PYG{p}{,} \PYG{n}{data\PYGZus{}koncowa}\PYG{p}{)}\PYG{p}{)}
    \PYG{c+c1}{\PYGZsh{} ... (logika wyświetlania raportu) ...}

\PYG{k}{def}\PYG{+w}{ }\PYG{n+nf}{znajdz\PYGZus{}najaktywniejszych\PYGZus{}klientow}\PYG{p}{(}\PYG{n}{data\PYGZus{}od}\PYG{p}{,} \PYG{n}{data\PYGZus{}do}\PYG{p}{,} \PYG{n}{limit}\PYG{o}{=}\PYG{l+m+mi}{5}\PYG{p}{)}\PYG{p}{:}
\PYG{+w}{ }\PYG{l+s+sd}{\PYGZdq{}\PYGZdq{}\PYGZdq{}Wyświetla listę najczęściej wchodzących klientów w danym okresie.\PYGZdq{}\PYGZdq{}\PYGZdq{}}
 \PYG{n+nb}{print}\PYG{p}{(}\PYG{l+s+sa}{f}\PYG{l+s+s2}{\PYGZdq{}}\PYG{l+s+se}{\PYGZbs{}n}\PYG{l+s+s2}{\PYGZhy{}\PYGZhy{}\PYGZhy{} TOP }\PYG{l+s+si}{\PYGZob{}}\PYG{n}{limit}\PYG{l+s+si}{\PYGZcb{}}\PYG{l+s+s2}{ najaktywniejszych klientów od }\PYG{l+s+si}{\PYGZob{}}\PYG{n}{data\PYGZus{}od}\PYG{l+s+si}{\PYGZcb{}}\PYG{l+s+s2}{ do }\PYG{l+s+si}{\PYGZob{}}\PYG{n}{data\PYGZus{}do}\PYG{l+s+si}{\PYGZcb{}}\PYG{l+s+s2}{ \PYGZhy{}\PYGZhy{}\PYGZhy{}}\PYG{l+s+s2}{\PYGZdq{}}\PYG{p}{)}
 \PYG{n}{conn} \PYG{o}{=} \PYG{n}{get\PYGZus{}connection}\PYG{p}{(}\PYG{p}{)}
 \PYG{n}{query} \PYG{o}{=} \PYG{l+s+s2}{\PYGZdq{}\PYGZdq{}\PYGZdq{}}
\PYG{l+s+s2}{ SELECT k.imie, k.nazwisko, COUNT(w.wejscie\PYGZus{}id) AS liczba\PYGZus{}wejsc}
\PYG{l+s+s2}{ FROM Wejscia w}
\PYG{l+s+s2}{ JOIN Klienci k ON w.klient\PYGZus{}id = k.klient\PYGZus{}id}
\PYG{l+s+s2}{ WHERE w.data\PYGZus{}wejscia::date BETWEEN }\PYG{l+s+si}{\PYGZpc{}s}\PYG{l+s+s2}{ AND }\PYG{l+s+si}{\PYGZpc{}s}
\PYG{l+s+s2}{ GROUP BY k.klient\PYGZus{}id, k.imie, k.nazwisko}
\PYG{l+s+s2}{ ORDER BY liczba\PYGZus{}wejsc DESC}
\PYG{l+s+s2}{ LIMIT }\PYG{l+s+si}{\PYGZpc{}s}\PYG{l+s+s2}{;}
\PYG{l+s+s2}{ }\PYG{l+s+s2}{\PYGZdq{}\PYGZdq{}\PYGZdq{}}
 \PYG{k}{with} \PYG{n}{conn}\PYG{o}{.}\PYG{n}{cursor}\PYG{p}{(}\PYG{p}{)} \PYG{k}{as} \PYG{n}{cur}\PYG{p}{:}
     \PYG{n}{cur}\PYG{o}{.}\PYG{n}{execute}\PYG{p}{(}\PYG{n}{query}\PYG{p}{,} \PYG{p}{(}\PYG{n}{data\PYGZus{}od}\PYG{p}{,} \PYG{n}{data\PYGZus{}do}\PYG{p}{,} \PYG{n}{limit}\PYG{p}{)}\PYG{p}{)}
     \PYG{k}{for} \PYG{n}{row} \PYG{o+ow}{in} \PYG{n}{cur}\PYG{o}{.}\PYG{n}{fetchall}\PYG{p}{(}\PYG{p}{)}\PYG{p}{:}
         \PYG{n+nb}{print}\PYG{p}{(}\PYG{l+s+sa}{f}\PYG{l+s+s2}{\PYGZdq{}}\PYG{l+s+s2}{Klient: }\PYG{l+s+si}{\PYGZob{}}\PYG{n}{row}\PYG{p}{[}\PYG{l+m+mi}{0}\PYG{p}{]}\PYG{l+s+si}{\PYGZcb{}}\PYG{l+s+s2}{ }\PYG{l+s+si}{\PYGZob{}}\PYG{n}{row}\PYG{p}{[}\PYG{l+m+mi}{1}\PYG{p}{]}\PYG{l+s+si}{\PYGZcb{}}\PYG{l+s+s2}{, Liczba wejść: }\PYG{l+s+si}{\PYGZob{}}\PYG{n}{row}\PYG{p}{[}\PYG{l+m+mi}{2}\PYG{p}{]}\PYG{l+s+si}{\PYGZcb{}}\PYG{l+s+s2}{\PYGZdq{}}\PYG{p}{)}
 \PYG{n}{conn}\PYG{o}{.}\PYG{n}{close}\PYG{p}{(}\PYG{p}{)}

 \PYG{k}{def}\PYG{+w}{ }\PYG{n+nf}{generuj\PYGZus{}raport\PYGZus{}sprzedazy}\PYG{p}{(}\PYG{n}{data\PYGZus{}od}\PYG{p}{,} \PYG{n}{data\PYGZus{}do}\PYG{p}{)}\PYG{p}{:}
\PYG{+w}{ }\PYG{l+s+sd}{\PYGZdq{}\PYGZdq{}\PYGZdq{}Oblicza sumę sprzedaży i liczbę sprzedanych karnetów w danym okresie.\PYGZdq{}\PYGZdq{}\PYGZdq{}}
 \PYG{n+nb}{print}\PYG{p}{(}\PYG{l+s+sa}{f}\PYG{l+s+s2}{\PYGZdq{}}\PYG{l+s+se}{\PYGZbs{}n}\PYG{l+s+s2}{\PYGZhy{}\PYGZhy{}\PYGZhy{} Raport sprzedaży od }\PYG{l+s+si}{\PYGZob{}}\PYG{n}{data\PYGZus{}od}\PYG{l+s+si}{\PYGZcb{}}\PYG{l+s+s2}{ do }\PYG{l+s+si}{\PYGZob{}}\PYG{n}{data\PYGZus{}do}\PYG{l+s+si}{\PYGZcb{}}\PYG{l+s+s2}{ \PYGZhy{}\PYGZhy{}\PYGZhy{}}\PYG{l+s+s2}{\PYGZdq{}}\PYG{p}{)}
 \PYG{n}{conn} \PYG{o}{=} \PYG{n}{get\PYGZus{}connection}\PYG{p}{(}\PYG{p}{)}
 \PYG{n}{query} \PYG{o}{=} \PYG{l+s+s2}{\PYGZdq{}}\PYG{l+s+s2}{SELECT COUNT(karnet\PYGZus{}id), SUM(cena) FROM Karnety WHERE data\PYGZus{}zakupu BETWEEN }\PYG{l+s+si}{\PYGZpc{}s}\PYG{l+s+s2}{ AND }\PYG{l+s+si}{\PYGZpc{}s}\PYG{l+s+s2}{;}\PYG{l+s+s2}{\PYGZdq{}}
 \PYG{k}{with} \PYG{n}{conn}\PYG{o}{.}\PYG{n}{cursor}\PYG{p}{(}\PYG{p}{)} \PYG{k}{as} \PYG{n}{cur}\PYG{p}{:}
     \PYG{n}{cur}\PYG{o}{.}\PYG{n}{execute}\PYG{p}{(}\PYG{n}{query}\PYG{p}{,} \PYG{p}{(}\PYG{n}{data\PYGZus{}od}\PYG{p}{,} \PYG{n}{data\PYGZus{}do}\PYG{p}{)}\PYG{p}{)}
     \PYG{n}{result} \PYG{o}{=} \PYG{n}{cur}\PYG{o}{.}\PYG{n}{fetchone}\PYG{p}{(}\PYG{p}{)}
     \PYG{n+nb}{print}\PYG{p}{(}\PYG{l+s+sa}{f}\PYG{l+s+s2}{\PYGZdq{}}\PYG{l+s+s2}{Liczba sprzedanych karnetów: }\PYG{l+s+si}{\PYGZob{}}\PYG{n}{result}\PYG{p}{[}\PYG{l+m+mi}{0}\PYG{p}{]}\PYG{+w}{ }\PYG{o+ow}{or}\PYG{+w}{ }\PYG{l+m+mi}{0}\PYG{l+s+si}{\PYGZcb{}}\PYG{l+s+s2}{\PYGZdq{}}\PYG{p}{)}
     \PYG{n+nb}{print}\PYG{p}{(}\PYG{l+s+sa}{f}\PYG{l+s+s2}{\PYGZdq{}}\PYG{l+s+s2}{Łączna kwota sprzedaży: }\PYG{l+s+si}{\PYGZob{}}\PYG{n}{result}\PYG{p}{[}\PYG{l+m+mi}{1}\PYG{p}{]}\PYG{+w}{ }\PYG{o+ow}{or}\PYG{+w}{ }\PYG{l+m+mf}{0.00}\PYG{l+s+si}{\PYGZcb{}}\PYG{l+s+s2}{ PLN}\PYG{l+s+s2}{\PYGZdq{}}\PYG{p}{)}
 \PYG{n}{conn}\PYG{o}{.}\PYG{n}{close}\PYG{p}{(}\PYG{p}{)}
\end{sphinxVerbatim}
\sphinxSetupCaptionForVerbatim{Skrypty w SQLite}
\def\sphinxLiteralBlockLabel{\label{\detokenize{rozdzial4/rozdzial4:id7}}}
\begin{sphinxVerbatim}[commandchars=\\\{\}]
\PYG{k+kn}{import}\PYG{+w}{ }\PYG{n+nn}{psycopg2}
\PYG{k+kn}{from}\PYG{+w}{ }\PYG{n+nn}{datetime}\PYG{+w}{ }\PYG{k+kn}{import} \PYG{n}{date}\PYG{p}{,} \PYG{n}{timedelta}

\PYG{c+c1}{\PYGZsh{} \PYGZhy{}\PYGZhy{}\PYGZhy{} KONFIGURACJA \PYGZhy{}\PYGZhy{}\PYGZhy{}}
\PYG{n}{DB\PYGZus{}FILE} \PYG{o}{=} \PYG{l+s+s2}{\PYGZdq{}}\PYG{l+s+s2}{silownia.db}\PYG{l+s+s2}{\PYGZdq{}} \PYG{c+c1}{\PYGZsh{} Nazwa pliku bazy danych}

\PYG{k}{def}\PYG{+w}{ }\PYG{n+nf}{get\PYGZus{}connection}\PYG{p}{(}\PYG{p}{)}\PYG{p}{:}
\PYG{+w}{ }\PYG{l+s+sd}{\PYGZdq{}\PYGZdq{}\PYGZdq{}Nawiązuje połączenie z bazą danych SQLite.\PYGZdq{}\PYGZdq{}\PYGZdq{}}
 \PYG{k}{return} \PYG{n}{sqlite3}\PYG{o}{.}\PYG{n}{connect}\PYG{p}{(}\PYG{n}{DB\PYGZus{}FILE}\PYG{p}{)}

 \PYG{k}{def}\PYG{+w}{ }\PYG{n+nf}{znajdz\PYGZus{}klientow\PYGZus{}z\PYGZus{}wygaslym\PYGZus{}karnetem\PYGZus{}sqlite}\PYG{p}{(}\PYG{n}{dni\PYGZus{}od\PYGZus{}wyga\PYGZus{}do\PYGZus{}wyga}\PYG{p}{)}\PYG{p}{:}
\PYG{+w}{ }\PYG{l+s+sd}{\PYGZdq{}\PYGZdq{}\PYGZdq{}Znajduje klientów, których ostatni karnet wygasł w zadanym przedziale dni temu.\PYGZdq{}\PYGZdq{}\PYGZdq{}}
 \PYG{n+nb}{print}\PYG{p}{(}\PYG{l+s+sa}{f}\PYG{l+s+s2}{\PYGZdq{}}\PYG{l+s+se}{\PYGZbs{}n}\PYG{l+s+s2}{\PYGZhy{}\PYGZhy{}\PYGZhy{} [SQLite] Klienci, których karnet wygasł od }\PYG{l+s+si}{\PYGZob{}}\PYG{n}{dni\PYGZus{}od\PYGZus{}wyga\PYGZus{}do\PYGZus{}wyga}\PYG{p}{[}\PYG{l+m+mi}{0}\PYG{p}{]}\PYG{l+s+si}{\PYGZcb{}}\PYG{l+s+s2}{ do }\PYG{l+s+si}{\PYGZob{}}\PYG{n}{dni\PYGZus{}od\PYGZus{}wyga\PYGZus{}do\PYGZus{}wyga}\PYG{p}{[}\PYG{l+m+mi}{1}\PYG{p}{]}\PYG{l+s+si}{\PYGZcb{}}\PYG{l+s+s2}{ dni temu \PYGZhy{}\PYGZhy{}\PYGZhy{}}\PYG{l+s+s2}{\PYGZdq{}}\PYG{p}{)}
 \PYG{n}{conn} \PYG{o}{=} \PYG{n}{get\PYGZus{}connection}\PYG{p}{(}\PYG{p}{)}
 \PYG{c+c1}{\PYGZsh{} W SQLite do znalezienia ostatniego karnetu używamy podzapytania z GROUP BY i MAX()}
 \PYG{n}{query} \PYG{o}{=} \PYG{l+s+s2}{\PYGZdq{}\PYGZdq{}\PYGZdq{}}
\PYG{l+s+s2}{ SELECT k.imie, k.nazwisko, k.email, sub.max\PYGZus{}data}
\PYG{l+s+s2}{ FROM Klienci k}
\PYG{l+s+s2}{ JOIN (}
\PYG{l+s+s2}{     SELECT klient\PYGZus{}id, MAX(data\PYGZus{}waznosci) as max\PYGZus{}data FROM Karnety GROUP BY klient\PYGZus{}id}
\PYG{l+s+s2}{ ) AS sub ON k.klient\PYGZus{}id = sub.klient\PYGZus{}id}
\PYG{l+s+s2}{ WHERE sub.max\PYGZus{}data BETWEEN ? AND ?;}
\PYG{l+s+s2}{ }\PYG{l+s+s2}{\PYGZdq{}\PYGZdq{}\PYGZdq{}}
 \PYG{n}{date\PYGZus{}to} \PYG{o}{=} \PYG{p}{(}\PYG{n}{date}\PYG{o}{.}\PYG{n}{today}\PYG{p}{(}\PYG{p}{)} \PYG{o}{\PYGZhy{}} \PYG{n}{timedelta}\PYG{p}{(}\PYG{n}{days}\PYG{o}{=}\PYG{n}{dni\PYGZus{}od\PYGZus{}wyga\PYGZus{}do\PYGZus{}wyga}\PYG{p}{[}\PYG{l+m+mi}{0}\PYG{p}{]}\PYG{p}{)}\PYG{p}{)}\PYG{o}{.}\PYG{n}{isoformat}\PYG{p}{(}\PYG{p}{)}
 \PYG{n}{date\PYGZus{}from} \PYG{o}{=} \PYG{p}{(}\PYG{n}{date}\PYG{o}{.}\PYG{n}{today}\PYG{p}{(}\PYG{p}{)} \PYG{o}{\PYGZhy{}} \PYG{n}{timedelta}\PYG{p}{(}\PYG{n}{days}\PYG{o}{=}\PYG{n}{dni\PYGZus{}od\PYGZus{}wyga\PYGZus{}do\PYGZus{}wyga}\PYG{p}{[}\PYG{l+m+mi}{1}\PYG{p}{]}\PYG{p}{)}\PYG{p}{)}\PYG{o}{.}\PYG{n}{isoformat}\PYG{p}{(}\PYG{p}{)}

 \PYG{k}{with} \PYG{n}{conn}\PYG{p}{:} \PYG{c+c1}{\PYGZsh{} Użycie `with conn` automatycznie zarządza transakcjami}
     \PYG{n}{cur} \PYG{o}{=} \PYG{n}{conn}\PYG{o}{.}\PYG{n}{cursor}\PYG{p}{(}\PYG{p}{)}
     \PYG{n}{cur}\PYG{o}{.}\PYG{n}{execute}\PYG{p}{(}\PYG{n}{query}\PYG{p}{,} \PYG{p}{(}\PYG{n}{date\PYGZus{}from}\PYG{p}{,} \PYG{n}{date\PYGZus{}to}\PYG{p}{)}\PYG{p}{)}
     \PYG{k}{for} \PYG{n}{row} \PYG{o+ow}{in} \PYG{n}{cur}\PYG{o}{.}\PYG{n}{fetchall}\PYG{p}{(}\PYG{p}{)}\PYG{p}{:}
         \PYG{n+nb}{print}\PYG{p}{(}\PYG{l+s+sa}{f}\PYG{l+s+s2}{\PYGZdq{}}\PYG{l+s+s2}{Klient: }\PYG{l+s+si}{\PYGZob{}}\PYG{n}{row}\PYG{p}{[}\PYG{l+m+mi}{0}\PYG{p}{]}\PYG{l+s+si}{\PYGZcb{}}\PYG{l+s+s2}{ }\PYG{l+s+si}{\PYGZob{}}\PYG{n}{row}\PYG{p}{[}\PYG{l+m+mi}{1}\PYG{p}{]}\PYG{l+s+si}{\PYGZcb{}}\PYG{l+s+s2}{, Email: }\PYG{l+s+si}{\PYGZob{}}\PYG{n}{row}\PYG{p}{[}\PYG{l+m+mi}{2}\PYG{p}{]}\PYG{l+s+si}{\PYGZcb{}}\PYG{l+s+s2}{, Karnet wygasł: }\PYG{l+s+si}{\PYGZob{}}\PYG{n}{row}\PYG{p}{[}\PYG{l+m+mi}{3}\PYG{p}{]}\PYG{l+s+si}{\PYGZcb{}}\PYG{l+s+s2}{\PYGZdq{}}\PYG{p}{)}

\PYG{k}{def}\PYG{+w}{ }\PYG{n+nf}{generuj\PYGZus{}raport\PYGZus{}sprzedazy\PYGZus{}sqlite}\PYG{p}{(}\PYG{n}{data\PYGZus{}od}\PYG{p}{,} \PYG{n}{data\PYGZus{}do}\PYG{p}{)}\PYG{p}{:}
\PYG{+w}{ }\PYG{l+s+sd}{\PYGZdq{}\PYGZdq{}\PYGZdq{}Oblicza sumę sprzedaży i liczbę sprzedanych karnetów w danym okresie.\PYGZdq{}\PYGZdq{}\PYGZdq{}}
 \PYG{n+nb}{print}\PYG{p}{(}\PYG{l+s+sa}{f}\PYG{l+s+s2}{\PYGZdq{}}\PYG{l+s+se}{\PYGZbs{}n}\PYG{l+s+s2}{\PYGZhy{}\PYGZhy{}\PYGZhy{} [SQLite] Raport sprzedaży od }\PYG{l+s+si}{\PYGZob{}}\PYG{n}{data\PYGZus{}od}\PYG{l+s+si}{\PYGZcb{}}\PYG{l+s+s2}{ do }\PYG{l+s+si}{\PYGZob{}}\PYG{n}{data\PYGZus{}do}\PYG{l+s+si}{\PYGZcb{}}\PYG{l+s+s2}{ \PYGZhy{}\PYGZhy{}\PYGZhy{}}\PYG{l+s+s2}{\PYGZdq{}}\PYG{p}{)}
 \PYG{n}{conn} \PYG{o}{=} \PYG{n}{get\PYGZus{}connection}\PYG{p}{(}\PYG{p}{)}
 \PYG{n}{query} \PYG{o}{=} \PYG{l+s+s2}{\PYGZdq{}}\PYG{l+s+s2}{SELECT COUNT(karnet\PYGZus{}id), SUM(cena) FROM Karnety WHERE data\PYGZus{}zakupu BETWEEN ? AND ?;}\PYG{l+s+s2}{\PYGZdq{}}
 \PYG{k}{with} \PYG{n}{conn}\PYG{p}{:}
     \PYG{n}{cur} \PYG{o}{=} \PYG{n}{conn}\PYG{o}{.}\PYG{n}{cursor}\PYG{p}{(}\PYG{p}{)}
     \PYG{n}{cur}\PYG{o}{.}\PYG{n}{execute}\PYG{p}{(}\PYG{n}{query}\PYG{p}{,} \PYG{p}{(}\PYG{n}{data\PYGZus{}od}\PYG{p}{,} \PYG{n}{data\PYGZus{}do}\PYG{p}{)}\PYG{p}{)}
     \PYG{n}{result} \PYG{o}{=} \PYG{n}{cur}\PYG{o}{.}\PYG{n}{fetchone}\PYG{p}{(}\PYG{p}{)}
     \PYG{n+nb}{print}\PYG{p}{(}\PYG{l+s+sa}{f}\PYG{l+s+s2}{\PYGZdq{}}\PYG{l+s+s2}{Liczba sprzedanych karnetów: }\PYG{l+s+si}{\PYGZob{}}\PYG{n}{result}\PYG{p}{[}\PYG{l+m+mi}{0}\PYG{p}{]}\PYG{+w}{ }\PYG{o+ow}{or}\PYG{+w}{ }\PYG{l+m+mi}{0}\PYG{l+s+si}{\PYGZcb{}}\PYG{l+s+s2}{\PYGZdq{}}\PYG{p}{)}
     \PYG{n+nb}{print}\PYG{p}{(}\PYG{l+s+sa}{f}\PYG{l+s+s2}{\PYGZdq{}}\PYG{l+s+s2}{Łączna kwota sprzedaży: }\PYG{l+s+si}{\PYGZob{}}\PYG{n}{result}\PYG{p}{[}\PYG{l+m+mi}{1}\PYG{p}{]}\PYG{+w}{ }\PYG{o+ow}{or}\PYG{+w}{ }\PYG{l+m+mf}{0.00}\PYG{l+s+si}{\PYGZcb{}}\PYG{l+s+s2}{ PLN}\PYG{l+s+s2}{\PYGZdq{}}\PYG{p}{)}

\PYG{k}{def}\PYG{+w}{ }\PYG{n+nf}{znajdz\PYGZus{}najaktywniejszych\PYGZus{}klientow\PYGZus{}sqlite}\PYG{p}{(}\PYG{n}{data\PYGZus{}od}\PYG{p}{,} \PYG{n}{data\PYGZus{}do}\PYG{p}{,} \PYG{n}{limit}\PYG{o}{=}\PYG{l+m+mi}{5}\PYG{p}{)}\PYG{p}{:}
\PYG{+w}{ }\PYG{l+s+sd}{\PYGZdq{}\PYGZdq{}\PYGZdq{}Wyświetla listę najczęściej wchodzących klientów w danym okresie.\PYGZdq{}\PYGZdq{}\PYGZdq{}}
 \PYG{n+nb}{print}\PYG{p}{(}\PYG{l+s+sa}{f}\PYG{l+s+s2}{\PYGZdq{}}\PYG{l+s+se}{\PYGZbs{}n}\PYG{l+s+s2}{\PYGZhy{}\PYGZhy{}\PYGZhy{} [SQLite] TOP }\PYG{l+s+si}{\PYGZob{}}\PYG{n}{limit}\PYG{l+s+si}{\PYGZcb{}}\PYG{l+s+s2}{ najaktywniejszych klientów od }\PYG{l+s+si}{\PYGZob{}}\PYG{n}{data\PYGZus{}od}\PYG{l+s+si}{\PYGZcb{}}\PYG{l+s+s2}{ do }\PYG{l+s+si}{\PYGZob{}}\PYG{n}{data\PYGZus{}do}\PYG{l+s+si}{\PYGZcb{}}\PYG{l+s+s2}{ \PYGZhy{}\PYGZhy{}\PYGZhy{}}\PYG{l+s+s2}{\PYGZdq{}}\PYG{p}{)}
 \PYG{n}{conn} \PYG{o}{=} \PYG{n}{get\PYGZus{}connection}\PYG{p}{(}\PYG{p}{)}
 \PYG{c+c1}{\PYGZsh{} Używamy funkcji DATE() do wyciągnięcia daty z pełnego timestampa}
 \PYG{n}{query} \PYG{o}{=} \PYG{l+s+s2}{\PYGZdq{}\PYGZdq{}\PYGZdq{}}
\PYG{l+s+s2}{ SELECT k.imie, k.nazwisko, COUNT(w.wejscie\PYGZus{}id) AS liczba\PYGZus{}wejsc}
\PYG{l+s+s2}{ FROM Wejscia w}
\PYG{l+s+s2}{ JOIN Klienci k ON w.klient\PYGZus{}id = k.klient\PYGZus{}id}
\PYG{l+s+s2}{ WHERE DATE(w.data\PYGZus{}wejscia) BETWEEN ? AND ?}
\PYG{l+s+s2}{ GROUP BY k.klient\PYGZus{}id}
\PYG{l+s+s2}{ ORDER BY liczba\PYGZus{}wejsc DESC}
\PYG{l+s+s2}{ LIMIT ?;}
\PYG{l+s+s2}{ }\PYG{l+s+s2}{\PYGZdq{}\PYGZdq{}\PYGZdq{}}
 \PYG{k}{with} \PYG{n}{conn}\PYG{p}{:}
     \PYG{n}{cur} \PYG{o}{=} \PYG{n}{conn}\PYG{o}{.}\PYG{n}{cursor}\PYG{p}{(}\PYG{p}{)}
     \PYG{n}{cur}\PYG{o}{.}\PYG{n}{execute}\PYG{p}{(}\PYG{n}{query}\PYG{p}{,} \PYG{p}{(}\PYG{n}{data\PYGZus{}od}\PYG{p}{,} \PYG{n}{data\PYGZus{}do}\PYG{p}{,} \PYG{n}{limit}\PYG{p}{)}\PYG{p}{)}
     \PYG{k}{for} \PYG{n}{row} \PYG{o+ow}{in} \PYG{n}{cur}\PYG{o}{.}\PYG{n}{fetchall}\PYG{p}{(}\PYG{p}{)}\PYG{p}{:}
         \PYG{n+nb}{print}\PYG{p}{(}\PYG{l+s+sa}{f}\PYG{l+s+s2}{\PYGZdq{}}\PYG{l+s+s2}{Klient: }\PYG{l+s+si}{\PYGZob{}}\PYG{n}{row}\PYG{p}{[}\PYG{l+m+mi}{0}\PYG{p}{]}\PYG{l+s+si}{\PYGZcb{}}\PYG{l+s+s2}{ }\PYG{l+s+si}{\PYGZob{}}\PYG{n}{row}\PYG{p}{[}\PYG{l+m+mi}{1}\PYG{p}{]}\PYG{l+s+si}{\PYGZcb{}}\PYG{l+s+s2}{, Liczba wejść: }\PYG{l+s+si}{\PYGZob{}}\PYG{n}{row}\PYG{p}{[}\PYG{l+m+mi}{2}\PYG{p}{]}\PYG{l+s+si}{\PYGZcb{}}\PYG{l+s+s2}{\PYGZdq{}}\PYG{p}{)}
\end{sphinxVerbatim}

\sphinxstepscope


\chapter{5. Podsumowanie, Wnioski i Repozytoria}
\label{\detokenize{rozdzial5/rozdzial5:podsumowanie-wnioski-i-repozytoria}}\label{\detokenize{rozdzial5/rozdzial5::doc}}

\section{Podsumowanie projektu}
\label{\detokenize{rozdzial5/rozdzial5:podsumowanie-projektu}}
\sphinxAtStartPar
Niniejszy projekt stanowił pełne ćwiczenie inżynierskie, obejmujące cały cykl życia systemu bazodanowego. Rozpoczynając od analizy potrzeb biznesowych fikcyjnej siłowni, poprzez staranne modelowanie danych, aż po fizyczną implementację i analizę aspektów niefunkcjonalnych, projekt ten z powodzeniem przełożył wymagania na działające, bezpieczne i wydajne rozwiązanie. Zastosowanie normalizacji zapewniło integralność danych, podczas gdy świadome planowanie indeksów i polityk bezpieczeństwa przygotowało system do działania w rzeczywistym środowisku.


\section{Wnioski}
\label{\detokenize{rozdzial5/rozdzial5:wnioski}}
\sphinxAtStartPar
Realizacja projektu pozwoliła na sformułowanie następujących wniosków:
\begin{enumerate}
\sphinxsetlistlabels{\arabic}{enumi}{enumii}{}{.}%
\item {} 
\sphinxAtStartPar
\sphinxstylestrong{Modelowanie jest kluczowe:} Czas poświęcony na staranne stworzenie modelu koncepcyjnego i logicznego procentuje na etapie implementacji i późniejszej konserwacji. Dobrze zaprojektowany schemat jest intuicyjny i łatwy do rozbudowy.

\item {} 
\sphinxAtStartPar
\sphinxstylestrong{Normalizacja to kompromis:} Chociaż dążenie do wyższych postaci normalnych jest teoretycznie pożądane, w praktyce należy uwzględniać również wydajność i logikę biznesową. Celowe, udokumentowane odstępstwa (jak przechowywanie ceny w momencie transakcji) są często uzasadnione.

\item {} 
\sphinxAtStartPar
\sphinxstylestrong{Wydajność i bezpieczeństwo nie są opcjonalne:} Aspekty te muszą być uwzględniane od samego początku procesu projektowego, a nie dodawane jako „łatki” na końcu. Strategiczne indeksowanie i model bezpieczeństwa oparty na rolach to fundamenty stabilnego systemu.

\item {} 
\sphinxAtStartPar
\sphinxstylestrong{Praktyka utrwala teorię:} Projekt ten był nieocenionym doświadczeniem, które pozwoliło na praktyczne zastosowanie i głębsze zrozumienie teoretycznych koncepcji omawianych na zajęciach, takich jak replikacja, partycjonowanie czy zaawansowane strategie backupu.

\end{enumerate}


\section{Możliwe kierunki dalszego rozwoju}
\label{\detokenize{rozdzial5/rozdzial5:mozliwe-kierunki-dalszego-rozwoju}}
\sphinxAtStartPar
Zaprojektowana baza danych stanowi solidny fundament, który można rozwijać w wielu kierunkach, aby zwiększyć jej wartość biznesową:
\begin{itemize}
\item {} 
\sphinxAtStartPar
\sphinxstylestrong{Moduł rezerwacji zajęć:} Dodanie tabel \sphinxtitleref{Zajecia}, \sphinxtitleref{Instruktorzy} oraz \sphinxtitleref{Rezerwacje} w celu umożliwienia klientom rezerwacji miejsc na zajęciach grupowych.

\item {} 
\sphinxAtStartPar
\sphinxstylestrong{Integracja z systemem płatności:} Połączenie z bramką płatniczą w celu automatyzacji sprzedaży karnetów online.

\item {} 
\sphinxAtStartPar
\sphinxstylestrong{Aplikacja kliencka:} Stworzenie aplikacji mobilnej lub webowej dla klientów, gdzie mogliby sprawdzać ważność swojego karnetu, historię wejść i rezerwować zajęcia.

\item {} 
\sphinxAtStartPar
\sphinxstylestrong{Zaawansowana analityka:} Budowa hurtowni danych i wykorzystanie narzędzi BI (Business Intelligence) do analizy trendów, np. godzin największego obłożenia siłowni, najpopularniejszych typów karnetów czy segmentacji klientów.

\end{itemize}


\section{Spis repozytoriów}
\label{\detokenize{rozdzial5/rozdzial5:spis-repozytoriow}}\begin{enumerate}
\sphinxsetlistlabels{\arabic}{enumi}{enumii}{}{.}%
\item {} 
\sphinxAtStartPar
\sphinxstylestrong{Repozytorium niniejszego sprawozdania:}
\sphinxurl{https://github.com/HoszeQ/karnety\_silownia\_sprawozdanie}

\item {} 
\sphinxAtStartPar
\sphinxstylestrong{Repozytorium projektu grupowego:}
\sphinxurl{https://github.com/m-smieja/Kopie\_zapasowe\_i\_odzyskiwanie\_danych}

\item {} 
\sphinxAtStartPar
\sphinxstylestrong{Repozytorium pracy pt. Konfiguracja\_baz\_danych:}
\sphinxurl{https://github.com/Chaiolites/Konfiguracja\_baz\_danych.git}

\item {} 
\sphinxAtStartPar
\sphinxstylestrong{Repozytorium pracy pt. Kontrola\_i\_konserwacja:}
\sphinxurl{https://github.com/Pi0trM/Kontrola\_i\_konserwacja.git}

\item {} 
\sphinxAtStartPar
\sphinxstylestrong{Repozytorium pracy pt. Partycjonowanie\sphinxhyphen{}danych:}
\sphinxurl{https://github.com/BartekHen/Partycjonowanie-danych.git}

\item {} 
\sphinxAtStartPar
\sphinxstylestrong{Repozytorium pracy pt. Sprzet\sphinxhyphen{}dla\sphinxhyphen{}bazy\sphinxhyphen{}danych:}
\sphinxurl{https://github.com/oszczeda/Sprzet-dla-bazy-danych.git}

\item {} 
\sphinxAtStartPar
\sphinxstylestrong{Repozytorium pracy pt. Wydajnosc\sphinxhyphen{}Skalowanie\sphinxhyphen{}i\sphinxhyphen{}Replikacja:}
\sphinxurl{https://github.com/Broksonn/Wydajnosc\_Skalowanie\_i\_Replikacja.git}

\item {} 
\sphinxAtStartPar
\sphinxstylestrong{Repozytorium pracy pt. Bezpieczenstwo:}
\sphinxurl{https://github.com/BlazejUl/bezpieczenstwo.git}

\item {} 
\sphinxAtStartPar
\sphinxstylestrong{Repozytorium pracy pt. Monitorowanie\sphinxhyphen{}i\sphinxhyphen{}diagnostyka:}
\sphinxurl{https://github.com/GrzegorzSzczepanek/repo-wspolne.git}

\end{enumerate}



\renewcommand{\indexname}{Indeks}
\printindex
\end{document}